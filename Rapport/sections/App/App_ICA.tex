\chapter{Independent Component Analysis}\label{app:ICA}
This appendix provide the basic theory of independent component analysis (ICA). The theory is necessary if one wants a deeper understanding towards the justification of applying result from ICA as a reference for evaluation of the main algorithm proposed in this thesis.   
The appendix concludes with an algorithm specifying ICA method which is applied in the thesis. Additionally, an verification test is conducted to evaluate the applied ICA method on the synthetic data, cf. \ref{sec:dataset}.   
\section{Independent Component Analysis}\label{sec:ICA}
%Independent component analysis (ICA) is a common method within signal processing and especially for sparse signal recovery. The intention of ICA is to separate a multivariate signal into statistical independent and non-Gaussian signals and furthermore $\textbf{X}$ and identify the mixing matrix $\textbf{A}$ given the observed measurements $\textbf{Y}$. \\
%The main aspect of ICA is to assume statistical independent between the wanted components and nongaussiantity of the data. \cite[p. 3]{ICA}. 
%\\
%Through this section the mathematical concepts of ICA will be explained and defined in the noise-less case.
%\\ \\
%To understand what ICA is let us describe a situation where ICA could be used. Two people are having a conversation inside a room full of people which talk simultaneous with each other. The conversation between the two first mentioned people will the be affected by the surrounding conversations and noise. Such environment is often referred to as the cocktail party problem and is a difficult environment to be in as a hearing impaired.
%\\
%Let us describe the situation with some mathematical notations. The observed conversation which is affected by surrounding noise is denoted $\mathbf{y}$, the original individual conversations in the room is denoted by $\mathbf{x}$. All the original conversations are effected by each other and possibly noise, let us denoted this mixture by a matrix $\mathbf{A}$. We omit the time indexes and the time dependence to view the problem with random vectors. The problem can then be described as a linear model
%\begin{align}
%\mathbf{y} &= \mathbf{Ax} = \sum_{i=1}^n \mathbf{a}_i x_i,  \label{eq:ICA_1} \\
%y_i &= a_{i1} x_1 + a_{i2} x_2 + \cdots + a_{in} x_n, \quad i = 1, \dots, n. \label{eq:ICA_2}
%\end{align}
%The only known variables in this model are the observed conversation $\mathbf{y}$, the rest are unknown. 
%%The observation $y_i (t)$ would instead be a sample from the random vector if needed.

Independent component analysis (ICA) is a method that applies to  the general problem of decomposition of a measurement vector into a source vector and a mixing matrix. The intention of ICA is to separate a multivariate signal into statistical independent and non-Gaussian signals and furthermore identify the mixing matrix $\textbf{A}$, given only the observed measurements $\textbf{Y}$.
A well known application example of source separation is the cocktail party problem, where it is sought to listen to one specific person speaking in a room full of people having interfering conversations. 
Let $\textbf{y}\in \mathbb{R}^{M}$ be a single measurement from $M$ microphones containing a linear mixture of all the speak signal that are present in the room. When additional noise is not considered the problem can be described as the familiar linear model, 
\begin{align}
\textbf{y}=\textbf{Ax}\label{eq:ICA1}
\end{align}
where $\textbf{x}\in \mathbb{R}^{N}$ contain the $N$ underlying speak signals and $\textbf{A}$ is a mixing matrix where the coefficients depends (more or less?) on the distance from the source to the microphone. As such each $y_i$ is a weighted sum of all the present sources of speak.
\\ \\
By ICA both the mixing matrix $\textbf{A}$ and the sources signals $\textbf{x}$ are sought estimated from the observed measurements $\textbf{y}$. The main attribute of ICA is the assumption that the sources in $\textbf{x}$ are statistical independent and non-Gaussian distributed, hence the name independent components\todo{when we assume independence it is enough to solve system, why?}.\\     
By independence, one means that changes in one source signal do not affect the other source signals. Theoretically that is the joint probability density function (pdf) of $\textbf{x}$ can be factorised into the product of the marginal pdfs of the components $x_i$
\begin{align*}
p(x_1, x_2, \dots, x_n) = p_1 (x_1) p_2(x_2) \cdots p_n(x_n),
\end{align*}

%Furthermore it is assumed that the independent components $\mathbf{x}$ do not have Gaussian distribution as this will effect the recovery of $\mathbf{x}$ due to ICA using higher-order cumulant method. For Gaussian distribution the cumulant will be zero which is not wanted in the recovery process. Thus the distribution of the independent components are unknown. This will further be described in section \ref{sec:est_ica}.

\todo{herover er ukommenteret et afsnit jeg ikke forstår}The possibility of separating a signal into independent and non-Gaussian components originates from the central limit theorem\cite[p. 34]{ICA}. The theorem state that the distribution any linear mixture of two or more independent random variables tents toward a Gaussian distribution, under certain conditions. Thus, when a non-Gaussian distribution of the independent components is achieved through optimization it must be the original sources.

\subsection{Assumptions and Preprocessing}
For simplicity assume $\textbf{A}$ is square i.e. $M=N$ and invertible. As such when $\textbf{A}$ has been estimated the inverse is computed the components can simply be estimated as $\textbf{x}=\textbf{A}^{-1}\textbf{y}$\cite[p. 152-153]{ICA}.
\\ 
As both $\textbf{A}$ and $\textbf{x}$ are unknown the variances of the independent components can not be determined. However it is reasonable to assume that $\mathbf{x}$ has unit variance, as $\textbf{A}$ will adapt to this restriction. Any scalar multiplier within a source can be cancelled out by dividing the corresponding column in $\textbf{A}$ with the same scala\cite[p. 154]{ICA}.\\  
For further simplification it is assumed without loss of generality that $\mathbb{E}[\mathbf{y}] = 0$ and $\mathbb{E}[\mathbf{x}] = 0$\cite[p. 154]{ICA}. In case this assumption is not true, the measurements can be centred by subtracting the mean as preprocessing before doing ICA.\\
A preprocessing step central to ICA is to whiten\todo{whiteining is a linear change of coordinates of the mixed data\url{http://arnauddelorme.com/ica_for_dummies/} "By rotating the axis and minimizing Gaussianity of the projection in the first scatter plot, ICA is able to recover the original sources which are statistically independent} the measurements $\mathbf{y}$. By the whitening process any correlation in the measurements are removed and unit variance is ensured. This ensures that the independent components $\mathbf{x}$ are uncorrelated and have unit variance(true?). Furthermore, this reduces the complexity of ICA and therefore simplifies the recovering process.
\\
Whitening is a linear transformation of the observed data. That is multiplying the measurement vector $\textbf{y}$ with a whitening matrix $\textbf{V}$,
\begin{align*}
\textbf{y}_{white} = \textbf{V}\textbf{y}
\end{align*} 
to obtain a new measurement vector $\textbf{y}_{white}$ that is white. To obtain a whitening matrix the eigenvalue decomposition (EVD) of the covariance matrix can be used,
\begin{align*}
\mathbb{E}[\mathbf{yy}^T] = \mathbf{EDE}^T
\end{align*}
here $\mathbf{D}$ is a diagonal matrix of eigenvalues and $\mathbf{E}$ is the associated eigenvectors. From $\mathbf{E}$ and $\mathbf{D}$ a whitening matrix is constructed \cite[p.159]{ICA}.
\begin{align*}
\mathbf{V} = \mathbf{ED}^{-1/2} \mathbf{E}^T.
\end{align*}
Where $\mathbf{D}^{-1/2}=diag{d_1^{-1/2},\hdots,d_n^{-1/2}}$ is a componentwise operation.\\  
By multiplying the measurement vector $\mathbf{y}$ with a whitening matrix $\mathbf{V}$ the data becomes white
\begin{align}
\mathbf{y}_{\text{white}} &= \mathbf{Vy} = \mathbf{VAx} = \mathbf{A}_{\text{white}} \mathbf{x} \nonumber
\end{align}
Furthermore the mixing matrix $\mathbf{A}_{\text{white}}$ becomes orthogonal 
\begin{align*}
 \mathbb{E}[\mathbf{y}_{\text{white}} \mathbf{y}_{\text{white}}^T] = \mathbf{A}_{\text{white}} \mathbb{E}[\mathbf{xx}^T] \mathbf{A}_{\text{white}}^T = \mathbf{A}_{\text{white}} \mathbf{A}_{\text{white}}^T = \mathbf{I}.
 \end{align*} 
Consequently ICA can restrict its search for the mixing matrix to the orthogonal matrix space -- That is instead of estimating $n^2$ parameters ICA now only has to estimate an orthogonal matrix which has $n(n-1)/2$ parameters/degrees of freedom \cite[p. 159]{ICA}
\todo{se udkommentering herunder?}.
%%%%% forstår ikke det sidste her
%Furthermore, to support the assumption of nongaussiantity for an orthogonal mixing matrix $\mathbf{A}_{\text{white}}$ it is not possible to distinct the pdfs of the $\mathbf{y}_{\text{white}}$ and $\mathbf{x}$ as $\mathbf{A}_{\text{white}}$ is no longer included in the pdf of $\mathbf{y}_{\text{white}}$ and therefore are the two pdfs equal \cite[p. 161-163]{ICA}\todo{can we elaborate this last part}.
% evt. læs igen side 162 nederest ICA bog, gives why the moel is well defined

\subsection{Recovery of the Independent Components}\label{sec:est_ica}
Now the ICA model is established, the next step is the estimation of the mixing coefficients $a_{ij}$ and independent components $x_i$. The simple and intuitive method is to take advantage of the assumption of non-Gaussian independent components. Consider again the ICA model of a single measurement vector $\textbf{y}=\textbf{Ax}$ where the independent components can be estimated by the inverted model $\textbf{x}=\textbf{A}^{-1}\textbf{y}$. Let $\textbf{A}^{-1}=\textbf{B}$, now a single independent component can be seen as the linear combination 
\begin{align}
x_j = \textbf{b}_{j}^T \textbf{y} = \sum_{i}b_iy_i \label{eq:ICA_comp}
\end{align} 
where $\textbf{b}_{j}^T$ is the $j^{th}$ row of $\textbf{B}$. The issue is now to determined $\textbf{b}_j$ such that it equals the $j^{th}$ row from the inverse $\textbf{A}$. As $\textbf{A}$ is unknown it is not possible to determine $\textbf{b}_j$ exactly, but an estimate can be found to make a good approximation. Rewriting \eqref{eq:ICA_comp} 
\begin{align*}
x_j = \textbf{b}_{j}^T \textbf{y} = \textbf{b}_{j}^T \textbf{Ax} = \textbf{q}^T\textbf{x} = \sum_{i=1}q_i x_i
\end{align*}
it is seen how $x_j$ is a linear combination of all $\textbf{x}_i$, thus the equality only holds true when $\textbf{q}$ consist of only one non-zero element that equals 1.  
Due to the central limit theorem the distribution of $\textbf{q}^T\textbf{x}$ most non-Gaussian when it equals one of the independent components which was assumed non-Guassian. Then, since $\textbf{q}^T\textbf{x} = \textbf{b}_{j}^T \textbf{y}$, it is possible to vary the coefficients in $\textbf{b}$ and look at the distribution of $\textbf{b}_{j}^T \textbf{y}$. Finding the vector $\textbf{b}$ that maximize the non-Gaussianity would then corresponds to $\textbf{q}=\textbf{A}^T\textbf{b}$ having only a single non-zero element. Thus maximizing the non-Gaussianity of $\textbf{b}_{j}^T \textbf{y}$ results in one of the independent components\cite[p. 166]{ICA}. \\
Considering the n-dimensional space of vectors $\textbf{b}$ there exist $2n$ local maxima, corresponding to $x_i$ and $-x_i$ for all $n$ independent components\cite[p. 166]{ICA}. 

\subsection{Kurtosis}
To maximize the non-gaussianity a measure for gaussianity is needed. Kurtosis is a quantitative measure used for nongaussianity of random variables. Kurtosis of a random variable $y$ is the fourth-order cumulant denotet by $\text{kurt}(y)$. For $y$ with zero mean and unit variance kurtosis reduces to \todo{uddyb? og tilføj kilde til kurtosis}
\begin{align*}
\text{kurt} (y) = \mathbb{E}[y^4] - 3.
\end{align*} 
It is seen that the kurtosis is a normalized version of the fourth-order moment defined as $\mathbb{E}[y^4]$. 
For a Gaussian random variable the fourth-order moment equals $3(\mathbb{E}[y^2])^2$ hence the corresponding kurtosis will be zero \cite[p. 171]{ICA}. Consequently the kurtosis of non-Gaussian random variables will almost always be different from zero.\\
The kurtosis is a common measure for non-Gaussianity due to its simplicity  both theoretical and computational. The kurtosis can be estimated computationally by the forth-order moment of sample data when the variance is constant.
Furthermore, for two independent random variables $x_1, x_2$ the following linear properties applies to the kurtosis of the sum
\begin{align*}
\text{kurt}(x_1 + x_2)=\text{kurt}(x_1)+ \text{kurt}(x_2) \quad \text{and} \quad \text{kurt}(\alpha x_1) = \alpha^4 \text{kurt}(x_1)
\end{align*}  
However, one complication concerning kurtosis as a measure is that kurtosis is sensitive to outliers \cite[p. 182]{ICA}.\\ 
\\      
Consider again the vector $\textbf{q} = \textbf{A}^T\textbf{b}$ such that $\textbf{b}_{j}^T \textbf{y} = \sum_{i=1}q_ix_i$. 
By the additive property of kurtosis
\begin{align*}
\text{kurt}\left( \textbf{b}_{j}^T \textbf{y}\right) = \sum_{i=1}q_i^4\text{kurt}(x_i).
\end{align*}
Then the assumption of the independent components having unit variance results in $\mathbb{E}[x_j]= \sum_{i=1}q_i^2=1$\todo{hvordan kommer dette frem?}. That is geometrically that $\textbf{q}$ is constraint to the unit sphere, $\| \textbf{q}\|^2 = 1$. By this the optimisation problem of maximising the kurtosis of $\textbf{b}_{j}^T \textbf{y}$ is similar to maximizing $\vert \text{kurt}(x_j)\vert = \vert \sum_{i=1}q_i^4\text{kurt}(x_i) \vert $ on the unit sphere.\\
Due to the described preprocessing $\textbf{b}$ is assumed to be white and it can be shown that $\Vert \textbf{q} \Vert = \Vert \textbf{b}_j \Vert $\cite[p. 174]{ICA}. This show that constraining   $\Vert \textbf{q} \Vert$ to one is similar to constraining $\Vert \textbf{b}_j \Vert$ to one. 

\subsection{The Gradient Algorithm with Kurtosis}\label{sec:gra_kur}
In practise, to recover the mixing matrix $\textbf{A}$ by maximizing the kurtosis of $\textbf{b}_{j}^T \textbf{y}$, gradient optimisation methods are used.\\
The general idea behind a gradient algorithm is to determine the direction for which $\text{kurt}(\textbf{b}_{j}^T \textbf{y})$ is growing the most, based on the gradient. \\

The gradient of $\vert \text{kurt}(\textbf{b}_{j}^T \textbf{y}) \vert$ is computed as
\begin{align}\label{eq:kurt}
\frac{\partial \vert \text{kurt}(\textbf{b}_{j}^T \textbf{y})\vert}{\partial \mathbf{b}_j} &= 4 \text{sign}(\text{kurt}(\textbf{b}_{j}^T \textbf{y})) (\mathbb{E}[\mathbf{y} (\textbf{b}_{j}^T \textbf{y})^3] - 3 \mathbf{y} \mathbb{E}[(\textbf{b}_{j}^T \textbf{y})^2]) 
\end{align} 
As $\mathbb{E}[(\textbf{b}_{j}^T \textbf{y})^2] =\Vert \mathbf{y} \Vert^2)$ for whitened data the corresponding term does only affect the norm of $\textbf{b}_j$ within the gradient algorithm. Thus, as it is only the direction that is of interest, this term can be omitted. Because the optimisation is restricted to the unit sphere a projection of $\textbf{b}_j$ onto the unit sphere must be performed in every step of the gradient method. This is done by dividing $\textbf{b}_j$ by its norm. This gives update step 
\begin{align*}
\Delta &\textbf{b}_j \propto \text{sign}\left( \text{kurt}(\textbf{b}_{j}^T \textbf{y}) \right) \mathbb{E}[\textbf{y}(\textbf{b}_{j}^T \textbf{y})^3] \\
\textbf{b}_j &\leftarrow \textbf{b}_j/\Vert \textbf{b}_j \Vert
\end{align*}  
The expectation operator can be omitted in order to achieve an adaptive version of the algorithm, now using every measurement $\textbf{y}$. However, the expectation operator from the definition of kurtosis can not be omitted and must therefore be estimated. This can be done by a time-average estimate, denoted as $\gamma$ and serving as the learning rate of the gradient method.
\begin{align*}
\Delta \gamma \propto((\textbf{b}_{j}^T \textbf{y})^4 - 3) - \gamma
\end{align*}

\subsection{Basic ICA algorithm} 
Algorithm \ref{alg:basicICA} combined the above theory, to give an overview of the ICA procedure. Estimating the mixing matrix and the corresponding independent components, from the given measurements. 
\begin{algorithm}[H]
\caption{Basis ICA}
\begin{algorithmic}[1]
			\Procedure{Pre-processing}{$\textbf{y}$}
			\State $\text{Center measurements} \quad \textbf{y} \gets \textbf{y} - \bar{\textbf{y}}$
			\State $\text{Whitening} \quad \textbf{y}\gets \textbf{y}_{white}$ 
			\EndProcedure  
			\State
            \Procedure{ICA}{$\textbf{y}$}    
			\State$k=0$            
            \State$\text{Initialise random vector} \quad \textbf{b}_{j(k)}$ \Comment{unit norm}
            \State$\text{Initialise random value} \quad \gamma_{(k)}$
            \For{$j \gets 1,2, \hdots ,N$ }
            
            	\While{$\text{convergance critia not meet}$} 
               		\State $k = k+1$
                	\State $\textbf{b}_{j(k)} \gets \text{sign}\gamma_{(k-1)} \textbf{y}(\textbf{b}_{j}^T \textbf{y})^3$
                	\State $\textbf{b}_{j(k)} \gets \textbf{b}_j/\Vert \textbf{b}_j \Vert $ 
                	\State $\gamma_{(k)} \gets ((\textbf{b}_{j}^T \textbf{y})^4 - 3) - \gamma_{(k-1)} $
          		\EndWhile
          		\State $x_{j} = \textbf{b}_{j}^T\textbf{y}$
          	\EndFor
          	
            \EndProcedure
        \end{algorithmic} 
        \label{alg:basicICA}
\end{algorithm}

\subsection{ICA for sparse signal recovery}
ICA is widely used within sparse signal recovery.   
When ICA is applied to a measurement vector $\textbf{y}\in\mathbb{R}^{M}$ it is possible to separate the mixed signal into $M$ or less independent components. However, by assuming that the  independent components makes a $k$-sparse signal it is possible to apply ICA within sparse signal recovery of cases where $M < N$ and $k \leq M$. \\
To apply ICA to such case the independent components are obtained by the pseudo-inverse solution 
\begin{align*}
\hat{\mathbf{x}} = \mathbf{A}_S^{\dagger} \mathbf{y}
\end{align*}
where $\mathbf{A}_S$ is derived from the dictionary matrix $\mathbf{A}$ by containing only the columns associated with the non-zero entries of $\textbf{x}$, specified by the support set $S$. Extension to the basic ICA algorithm can be found in appendix \ref{app:ICA}. 

 

 

\section{Fixed-Point Algorithm - FastICA}
An advantage of gradient algorithms is the possibility of fast adoption in non-stationary environments due the use of all input, $\textbf{y}$, at once. A disadvantage of the gradient algorithm is the resulting slow convergence, depending on the choice of $\gamma$ for which a bad choice in practise can disable convergence. A fixed-point iteration algorithm to maximise the non-Gaussianity is an alternative that could be used.\\
Consider the gradient step derived in section \ref{sec:gra_kur}.
In the fixed point iteration the sequence of $\gamma$ is omitted and replaced by a constant. This builds upon the fact that for a stable point of the gradient algorithm the gradient must point in the direction of $\textbf{b}_j$, hence be equal to $\textbf{b}_j$. In this case adding the gradient to $\textbf{b}_j$ does not change the direction and convergence is achieved.\\    
Letting the gradient given in \eqref{eq:kurt} be equal to $\mathbf{w}$ and  considering the same simplifications again
suggests the new update step as \cite[p. 179]{ICA}
\begin{align*}
\mathbf{b}_j \gets \mathbb{E}[\mathbf{y}(\textbf{b}_{j}^T \textbf{y})^3] - 3 \mathbf{b}_j.
\end{align*}
After the fixed point iteration $\textbf{b}_j$ is again divided by its norm to withhold the constraint $\Vert \textbf{b}_j \Vert = 1$.   
Instead of $\gamma$ the fixed-point algorithm compute $\mathbf{b}_j$ directly from previous $\mathbf{b}_j$.\\
The fixed-point algorithm is referred to as FastICA. The algorithm has shown to converge fast and reliably, then the current and previous $\mathbf{w}$ laid in the same direction \cite[p. 179]{ICA}. 

\subsection{Negentropy}
An alternative measure of non-Gaussianity is the negentropy, which is based on the differential entropy. The differential entropy $H$ of a random vector $\mathbf{y}$ with density $p_y (\boldsymbol{\eta})$ is defined as
\begin{align*}
H(\mathbf{y}) = - \int p_y (\boldsymbol{\eta}) \log (p_y (\boldsymbol{\eta})) \ d\boldsymbol{\eta}.
\end{align*}
The entropy describes the information that a random variable gives. The more unpredictable and unstructured a random variable is higher is the entropy, e.g. Gaussian random variables have a high entropy, in fact the highest entropy among the random variables of the same variance \cite[p. 182]{ICA}.
\\ \\
Negentropy is a normalised version of the differential entropy such that the measure of non-Gaussianity is zero when the random variable is Gaussian and non-negative otherwise. The negentropy $J$ of a random vector $\mathbf{y}$ is defined as 
\begin{align*}
J(\mathbf{y}) = H(\mathbf{y}_{\text{gaus}}) - H(\mathbf{y}),
\end{align*}
with $\mathbf{y}_{\text{gaus}}$ being a Gaussian random variable of the same covariance and correlation as $\mathbf{y}$ \cite[p. 182]{ICA}.
\\ \\
As the kurtosis is sensitive for outliers the negentropy is instead difficult to compute computationally as the negentropy require a estimate of the pdf. As such an approximation of the negentropy is needed.\\

To approximate the negentropy it is common to use the higher order comulants including the kurtosis. The following approximation is stated without further elaboration, the derivation can be found in \cite[p. 182]{ICA}. 

\subsection{Fixed-Point Algorithm with Negentropy}
Maximization of negentropy by use of the fixed-point algorithm is now presented, for derivation of the fixed point iteration see \cite[p. 188]{ICA}. Algorithm \ref{alg:fastICA} show Fast ICA using negentropy, this is the algorithm which is implemented for comparison with the source separation methods which are tested in this thesis.    

\begin{algorithm}[H]
\caption{Fast ICA -- with negentropy }
\begin{algorithmic}[1]
			\Procedure{Pre-processing}{$\textbf{y}$}
			\State $\text{Center measurements} \quad \textbf{y} \gets \textbf{y} - \bar{\textbf{y}}$
			\State $\text{Whitening} \quad \textbf{y}\gets \textbf{y}_{white}$ 
			\EndProcedure  
			\State
            \Procedure{FastICA}{$\textbf{y}$}    
			\State$k=0$            
            \State$\text{Initialise random vector} \quad \textbf{b}_{j(k)}$ \Comment{unit norm}
            \For{$j \gets 1,2, \hdots ,N$ }
            
            	\While{$\text{convergance critia not meet}$} 
               		\State $k = k+1$
                	\State $\textbf{b}_{j(k)} \gets \mathbb{E}[ \textbf{y}(\textbf{b}_{j}^T \textbf{y})] - \mathbb{E}[g'(\textbf{b}_{j}^T \textbf{y})] \textbf{b}_{j}$ \Comment{$g$ defined in \cite[p. 190]{ICA}} 
                	\State $\textbf{b}_{j(k)} \gets \textbf{b}_j/\Vert \textbf{b}_j \Vert $ 
          		\EndWhile
          		\State $x_{j} = \textbf{b}_{j}^T\textbf{y}$
          	\EndFor
          	
            \EndProcedure
        \end{algorithmic} 
        \label{alg:fastICA}
\end{algorithm}


%\section{OMP}
%$\mathbf{A}^\ast$ is the adjoint of a matrix $\mathbf{A}$.
%\begin{algorithm}[H]
%\caption{Orthogonal Matching Pursuit (OMP)}\label{alg:OMP}
%\begin{algorithmic}[1]
%			\State$k = 0$			
%			\State$\text{Initialize} \quad S_{(0)} =\emptyset$ 
%			\State$\text{Initialize} \quad x_{(0)} =\mathbf{0}$
%            \Procedure{OMP}{$\mathbf{A}, \mathbf{y}$}    
%            \While{$\text{stopping criteria not meet}$} 
%                \State$k = k + 1$
%				\State$j_{(k)} = \arg \max_{j \in [N]} \lbrace \vert (\mathbf{A}^\ast (\mathbf{y} - \mathbf{Ax}_{(k-1)}))_j \vert \rbrace$				
%				\State$S_{(k)} = S_{(k-1)} \cup \lbrace j_{(k)} \rbrace$ 
%				\State$\mathbf{x}_{(k)} = \arg \min_{\mathbf{z} \in \mathbb{C}^N} \lbrace \Vert \mathbf{y} - \mathbf{Az} \Vert_2 \  \vert \ \text{supp}(\mathbf{z}) \subset S_{(k)} \rbrace$
%          		\EndWhile
%          		\State$\mathbf{x}^\ast = \mathbf{x}_{(k)}$
%            \EndProcedure
%        \end{algorithmic} 
%        \label{alg:OMP}
%\end{algorithm}

\section{Verification of fast ICA on synthetic data}\label{app:ica_test}
The purpose of this section is to verify the fast ICA algorithm which is used in this thesis. By this verification the purpose is to justify the ICA algorithm as a reference point with respect to performance of the developed main algorithm.

The fast ICA algorithm is tested on synthetic data simulated as described in section \ref{sec:dataset}. 
Consider the following linear system, which makes a model of EEG measurements.  
\begin{align*}
\textbf{Y}=\textbf{AX}
\end{align*}
where $\textbf{Y}^{M\times L}$, $\textbf{A}^{M\times N}$ and $\textbf{x}^{N\times L}$. It is expected that the fast ICA algorithm manage to solve the linear system for $\textbf{X}$ and $\textbf{A}$ given only the measurements $\textbf{Y}$, in the case where $M=N$.  

The fast ICA algorithm is applied to $\textbf{Y}$ and returns the estimates $\hat{\textbf{X}}_{ICA}$ and $\hat{\textbf{A}}_{ICA}$. 
When using the fast ICA algorithm the output $\hat{\textbf{X}}_{ICA}$ do not correspond one to one with the true source signal, which become an issue when the estimation error is measured by the mean squared error (MSE) cf. subsection \ref{sec:mse}.  The fast ICA algorithm is invariant towards the amplitude and phase of the signal and furthermore the rows are not necessarily place the original location. 
In order to get a valid MSE measure of the estimate, a function is defined to fit the estimate to the true source signal $\textbf{X}$. The function manage to pair the rows and change the phase, such that the total MSE is minimized. Furthermore each row of the estimate is scaled by the relationship between the  maximum value between the true row and the estimated row.
From empirical observations only the phase shift performed by multiplying with $(-1)$ has shown necessary, hence it is easily applied to the fitting function.
When the fitting function is applied two the estimate, the full potential of the fast ICA algorithm is considered reached.       

Figure \ref{fig:appica1} illustrates $\hat{\textbf{X}}_{ICA}$, without used of the fitting function, resulting from the fast ICA algorithm applied to a simulated deterministic data set $\textbf{Y}$ specified by $M=N=k=4$ and $L=1000$. In the figure $\textbf{Y}$, $\textbf{X}$ and $\hat{\textbf{X}}_{ICA}$ are plotted separately and it is clear to see the invariance towards amplitude and phase.
The MSE from the original $\hat{\textbf{X}}_{ICA}$ becomes
\begin{align*}
MSE(\textbf{X},\hat{\textbf{X}}_{ICA}) = 0.608.
\end{align*}

In figure \ref{fig:appica1} the fitting function has been applied to $\hat{\textbf{X}}_{ICA}$. Each row of the fitted estimate is now plotted with the corresponding row of the true source signals. 
The resulting MSE becomes 
\begin{align*}
MSE(\textbf{X},\hat{\textbf{X}}_{ICA}) = 0.046.
\end{align*}  
that is an essential change from the first measured MSE, and it is considered to provide at more valid measure of the estimate. From the visualisation and the corresponding MSE it is found that the fast ICA algorithm manage to estimate the source signals of the deterministic data set with a sufficiently small error\todo{inset captions og aling figure}. 
\begin{figure}[H]
    \begin{minipage}[t]{.45\textwidth}
		\centering
		\includegraphics[scale=0.5]{figures/ICAapp/ICA_app1.png}
	\caption{}
	\label{fig:appica1}
    \end{minipage} 
    \hfill
    \begin{minipage}[t]{.45\textwidth}
		\centering
		\includegraphics[scale=0.5]{figures/ICAapp/ICA_app2.png}
	\caption{}
	\label{fig:appica2}
    \end{minipage}
\end{figure}

A similar test is now performed on a stochastic data set  $\textbf{Y}$, cf. section \ref{sec:stoch_data}, again specified by $M=N=k=4$ and $L=1000$. 
Figure \ref{fig:appica3} show the comparison of the fitted $\hat{\textbf{X}}_{ICA}$ and the true source signals $\textbf{X}$. Note that only the first 100 samples are plotted for easier visualization. The resulting MSE becomes:
\begin{align*}
MSE(\textbf{X},\hat{\textbf{X}}_{ICA}) = 0.037.
\end{align*}       
Again the the MSE is considered sufficiently small and by that the fast ICA is considered verified with respect to solving a linear system with $M=N$. 
\begin{figure}[H]
	\centering
	\includegraphics[scale=0.5]{figures/ICAapp/ICA_app3.png}
	\caption{}
	\label{fig:appica3}
\end{figure}

Consider now the case where $k \leq N=M$, that is the sources signal matrix has $k$ non-zeros rows. 
The fast ICA algorithm is now applied to the a stochastic data set $\textbf{Y} $ specified by $N=M=6$, $k = 4$ and $L=1000$. Figure \ref{fig:appica5} and \ref{fig:appica6} show the comparison between the resulting $\hat{\textbf{X}}_{ICA}$ before and after the application of the fitting function, respectively. The resulting MSE becomes:
\begin{align*}
MSE(\textbf{X},\hat{\textbf{X}}_{ICA}) = 1.784.
\end{align*} 
It is seen from figure \ref{fig:appica6} that the fast ICA algorithm manage to  detect the zero rows of $\textbf{X}$. Without further test, this indicates the possibility of estimating $k$ from the fast ICA algorithm. 
\begin{figure}[H]
    \begin{minipage}[t]{.45\textwidth}
		\centering
		\includegraphics[scale=0.5]{figures/ICAapp/ICA_app5.png}
	\caption{}
	\label{fig:appica5}
    \end{minipage} 
    \hfill
    \begin{minipage}[t]{.45\textwidth}
		\centering
		\includegraphics[scale=0.5]{figures/ICAapp/ICA_app6.png}
	\caption{}
	\label{fig:appica6}
    \end{minipage}
\end{figure}

With these tests the quality of the fast ICA algorithm has been verified. As such the fast ICA algorithm can be used as a reference, when applied to real EEG data. It is further established that $k\leq N$ can be estimated by fast ICA. 

Remember though that the ICA estimate is conditioned under $k\leq N=M$, however this condition is not certain for real EEG data as the true $N$ is always unknown.   











  


