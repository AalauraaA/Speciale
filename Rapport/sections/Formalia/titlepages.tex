\pdfbookmark[0]{English title page}{label:titlepage_en}
\aautitlepage{%
  \englishprojectinfo{
    Sparse Bayesian Learning for EEG Source Recovery %title
  }{%
     %theme
     Mathematical modelling of EEG measurements for blind source recovery based on existing methods 
  }{%
    Fall Semester 2019 \\ %project period
    Spring Semester 2020
  }{%
    Mattek10b % project group
  }{%
    %list of group members
    Trine Nyholm Kragh\\ 
    Laura Nyrup Mogensen
  }{%
    %list of supervisors
    Jan Østergaard\\
    Rasmus Waagepetersen
    
  }{%
    1 % number of printed copies
  }{%
    \today % date of completion
  }%
}{%department and address
  \textbf{Mathematical Engineering}\\
  Aalborg University\\
  \href{http://www.aau.dk}{http://www.aau.dk}
}{% the abstract
This thesis treats the problem of recovering original brain source signals from low-density EEG scalp measurements. 
Based on state of the art methods, an algorithm is proposed to reproduce the current results.
The algorithm leverage a covariance-domain dictionary learning (Cov-DL) method and a multiple sparse Bayesian learning (M-SBL) method. The proposed application of Cov-DL did not succeed. 
Thus, an alternative solution was proposed. 
The final algorithm was tested on EEG and compared to solutions obtained by independent component analysis of high-density EEG. 
A frequency analysis was performed comparing raw EEG and the recovered sources. 
With respect to practical use an estimation of the unknown number of active source signals was proposed. 
It is concluded that the proposed algorithm is not able to  reproduce the state of the art results. 
However, the M-SBL method alone is successful and a potential is seen for an estimation of the number of active sources, from M-SBL. 
}
%
%\cleardoublepage
%{\selectlanguage{danish}
%\pdfbookmark[0]{Danish title page}{label:titlepage_da}
%\aautitlepage{%
%  \danishprojectinfo{
%    Bayesian Bibliotek Læring for EEG Kilde Identifikation %title
%  }{%
%     %theme
%  }{%
%    Efterårssemestret 2019 \\ %project period
%	Forårssemestret 2020
%  }{%
%    Mattek9b % project group
%  }{%
%    %list of group members
%    Trine Nyholm Kragh\\ 
%    Laura Nyrup Mogensen
%  }{%
%    %list of supervisors
%    Jan Østergaard \\
%    Rasmus Waagepetersen
%  }{%
%    1 % number of printed copies
%  }{%
%    \today % date of completion
%  }%
%}{%department and address
%  \textbf{Matematik-Teknologi}\\
%  Aalborg Universitet\\
%  \href{http://www.aau.dk}{http://www.aau.dk}
%}{% the abstract
%  Her er resuméet
%}}
