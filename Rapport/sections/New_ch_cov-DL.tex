\chapter{Covariance-Domain Dictionary Learning}\label{ch:Cov-DL}
Through this chapter the method ''covariance-domain dictionary learning (Cov-DL)'' is presented. Along the presentation of the general method, necessary computational details are derived for the practical solution.
The purpose is to recover the mixing matrix $\textbf{A}$ from the MMV model, derived in chapter \ref{ch:system_model}, in the over-determined case.   

Cov-DL is an algorithm proposed by O. Balkan \cite{Balkan2015}, leveraging the increased dimensionality of the covariance domain. The method have shown successful recovering of the mixing matrix $\textbf{A}$, even in the over-determined case with more active sources $k$ than available measurements $M$, $k \geq M$. 
In short the algorithm consist of three steps. 
First the segmented MMV model of the EEG measurements is transformed into the covariance domain. Then, by the increased dimensionality of the covariance domain, it is possible to learn the mixing matrix of the covariance domain, denoted by $\textbf{D}$, based on the theory of compressive sensing. Here two different cases will appear dependent on the relation between the number of sources $N$ and the found dimension of the covariance domain, which of course depends on the number of measurements $M$. Lastly, an inverse transformation is performed on the found mixing matrix of the covariance domain $\textbf{D}$, in order to obtain the wanted mixing matrix $\textbf{A}$. 
An important aspect of this method is the prior assumption that the sources within one segment are uncorrelated, that is the rows of $\textbf{X}_s$ being mutually uncorrelated. 
%Notice that the Cov-DL algorithm do only recover the mixing matrix $\mathbf{A}$ given the measurements $\textbf{Y}_s$. Given $\textbf{A}$ the non-segmented source matrix $\mathbf{X}$ is to be recovered by use of the Multiple Sparse Bayesian Learning algorithm, this is described in chapter \ref{ch:M-SBL} 

The section is inspired by chapter 3 in \cite{phd2015} and the article \cite{Balkan2015}. Selected general theory supporting essential parts of the method is elaborated in appendix \ref{app:Cov-DL}.

\section{Covariance Domain Representation}\label{sec:cov}
Consider a single sample vector $\textbf{y}_i\in \mathbb{R}^{M}$, containing EEG measurements. 
The covariance of $\textbf{y}_i$ is defined as
\begin{align*}
\boldsymbol{\Sigma}_{\textbf{y}_i}=\mathbb{E}[(\textbf{y}_i-\mathbb{E}[\textbf{y}_i])(\textbf{y}_i-\mathbb{E}[\textbf{y}_i])^T],
\end{align*}
where $\mathbb{E}[\cdot]$ is the expected value operator. 
Let $\textbf{Y}_{s}=\left[\textbf{y}_1, \hdots ,\textbf{y}_{L_s}\right]$ be the observed measurements matrix containing all samples of segment $s$.
Furthermore, assume that all sample vectors $\textbf{y}_i$ within one segment has zero mean and the same distribution.  
Then $\mathbf{Y}_s \in \mathbb{R}^{M \times L_s}$ is to be described in the covariance domain by the sample covariance $\widehat{\boldsymbol{\Sigma}}$. The sample covariance $\widehat{\boldsymbol{\Sigma}}$ is defined as the empirical covariance among the $M$ measurements across the $L_s$ samples. That is a $M \times M$ matrix $\widehat{\boldsymbol{\Sigma}}_{\mathbf{Y}_s} = [\sigma_{jk}]$ with entries 
\begin{align*}
\sigma_{jk}= \frac{1}{L_s}\sum_{i=1}^{L_s} y_{ji} y_{ki}.
\end{align*}
Using matrix notation the sample covariance of $\mathbf{Y}_s$ can be written as
\begin{align*}
\widehat{\boldsymbol{\Sigma}}_{\mathbf{Y}_s} = \frac{1}{L_s} \mathbf{Y}_s \mathbf{Y}_s^T.
\end{align*} 
Similar, the source matrix $\mathbf{X}_s$ can be described in the covariance domain by the sample covariance matrix:
\begin{align*}
\widehat{\boldsymbol{\Sigma}}_{\mathbf{X}_s} &= \frac{1}{L_s} \mathbf{X}_s \mathbf{X}_s^T = \boldsymbol{\Lambda}_s + \boldsymbol{\varepsilon}. 
\end{align*}
The second equality comes from the assumption of the sources within $\mathbf{X}_s$ being uncorrelated within one segment. By uncorrelated sources $\mathbf{X}_s$ the sample covariance matrix is asumed to be nearly diagonal. Thus it can be written as $\boldsymbol{\Lambda}_s + \boldsymbol{\varepsilon}$ where $\boldsymbol{\Lambda}_s$ is a diagonal matrix consisting of the diagonal entries of $\widehat{\boldsymbol{\Sigma}}_{\mathbf{X}_s}$ and $ \boldsymbol{\varepsilon}$ is a non-diagonal matrix representing the estimation error \cite{Balkan2015}.

Each segment is now modelled in the covariance domain, note that the $\frac{1}{L_s}$ factor is left out in order to comply to the instruction of source. \todo{convenience alert!}
\begin{align} 
\widehat{\boldsymbol{\Sigma}}_{\mathbf{Y}_s} = \frac{1}{L_s}\mathbf{Y}_s \mathbf{Y}_s^T &= \frac{1}{L_s} \left( \mathbf{A}_s \mathbf{X}_s + \mathbf{E}_s \right) \left( \mathbf{A}_s \mathbf{X}_s + \mathbf{E}_s\right)^T \nonumber \\ 
 &= \frac{1}{L_s} \left( (\textbf{A}_s\textbf{X}_s)(\textbf{A}_s\textbf{X}_s)^T + \textbf{E}_s \textbf{E}_s^T + \textbf{E}_s (\textbf{A}_s\textbf{X}_s)^T + \textbf{A}_s\textbf{X}_s \textbf{E}_s^T \right) \nonumber \\
&= \frac{1}{L_s} \left( \textbf{A}_s\textbf{X}_s \textbf{X}_s^T \textbf{A}_s^T + \textbf{E}_s \textbf{E}_s^T + \textbf{E}_s \textbf{X}_s^T \textbf{A}_s^T + \textbf{A}_s\textbf{X}_s \textbf{E}_s^T \right) \nonumber \\
&=  \frac{1}{L_s} \left( \textbf{A}_s(\boldsymbol{\Lambda}_s +\boldsymbol{\varepsilon}) \textbf{A}_s^T + \textbf{E}_s \textbf{E}_s^T + \textbf{E}_s \textbf{X}_s^T \textbf{A}_s^T + \textbf{A}_s\textbf{X}_s \textbf{E}_s^T \right) \nonumber \\
&= \frac{1}{L_s} \left( \textbf{A}_s \boldsymbol{\Lambda}_s \textbf{A}_s^T + \textbf{A}_s \boldsymbol{\varepsilon} \textbf{A}_s^T + \textbf{E}_s \textbf{E}_s^T + \textbf{E}_s \textbf{X}_s^T \textbf{A}_s^T + \textbf{A}_s\textbf{X}_s \textbf{E}_s^T\right) \label{eq:noise1} \\
&= \frac{1}{L_s} \left( \textbf{A}_s \boldsymbol{\Lambda}_s \textbf{A}_s^T + \widetilde{\textbf{E}} \right) \label{eq:noise2}
\end{align}
From \eqref{eq:noise1} to \eqref{eq:noise2} all terms where noise, $\boldsymbol{\varepsilon}$ and $\mathbf{E}$, is included, are aggregated in a joint noise term $\widetilde{\textbf{E}}$. 
By vector notation \eqref{eq:noise2} is rewritten to be vectorized. 
Because the covariance matrix $\widehat{\boldsymbol{\Sigma}}_{\mathbf{Y}_s}$ is symmetric it is sufficient to vectorize only the lower triangular parts, including the diagonal. 
For this the function $\text{vec}(\cdot)$ is defined to map a symmetric $M \times M$ matrix into a vector of size $\widetilde{M} := \frac{M(M+1)}{2}$ making a row-wise vectorization of its lower triangular part. 
Furthermore, let vec$^{-1}(\cdot)$ be the inverse function for devectorisation. 
This result in the following model 
\begin{align}
\widehat{\boldsymbol{\Sigma}}_{\mathbf{Y}_s} &= \sum_{i=1}^{N} \textbf{a}_i \boldsymbol{\Lambda}_{s_{ii}} \textbf{a}_i^{T} + \widetilde{\textbf{E}}, \quad \boldsymbol{\Lambda}_{s_{ii}} \in \mathbb{R}_0 \nonumber \\
& \nonumber \\
\text{vec}(\widehat{\boldsymbol{\Sigma}}_{\mathbf{Y}_s}) &= \sum_{i=1}^N \text{vec}(\mathbf{a}_i \mathbf{a}_i^T) \boldsymbol{\Lambda}_{s_{ii}} + \text{vec}( \widetilde{\textbf{E}}) \nonumber \\
&= \sum_{i=1}^N \mathbf{d}_i \boldsymbol{\Lambda}_{s_{ii}} + \text{vec}( \widetilde{\textbf{E}}) \nonumber \nonumber \\
&= \mathbf{D} \boldsymbol{\delta}_s + \text{vec}( \widetilde{\textbf{E}}), \quad \forall s. \label{eq:cov1}
\end{align}
Here $\boldsymbol{\delta}_s \in \mathbb{R}^{N}$ contains the diagonal entries of the source sample-covariance matrix $\boldsymbol{\Lambda}_s$
and the matrix $\mathbf{D} \in \mathbb{R}^{\widetilde{M} \times N}$ consists of the columns $\mathbf{d}_i = \text{vec}(\mathbf{a}_i \mathbf{a}_i^T)$ where $\mathbf{a}_i$ is the rows of $\mathbf{A}$. Note that $\mathbf{D}$ and $\boldsymbol{\delta}_s$ are unknown while $\text{vec}(\widehat{\boldsymbol{\Sigma}}_{\mathbf{Y}_s})$ is known from the observed data.
By this transformation to the covariance domain, one segment is now represented by the single measurement model with $\widetilde{M}$ "measurements". 
It has been shown that this transformed model allows for identification of $k \leq \widetilde{M}$ active sources \cite{Pal2015}, which is a much weaker sparsity constraint than the original sparsity constraint $k \leq M$. 
The purpose of the Cov-DL algorithm is to leverage this model to find the dictionary $\mathbf{A}$ from $\mathbf{D}$ and then still allow for $k \leq \widetilde{M}$ active sources to be identified. 
That is the number of active sources are allowed to exceed the number of observations as intended.

\section{Determination of the Mixing Matrix}
The goal is now to learn first $\textbf{D}$ and then the associated mixing matrix $\textbf{A}$. 
Two methods are considered relying on the relation between $M$ and $N$. 
For now the noise vector is ignored.


\subsection{Under-determined system}\label{sec:cov1}
When $N > \widetilde{M}$ the transformed model \eqref{eq:cov1} makes an under-determined system.   
This is similar to the original MMV model \eqref{eq:MMV_model} being under-determined  when $N > M$. 
Thus, it is from the theory of compressive sensing again possible to solve the under-determined system if a certain sparsity is withhold. 
Namely $\boldsymbol{\delta}_s$ being $\widetilde{M}$-sparse.
Assuming the sufficient sparsity on $\boldsymbol{\delta}_s$ is withhold it is possible to learn the dictionary matrix of the covariance domain $\mathbf{D}$ by traditional dictionary learning methods applied to the observations represented in the covariance domain $\text{vec}(\widehat{\boldsymbol{\Sigma}}_{\mathbf{Y}_s})$ for all segments $s$.

\subsubsection{Dictionary Learning}\label{sec:dictionarylearning}
Within the theory of compressive sensing the matrix $\textbf{A}$ is referred to as a dictionary matrix, as it determines how a sparse vector $\textbf{x}$ is transformed to the original non-sparse signal. 
When the dictionary is not known i prior it is essential how to choose the the dictionary matrix in order to achieve the best recovery, of the sparse vector $\mathbf{x}$ from the measurements $\mathbf{y}$. 
This is clarified from the proof of theorem \ref{th:CS_A} in appendix \ref{app_sec:CS}. 
One choice is a pre-constructed dictionary. 
In many cases the use of a pre-constructed dictionary results in simple and fast algorithms for reconstruction of $\mathbf{x}$ \cite{Elad_book}. 
However, a pre-constructed dictionary is typically fitted to a specific kind of data. 
For instance the discrete Fourier transform or the discrete wavelet transform are used especially for sparse representation of images \cite{Elad_book}. 
Hence the results of using such dictionaries depend on how well they fit the data of interest, which is creating a certain limitation. 

The alternative option is to consider an adaptive dictionary based on a set of training data that resembles the data of interest. 
For this purpose learning methods are considered to empirically construct a fixed dictionary which can take part in the application. 
There exist several dictionary learning algorithms. One is the K-SVD algorithm which was presented in 2006 by Elad et al. and found to outperform pre-constructed dictionaries, when computational cost is of secondary interest \cite{Elad2006}. The concept of the K-SVD algorithm is introduced, and the more detailed algorithm is to be found in appendix \ref{app_sec:K-SVD_alg}. 
Consider the measurement matrix $\mathbf{Y} \in \mathbb{R}^M$ consisting of measurement vectors $\lbrace \mathbf{y}_j \rbrace_j^L$ making a set of $L$ training examples forming a linear system
\begin{align*}
\mathbf{y}_j = \mathbf{A} \mathbf{x}_j.
\end{align*}
from which one can learn a suitable dictionary $\hat{\mathbf{A}}$, and the sparse representation of the source matrix $\hat{\mathbf{X}} \in \mathbb{R}^N$ with the source vectors $\lbrace \hat{\mathbf{x}}_j \rbrace_j^L$.
For a known sparsity constraint $k$ the dictionary learning can be defined by the following optimisation problem. 
\begin{align}\label{eq:SVD1}
\min_{\mathbf{A}, \mathbf{X}} \sum_{j=1}^{L} \Vert \mathbf{y}_j - \mathbf{A} \mathbf{x}_j \Vert_2^2 \quad \text{subject to} \quad \Vert \textbf{x}_j \Vert_1 \leq k, \ 1 \leq j \leq L.
\end{align}
where both $\textbf{A}$ and $\textbf{x}_j$ are variables to be determined.\todo{comparison to l1 problem is removed here, ok?}  
Learning the dictionary by the K-SVD algorithm constitute joint solving of the optimization problem with respect to $\mathbf{A}$ and $\mathbf{X}$ respectively. An initial $\textbf{A}_0 = [\textbf{a}_0,\hdots,\textbf{a}_N]$ and the corresponding $\textbf{X}_0$ is determined. Then, for each iteration an update rule is applied to each column of $\textbf{A}_0$, that is updating first $\textbf{a}_j$ and then the corresponding row $\textbf{x}_i$. More details on the K-SVD algorithm is found in appendix \ref{app_sec:K-SVD_alg}. 
The uniqueness of $\mathbf{A}$ depends on the recovery sparsity condition. As clarified earlier in \ref{sec:sol_met} the recovery of a unique solution $\mathbf{X}^\ast$ is only possible if $k < M$ \cite{phd2015}.
%The dictionary learning algorithm K-SVD is a generalisation of the well known K-means clustering also referred to as vector quantization. In K-means clustering a set of $K$ vectors is learned referred to as mean vectors. Each signal sample is then represented by its nearest mean vector. That corresponds to the case with sparsity constraint $k = 1$ and the representation reduced to a binary scalar $x = \lbrace 1, 0 \rbrace$. Further instead of computing the mean of $K$ subsets the K-SVD algorithm computes the SVD factorisation of the $K$ different sub-matrices that correspond to the $K$ columns of $\textbf{A}$.


\subsubsection{Application of dictionary learning}
By the establishments of an dictionary learning algorithm it is now used to learn the transformed dictionary matrix $\textbf{D}$ in \eqref{eq:cov1}. Here the transformed and vectorised measurements $\left\{ \text{vec}\left( \hat{\Sigma}_\textbf{Y}\right), \forall s\right\}$ makes the training dataset. By this note that each segment the original measurement sample constitute only one sample in the covariance domain. Thus the number of training samples depends on the length of a segment.     
When K-SVD is applied and $\mathbf{D}$ is found it is possible to estimate the mixing matrix $\mathbf{A}$ that generated found $\textbf{D}$ through the relation 
\begin{align*}
\mathbf{d}_j = \text{vec}(\mathbf{a}_j \mathbf{a}_j^T).
\end{align*}
Here each column is found from the optimisation problem 
\begin{align*}
\min_{\textbf{a}_j} \| \text{vec}^{-1}(\textbf{d}_j) -\textbf{a}_j\textbf{a}_j^T\|_2^2, 
\end{align*}
for which the global minimizer is $\mathbf{a}^{\ast}_j=\sqrt{\lambda_j} \textbf{b}_j$\todo{redegørelse for resultatet her skal laves}. Here $\lambda_j$ is the largest eigenvalue of $\text{vec}^{-1}(\textbf{d}_j)$,
\begin{align*}
\text{vec}^{-1}(\textbf{d}_j) = 
\begin{bmatrix}
d_{11} & d_{12} & \cdots & d_{1N} \\
d_{21} & d_{22} & \cdots & d_{2N} \\
\vdots & \vdots & \ddots & \vdots \\
d_{N1} & d_{N2} & \cdots & d_{NN}
\end{bmatrix}, \quad j \in [N]
\end{align*}
and $\textbf{b}_j$ is the corresponding eigenvector.

By this each column of the mixing matrix $\textbf{A}$ can be estimated hence it is possible to determine the mixing matrix in the case where the measurements transformed into the covariance domain makes an under-determined system, but the necessary sparsity constraint, $\boldsymbol{\delta}_s$ being $\widetilde{M}$-sparse (instead of $M$-sparse), is withhold.    





\input{sections/ch_cov-DL/Cov-DL2.tex}

\section{Pseudo Code of the Cov-DL Algorithm}\label{seg:alg_cov}
\begin{algorithm}[H]
\caption{Cov-DL}
\begin{algorithmic}[1]
           \Procedure{Cov-DL}{$\textbf{Y}_s$}    
			\For{$s \gets 1,\hdots, \text{n\_seg}$}			
				\State$\text{compute sample covariance matrix}\ \widehat{\boldsymbol{\Sigma}}_{\textbf{Y}_s} $
				\State$\textbf{y}_{\text{cov}_s} = \text{vec}(\widehat{\boldsymbol{\Sigma}}_{\textbf{Y}_s})$	
			\EndFor			
			\State$\textbf{Y}_{\text{cov}} = \{\textbf{y}_{\text{cov}_s}\}_{s=1}^{\text{n\_seg}}$
			\State
			\If{$N \geq \widetilde{M}$}		
			\Procedure{K-SVD}{$\textbf{Y}_{\text{cov}}$}
			\State$\text{returns} \ \textbf{D} \in \mathbb{R}^{\widetilde{M}}\times N$
			\EndProcedure
			\For{$j \gets 1, \hdots, N$}
			\State$\textbf{T} = \text{vec}^{-1}(d_j)$            
			\State$\lambda_j\gets \max\{\text{eigenvalue}(\textbf{T})\}$
			\State$\textbf{b}_j \gets \ \text{eigenvector}(\lambda_j)$
			\State$\textbf{a}_j \gets \sqrt{\lambda_j}\textbf{b}_j$
			\EndFor
			\State$\textbf{A} = \{\textbf{a}_j\}_{j=1}^N$
			\EndIf
			\State
			\If{$N < \widetilde{M}$}
				\Procedure{PCA}{$\text{vec}(\boldsymbol{\Sigma}_{\textbf{Y}_s})$}
				\State$\text{returns} \ \textbf{U}\in \mathbb{R}^{\widetilde{M}\times N}$
				\EndProcedure
				\Procedure{Min. $\textbf{A}$ in }{$\Vert  \textbf{D}(\textbf{D}^T\textbf{D})^{-1}\textbf{D}^T - \textbf{U}(\textbf{U}^T\textbf{U})^{-1}\textbf{U}^T \Vert_{F}^{2}$}
				\State$\text{returns}\ \textbf{A}= \{\textbf{a}_j\}_{j=1}^{N}$
				\EndProcedure
			\EndIf
           \EndProcedure
        \end{algorithmic} 
        \label{alg:Cov1}
\end{algorithm}

\section{Considerations and Remarks}
Through this chapter different theory aspects haven been investigated to create a foundation to present one method to be use in the localization of the sources from EEG measurements -- the recovering of the mixing matrix $\mathbf{A}$ -- yet one method is still to be presented.
Before introducing a method to recover the source matrix $\mathbf{X}$ from the found mixing matrix $\mathbf{A}$ and EEG measurements $\mathbf{Y}$ some considerations and remarks regarding the Cov-DL algorithm must be taken. 
This will be used in the implementation of the Cov-DL which will be described in chapter \ref{ch:implementation}.

The length of each segment determined whenever the covariance of the source matrix $\mathbf{X}$ can be described as a diagonal matrix $\boldsymbol{\Lambda}$. 
That is a segment of $L_s$ samples becomes stationary and therefore the sources within that segment becomes uncorrelated -- the covariance of the source can be described by a diagonal matrix. 
The number of samples $L_s$ used in one segment affect whenever the segment is stationary or not. 
This must be taken into account in the preprocessing part for the implementation of Cov-DL when the EEG measurements are divided into segments.

For the Cov-DL algorithm when $\mathbf{D}$ is under-determined a dictionary learning algorithm K-SVD is used to learn the matrix $\mathbf{D}$ and by that an estimate, $\hat{\mathbf{A}}$, for the mixing matrix $\mathbf{A}$. 
Because of the segmentation the number of samples used in the dictionary learning are reduced remarkably and will affect the learning process. 
This is another point which must be taken into account in the preprocessing part of the implementation of Cov-DL. 
To improved the dictionary learning the overlapping of the segments can be look into as each segment will have some similarity and therefore learn towards one direction.

For the Cov-DL algorithm when $\mathbf{D}$ is over-determined the solution tends to be unique when $M < N < \widetilde{M}$ from testing the solution. 
That is the cost function tends toward a local minima and therefore an unique solution occur in first run of one trial. 
For the implementation of Cov-DL it would therefore be necessary to include several random initial points when finding the mixing matrix $\mathbf{A}$ for $\mathbf{D}$ being over-determined \todo{Dette afsnit skal lige revideres. Vi snakker om test og trial and error men vi har jo ikke beskrevet dette endnu - evt. mindre implementerings orienteret, i forhold til det vi har beskrevet endnu}.

For a general perspective the sources within the source matrix $\mathbf{X}$ must not be constant over time when using the MMV model \eqref{eq:MMV_model} \todo{Find lige kilde på dette argument}...
\chapter{Optimization Worksheet}
