\section{Covariance-Domain Dictionary Learning}\label{sec:Cov-DL}
Covariance-domain dictionary learning (Cov-DL) is an algorithm which can identify more sources $N$ than sensors $M$ for the linear model of observed EEG data
\begin{align*}
\mathbf{Y} = \mathbf{AX}.
\end{align*}
Cov-DL takes advantage of dictionary framework and transformation into another domain -- covariance domain -- to recover the mixing matrix $\mathbf{A}$ from the observed data $\mathbf{Y}$. Cov-DL work together with another algorithm to find the sparse source matrix $\mathbf{X}$, in this thesis M-SBL is used for the source recovery and is described in section \ref{sec:M-SBL}. 
\\
In the following section we assume that $\mathbf{X}$ is known but in practice a random sparse matrix will be used to represent the sources. 
\\
This section is inspired by chapter 3 in \cite{phd2015} and the article \cite{Balkan2015}.
\\ \\
(?)\\
In dictionary learning framework the inverse problem is defined as
\begin{align*}
\min_{A,X} = \frac{1}{2} \sum_{s=1}^{N_d} \Vert \mathbf{Y} - \mathbf{AX} \Vert_F^2 + \gamma \sum_{s=1}^{N_d} g(\mathbf{x}_s),
\end{align*}
where the function $g(\cdot)$ promotes sparsity of the source vector at time $t$ 
The true dictionary $\mathbf{A}$ is recovered if the sources $\mathbf{x}_s$ are sparse ($k_s < M$).


\paragraph{Introduction to Our Covariances:}
Let $s$ be the time segments that $\mathbf{Y}$ is divided into and let it be sampled with the frequency $S_f$ such that our observed data is known overlapping segments $\mathbf{Y}_s \in \mathbb{R}^{M \times t_s S_f}$ where $t_s$ is the length of the segments in seconds. With the segments the linear model still holds and is rewritten into
\begin{align*}
\mathbf{Y}_s = \mathbf{AX}_s, \quad \forall s.
\end{align*}
The sources are assumed uncorrelated in time segments of the whole time scheme of the observed data. 
\\
In the covariance-domain, the observed segmented data $\mathbf{Y}_s$ is described by its covariance:
\begin{align*}
\boldsymbol{\Sigma}_{\mathbf{Y}_s} &= \mathbf{A} \boldsymbol{\Lambda} \mathbf{A}^T + \mathbf{E}_s \\
\text{vech}(\boldsymbol{\Sigma}_{\mathbf{Y}_s}) &= \sum_{i=1}^N \boldsymbol{\Lambda}_{s_{ii}} \text{vech}(\mathbf{a}_i \mathbf{a}_i^T) + \text{vech}(\mathbf{E}_s) \\
\text{vech}(\boldsymbol{\Sigma}_{\mathbf{Y}_s}) &= \mathbf{D} \boldsymbol{\delta}_s + \text{vech}(\mathbf{E}_s), \quad \forall s.
\end{align*}
The vector $\boldsymbol{\delta}_s$ contains the diagonal entries of the source sample-covariance matrix
\begin{align*}
\boldsymbol{\Sigma}_{\mathbf{X}_s} = \frac{1}{L_s} \mathbf{X}_s \mathbf{X}_s^T,
\end{align*}
and the matrix $\mathbf{D} \in \mathbb{R}^{M(M+1)/2 \times N}$ consists of the columns $\mathbf{d}_i = \text{vech}(\mathbf{a}_i \mathbf{a}_i^T)$. D and $\delta_s$ are unknown.
\\
Our goal is to learn $\mathbf{D}$ and the find the associated matrix $\mathbf{A}$. When we have the dictionary matrix we can find  the mixing matrix by
\begin{align*}
\min_{\mathbf{a}_i} \Vert \mathbf{d}_i - \text{vech}(\mathbf{a}_i \mathbf{a}_i^T) \Vert_2^2.
\end{align*}




\paragraph{Introduction to Covariance Domain Transformation:}
By the assumption of uncorrelated sources, the sample covariance source matrix is given as
\begin{align*}
\boldsymbol{\Sigma}_{\mathbf{X}_s} &= \frac{1}{L_s} \mathbf{X}_s \mathbf{X}_s^T \\
&= \boldsymbol{\Lambda} + \mathbf{E},
\end{align*}
where $\boldsymbol{\Lambda}$ is the diagonal matrix of $\boldsymbol{\Sigma}_{\mathbf{X}_s}$. With this mindset, the linear model given in \eqref{} can then be modelled as
\begin{align*}
\mathbf{Y}_s \mathbf{Y}_s^T &= \mathbf{AX}_s \mathbf{X}_s^T \mathbf{A}^T \\
\boldsymbol{\Sigma}_{\mathbf{Y}_s} &= \mathbf{A} \boldsymbol{\Sigma}_{\mathbf{X}_s} \mathbf{A}^T \\
\boldsymbol{\Sigma}_{\mathbf{Y}_s} &= \mathbf{A} \boldsymbol{\Lambda} \mathbf{A}^T + \mathbf{E} \\
&= \sum_{i=1}^N \boldsymbol{\Lambda}_{ii} \mathbf{a}_i \mathbf{a}_i^T + \mathbf{E}.
\end{align*}
As the covariance matrices are symmetric the lower triangular part can be vectorised:
\begin{align*}
\text{vech}(\boldsymbol{\Sigma}_{\mathbf{Y}_s}) &= \sum_{i=1}^N \boldsymbol{\Lambda}_{ii} \text{vector}(\mathbf{a}_i \mathbf{a}_i^T) + \text{vech}(\mathbf{E}) \\
\text{vech}(\boldsymbol{\Sigma}_{\mathbf{Y}_s}) &= \sum_{i=1}^N \boldsymbol{\Lambda}_{ii} \mathbf{d}_i + \text{vech}(\mathbf{E}) \\
\text{vech}(\boldsymbol{\Sigma}_{\mathbf{Y}_s}) &= \mathbf{D} \delta + \text{vech}(\mathbf{E}),
\end{align*}
where $\mathbf{d}_i = \text{vech}(\mathbf{a}_i \mathbf{a}_i^T)$. The size of the vectorised covariance matrices is $\frac{M(M+1)}{2}$.
\\
By use of the covariance domain it is possible to identify $\mathcal{O}(M^2)$ sources given the true dictionary matrix $\mathbf{A}$.


\paragraph{Notes:}
In the case of EEG, this allows at most k = O(M) EEG sources to be simultaneously active which limits direct applicability of dictionary learning to low-density EEG systems.
\\
We wish to handled cases where we have $\binom{N}{k}$ sources, where $1 \leq k \leq N$ can be jointly active. 
\\
section 3.3.1 in phd.
















