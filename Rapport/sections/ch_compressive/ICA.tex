\section{Basic Theory of Independent Component Analysis}
Independent component analysis (ICA) is a method that applies to the general problem of decomposition of a measurement vector into a source vector and a mixing matrix. 
The intention of ICA is to separate a multivariate signal into statistical independent and non-Gaussian signals. And identify the mixing matrix $\mathbf{A}$, given only the observed measurements $\mathbf{Y}$.
A well-known application example of source separation is the cocktail party problem, where it is sought to listen to one specific person speaking in a room full of people having interfering conversations. 
Let $\mathbf{y} \in \mathbb{R}^M$ be a single measurement from $M$ microphones containing a linear mixture of all the speak signals that are present in the room. 
When additional noise is not considered, the problem can be described as the familiar linear system 
\begin{align}\label{eq:ICA1}
\mathbf{y} = \mathbf{Ax},
\end{align}
where $\mathbf{x} \in \mathbb{R}^N$ contain the $N$ underlying speak signals. $\mathbf{A}$ is a mixing matrix where the coefficients may depend on the distances from a source to the microphone. 
As such each $y_i$ is a weighted sum of all the sources of speak present to the $i$-th microphone.
By ICA both the mixing matrix $\mathbf{A}$ and the source signals $\mathbf{x}$ are sought estimated from the observed measurements $\mathbf{y}$. 
The main attribute of ICA is the assumption that the sources in $\mathbf{x}$ are statistically independent and non-Gaussian distributed, hence the name independent components.
\\     
By independence, one means in general that changes in one source signal do not affect the other source signals. 
In theory $N$ variables $x_1, \dots , x_N$ is independent if the joint probability density function (pdf) of $\mathbf{x}$ satisfies
\begin{align*}
p(x_1, x_2, \dots, x_N) = p_1(x_1) p_2(x_2) \cdots p_n(x_N).
\end{align*}
The possibility of separating a signal into independent and non-Gaussian components originates from the central limit theorem \cite[p. 34]{ICA}. 
The theorem states that the distribution of any linear mixture of two or more independent random variables tends toward a Gaussian distribution, under certain conditions. 
For instance the distribution of a mixture of two independent random variables is always closer to a Gaussian distribution than the original variables. In other word the original variables is most non-Gaussian. 
The application of the central limit theorem within ICA will be elaborated later in this appendix.

\subsection{Assumptions and Preprocessing}
For simplicity assumes that $\mathbf{A}$ is square i.e. $M = N$ and invertible. 
As such when $\mathbf{A}$ has been estimated the inverse is computed and the components can simply be estimated as $\mathbf{x} = \mathbf{A}^{-1} \mathbf{y}$ \cite[p. 152-153]{ICA}.
As both $\mathbf{A}$ and $\mathbf{x}$ are unknown the variances of the independent components can not be determined. 
However, it is reasonable to assume that $\mathbf{x}$ has unit variance -- $\mathbf{A}$ is assumed to have unit variance as well. 
Any scalar multiplier within a source can be canceled out by dividing the corresponding column in $\mathbf{A}$ with the same scalar \cite[p. 154]{ICA}.
For further simplification it is assumed without loss of generality that $\mathbb{E}[\mathbf{y}] = 0$ and $\mathbb{E}[\mathbf{x}] = 0$\cite[p. 154]{ICA}. 
In case this assumption is not true, the measurements can be centered by subtracting the mean as preprocessing before doing ICA.

A preprocessing step central to ICA is to whiten the measurements $\mathbf{y}$. 
By the whitening process any correlation in the measurements are removed and unit variance is ensured -- the independent components $\mathbf{x}$ becomes uncorrelated and have unit variance. 
Furthermore, this reduces the complexity of ICA and therefore simplifies the recovering process.
Whitening is a linear transformation of the observed data. 
This is multiplying the measurement vector $\mathbf{y}$ with a whitening matrix $\mathbf{V}$
\begin{align*}
\mathbf{y}_{\text{white}} = \mathbf{V} \mathbf{y},
\end{align*} 
to obtain a new measurement vector $\mathbf{y}_{\text{white}}$ which is considered white. 
To obtain a whitening matrix, the eigenvalue decomposition (EVD) of the covariance matrix can be used:
\begin{align*}
\mathbb{E}[\mathbf{yy}^T] = \mathbf{EDE}^T,
\end{align*}
where $\mathbf{D}$ is a diagonal matrix of eigenvalues and $\mathbf{E}$ is a matrix consisting of the associated eigenvectors. 
From $\mathbf{E}$ and $\mathbf{D}$ a whitening matrix $\mathbf{V}$ is constructed as
\begin{align*}
\mathbf{V} = \mathbf{ED}^{-1/2} \mathbf{E}^T,
\end{align*}
where $\mathbf{D}^{-1/2} = \text{diag}(d_1^{-1/2}, \dots, d_n^{-1/2})$ is a component-wise operation \cite[p. 159]{ICA}.
\\  
By multiplying the measurement vector $\mathbf{y}$ with a whitening matrix $\mathbf{V}$ the data becomes white
\begin{align*}
\mathbf{y}_{\text{white}} &= \mathbf{Vy} = \mathbf{VAx} = \mathbf{A}_{\text{white}} \mathbf{x}.
\end{align*}
Furthermore, the corresponding mixing matrix $\mathbf{A}_{\text{white}}$ becomes orthogonal 
\begin{align*}
 \mathbb{E}[\mathbf{y}_{\text{white}} \mathbf{y}_{\text{white}}^T] = \mathbf{A}_{\text{white}} \mathbb{E}[\mathbf{xx}^T] \mathbf{A}_{\text{white}}^T = \mathbf{A}_{\text{white}} \mathbf{A}_{\text{white}}^T = \mathbf{I},
 \end{align*} 
where $\mathbb{E}[\mathbf{xx}^T] = \mathbf{I}$ due to $\mathbf{x}$ having zero mean and uncorrelated entries. 

As a consequence ICA can restrict its search for the mixing matrix to the orthogonal matrix space. That is instead of estimating $N^2$ parameters ICA has only to estimate an orthogonal matrix which has $N(N-1)/2$ parameters \cite[p. 159]{ICA}.

\subsection{Recovery of the Independent Components}\label{sec:est_ica}
The estimation of the mixing coefficients $a_{ij}$ and the independent components $x_i$ by ICA is now elaborated, based on \cite[p. 166]{ICA}.  

The simple and intuitive method is to take advantage of the assumption of non-Gaussian independent components. 
Consider again the model of a single measurement vector $\mathbf{y} = \mathbf{Ax}$, where the data vector complies to the assumption of being mixture of independent components. Here the independent components can be estimated by the inverted model
\begin{align*}
\mathbf{x} = \mathbf{A}^{-1} \mathbf{y}.
\end{align*}
Consider first the estimation of a single independent component $x_i$.
Here a linear combination of $y_i$ is considered. Denote for now a single independent component by $z$ such that
\begin{align}
z = \mathbf{b}^T \mathbf{y} = \sum_k b_k y_k, \label{eq:ICA_comp}
\end{align} 
where one want to determine the vector $\mathbf{b}$. This can be rewritten to 
\begin{align*}
z = \mathbf{b}^T \mathbf{Ax}
\end{align*} 
From this it is seen that $\mathbf{y}$ is a linear combination of the $x_i$ with coefficients given by the vector $\mathbf{b}^T \mathbf{A}$.
Let now $\mathbf{b}^T \mathbf{A}$ be denoted by the $\textbf{q}$. As such  
\begin{align}
z = \mathbf{b}^T \mathbf{y} = \mathbf{b}^T \mathbf{Ax} = \mathbf{q}^T \mathbf{x} = \sum_k q_k x_k. \label{eq:ICA_comp2} 
\end{align}
By this expression, consider the thought of $\mathbf{b}$ being one of the rows in $\mathbf{A}^{-1}$, then the linear combination $\mathbf{b}^{T}\mathbf{y}$ is equal to one of the independent components. 
 
The objective is now to apply the central limit theorem to determine $\mathbf{b}$ such that it equals one of the rows of $\mathbf{A}^{-1}$. 
As $\mathbf{A}$ is unknown it is not possible to determine $\mathbf{b}$ exactly, but an estimate can be found to make a good approximation.

Due to $z$ denoting some $x_i$ it is clear that the equality in \eqref{eq:ICA_comp2} only holds true when $\mathbf{q}$ consist of only one non-zero element equal to 1.  
Thus, from the central limit theorem the distribution of $\mathbf{q}^T \mathbf{x}$ is most non-Gaussian when it equals one of the independent components which was assumed non-Gaussian. 
As such, since $\mathbf{q}^T \mathbf{x} = \mathbf{b}^{T} \mathbf{y}$, it is possible to vary the coefficients in $\mathbf{b}$ and look at the distribution of $\mathbf{b}_{i \cdot} \mathbf{y}$. 
Finding the vector $\mathbf{b}^T$ that maximizes the non-Gaussianity would then correspond to $\mathbf{q} = \mathbf{A}^T \mathbf{b}^T$ having only a single non-zero element. 
Thus maximizing the non-Gaussianity of $\mathbf{b}_{i \cdot} \mathbf{y}$ results in one of the independent components \cite[p. 166]{ICA}. 
Considering the $N$-dimensional space of vectors $\mathbf{b}^T$ there exist $2N$ local maxima, corresponding to $x_i$ and $-x_i$ for all $N$ independent components \cite[p. 166]{ICA}. 

\subsection{Kurtosis}
To maximize the non-Gaussianity, a measure for Gaussianity is needed. Kurtosis is a quantitative measure used for non-Gaussianity of random variables \cite[p. 171]{ICA}. 
Kurtosis of a random variable $y$ is the fourth-order cumulant denoted by $\text{kurt}(y)$. 
For $y$ with zero mean and unit variance, kurtosis reduces to 
\begin{align*}
\text{kurt} (y) = \mathbb{E}[y^4] - 3.
\end{align*} 
It is seen that the kurtosis is a normalized version of the fourth-order moment defined as $\mathbb{E}[y^4]$. 
For a Gaussian random variable the fourth-order moment equals $3(\mathbb{E}[y^2])^2$ hence the corresponding kurtosis will be zero \cite[p. 171]{ICA}. 
Consequently, the kurtosis of non-Gaussian random variables will almost always be different from zero.
\\
The kurtosis is a common measure for non-Gaussianity due to its simplicity both theoretical and computational. 
The kurtosis can be estimated computationally by the fourth-order moment of sample data when the variance is constant.
Furthermore, for two independent random variables $x_1, x_2$ the following linear properties applies to the kurtosis of the sum
\begin{align*}
\text{kurt}(x_1 + x_2) = \text{kurt}(x_1) + \text{kurt}(x_2) \quad \text{and} \quad \text{kurt}(\alpha x_1) = \alpha^4 \text{kurt}(x_1)
\end{align*}  
However, one complication concerning kurtosis as a measure is that kurtosis is sensitive to outliers \cite[p. 182]{ICA}.
\\ \\      
Consider from \eqref{eq:ICA_comp2} the vector $\mathbf{q} = \mathbf{A}^T \mathbf{b}$ such that $\mathbf{b}^T \mathbf{y} = \sum_{k=1} q_k x_k$. 
By the additive property of kurtosis
\begin{align*}
\text{kurt} \left( \mathbf{b}^T \mathbf{y} \right) = \sum_{k=1}q_k^4 \text{kurt}(x_k).
\end{align*}
Then the assumption of the independent components having unit variance results in $\mathbb{E}[x_i^2]= \sum_{k=1}q_k^2=1$. 
That is geometrically that $\mathbf{q}$ is constrained to the unit sphere, $\| \mathbf{q}\|^2 = 1$. 

By this an optimization problem maximizing the kurtosis of $\mathbf{b}^T \mathbf{y}$ is similar to maximizing $\vert \text{kurt}(x_i)\vert = \vert \sum_{k=1}q_k^4 \text{kurt}(x_k) \vert $ on the unit sphere.
Due to the described preprocessing $\mathbf{b}^T$ is assumed to be white and it can be shown that $\Vert \mathbf{q} \Vert = \Vert \mathbf{b}^T \Vert$ \cite[p. 174]{ICA}. 
This shows that constraining $\Vert \mathbf{q} \Vert$ to one is similar to constraining $\Vert \mathbf{b}^T \Vert$ to one. 

\subsection{Basic ICA algorithm}\label{sec:gra_kur}
Now a basic ICA algorithm is specified, this algorithm is based on the gradient optimization method with kurtosis.

The general idea behind a gradient algorithm is to determine the direction for which $\text{kurt}(\mathbf{b}^T \mathbf{y})$ is growing the most, based on the gradient. 
The gradient of $\vert \text{kurt}(\mathbf{b}^T \mathbf{y}) \vert$ is computed as
\begin{align}\label{eq:kurt}
\frac{\partial \vert \text{kurt}(\textbf{b}^T \textbf{y})\vert}{\partial \mathbf{b}} &= 4 \text{sign}(\text{kurt}(\textbf{b}^T \textbf{y})) (\mathbb{E}[\mathbf{y} (\textbf{b}^T \textbf{y})^3] - 3 \mathbf{y} \mathbb{E}[(\textbf{b}^T \textbf{y})^2]) 
\end{align} 
As $\mathbb{E}[(\mathbf{b}^T \mathbf{y})^2] = \Vert \mathbf{y} \Vert^2$ for whitened data the corresponding term does only affect the norm of $\mathbf{b}$ within the gradient algorithm. 
Thus, as it is only the direction that is of interest, this term can be omitted. 
Because the optimization is restricted to the unit sphere a projection of $\mathbf{b}$ onto the unit sphere must be performed in every step of the gradient method. 
This is done by dividing $\mathbf{b}$ by its norm. 
This gives update step \cite[p. 178]{ICA}
\begin{align*}
\Delta \textbf{b} &\propto \text{sign}\left( \text{kurt}(\textbf{b}^T \textbf{y}) \right) \mathbb{E}[\textbf{y}(\textbf{b}^T \textbf{y})^3] \\
\textbf{b} &\leftarrow \textbf{b}/\Vert \textbf{b} \Vert
\end{align*}  
The expectation operator can be omitted in order to achieve an adaptive version of the algorithm, now using every measurement $\mathbf{y}$. 
However, the expectation operator from the definition of kurtosis can not be omitted and must therefore be estimated. 
This can be done by $\gamma$ by serving it as the learning rate of the gradient method.
\begin{align*}
\Delta \gamma \propto((\textbf{b}^T \textbf{y})^4 - 3) - \gamma
\end{align*}
Algorithm \ref{alg:basicICA} combines the above theory, to give an overview of the basic ICA procedure. 
\begin{algorithm}[H]
\caption{Basis ICA}
\begin{algorithmic}[1]
			\Procedure{Pre-processing}{$\textbf{y}$}
			\State $\text{Center measurements} \quad \textbf{y} \gets \textbf{y} - \bar{\textbf{y}}$
			\State $\text{Whitening} \quad \textbf{y}\gets \textbf{y}_{\text{white}}$ 
			\EndProcedure  
			\State
            \Procedure{ICA}{$\textbf{y}$}    
			\State$k=0$            
            \State$\text{Initialize random vector} \quad \textbf{b}_{(k)}$ \Comment{unit norm}
            \State$\text{Initialize random value} \quad \gamma_{(k)}$
            \For{$j \gets 1,2, \hdots ,N$ }
            
            	\While{$\text{convergence critia not meet}$} 
               		\State $k = k+1$
                	\State $\textbf{b}_{(k)} \gets \text{sign}\gamma_{(k-1)} \textbf{y}(\textbf{b}_{(k-1)} \textbf{y})^3$
                	\State $\textbf{b}_{(k)} \gets \textbf{b}_{(k)}/\Vert \textbf{b}_{(k)} \Vert $ 
                	\State $\gamma_{(k)} \gets ((\textbf{b}_{(k)} \textbf{y})^4 - 3) - \gamma_{(k-1)} $
          		\EndWhile
          		\State $x_{j} = \textbf{b}^T\textbf{y}$
          	\EndFor
          	
            \EndProcedure
        \end{algorithmic} 
        \label{alg:basicICA}
\end{algorithm}

\subsection{ICA for sparse signal recovery}
ICA is widely used within sparse signal recovery.   
When ICA is applied to a measurement vector $\mathbf{y} \in \mathbb{R}^{M}$ it is possible to separate the mixed signal into $M$ or less independent components. 
However, by assuming that the  independent components make a $k$-sparse signal it is possible to apply ICA within sparse signal recovery of cases where $M < N$ and $k \leq M$. 
\\
To apply ICA to such cases, the independent components are obtained by the pseudo-inverse solution 
\begin{align*}
\hat{\mathbf{x}} = \mathbf{A}_S^{\dagger} \mathbf{y}
\end{align*}
where $\mathbf{A}_S$ is derived from the dictionary matrix $\mathbf{A}$ by containing only the columns associated with the non-zero entries of $\textbf{x}$, specified by the support set $S$, cf. appendix \ref{app_sec:CS}. 