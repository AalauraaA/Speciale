\section{System of Linear Equations}\label{sec:SMV}
Let $\mathbf{y} \in \mathbb{R}^M$ be some vector. By basic linear algebra $\mathbf{y}$ can be described as a linear combination of a coefficient matrix $\mathbf{A} \in \mathbb{R}^{M \times N}$ and some scalar vector $\mathbf{x} \in \mathbb{R}^N$ such that
\begin{align}\label{eq:SMV_model}
\mathbf{y} = \mathbf{Ax},
\end{align}
%The measurement vector $\mathbf{y}$ consist of $M$ measurements, $\mathbf{x}$ is an unknown vector of $N$ elements, and the coefficient matrix $\mathbf{A}$ models the linear measurement process column-wise.
Let $\mathbf{y}$ and $\mathbf{A}$ be known, then  
\ref{eq:SMV_model} makes a system of $M$ linear equations with $N$ unknowns, referred to as a linear system. 

To solve the linear system \ref{eq:SMV_model} with respect to $\textbf{x}$ one must look at the three different cases that can occur, depending on the relation between the number of equations $M$ and the number of unknowns $N$.
For $M = N$, the system has one unique solution, provided that a solution exist.  
If the square coefficient matrix $\mathbf{A}$ has full rank the solution can be found by inverting $\mathbf{A}$.
\begin{align*}
\mathbf{x} = \mathbf{A}^{-1} \mathbf{y}.
\end{align*}

For $M > N$ the system is over-determined, having more equations than unknown. In general there is no solution to an over-determined system. An/The exception occur when the system contains a sufficient amount of linearly dependent equations.    
For $M < N$ the system is under-determined, having fewer equations than unknowns. There exist infinitely many solutions to an under-determined system, provided that one solution exist\todo{note: skal vi nævne løsnings metoder for underdetermined system her?} \cite[p. ix]{CS}.  

Consider now $\mathbf{y} \in \mathbb{R}^M$ as the observed measurements provided by $M$ EEG sensors at time $t$. 
The linear system \ref{eq:SMV_model} is then considered as a single measurement vector (SMV) model.  
Modelling the EEG measurements by the SMV model embody the following interpretations, based on chapter \ref{ch:motivation}.
Remember from chapter \ref{ch:motivation} that EEG measurements basically is a mixture of original brain signals affected by volume conduction.
$\textbf{x}$ is seen as the original brain signal sources, each entry representing the signal of one source. 
Thus, $\textbf{x}\in \mathbb{R}^N$ is referred to as the source vector. $N$ is considered the maximum number of sources, however zero-entries may occur. Let $k$ denote the number of non-zero entries in $\textbf{x}$, referred to as the active sources at time $t$.   
The projection matrix $\textbf{A}$, referred to as the mixing matrix, models the volume conduction by mapping the source vector from $\mathbb{R}^N$ to $\mathbb{R}^M$, where $M$ is the number of sensors hence the dimension of the measurement vector $\textbf{y}$.             

\section{Multiple Measurement Vector Model of EEG}\label{sec:MMV}
In practise EEG measurements are sampled over time by a certain sample frequency. 
Thus multiple EEG measurement vectors are achieved.
Let $L$ be the total number of samples. Now the  
the SMV model is expanded to include $L$ measurement vectors:
\begin{align}\label{eq:MMV_model}
\mathbf{Y} = \mathbf{AX}+\textbf{E},
\end{align}
now $\mathbf{Y} \in \mathbb{R}^{M \times L}$ is the observed measurement matrix, $\mathbf{X} \in \mathbb{R}^{N \times L}$ is the source matrix, and $\mathbf{A} \in \mathbb{R}^{M \times N}$ is the mixing matrix. 
Furthermore $\textbf{E} \in \mathbb{R}^{M \times L}$ is consider an additional noise matrix, to be expected from psychical measurements.  
The model is now referred to as the multiple measurement vector (MMV) model.
As for \eqref{eq:SMV_model} the solution set of the linear system \eqref{eq:MMV_model} depends on the relation between $N$ and $M$ \cite[p. 42]{CS}. 

In chapter \ref{ch:motivation} it is specified that the case of more sources than sensors, $N>M$, is the case of interest in this thesis.  
%flytte sidste del?

\subsection{Segmentation}\label{seg_segmentation}
In chapter \ref{ch:motivation} it is argued that EEG measurements are only stationary within small segments. 
Hence the following segmentation is considered.

Let $f$ be the sample frequency of the observed EEG measurements $\mathbf{Y}$ and let $t$ be a time interval in seconds determining the duration of one segment. 
Here $s$ is the segment index. 
As such the observed EEG measurements can be divided into stationary segments  $\mathbf{Y}_s \in \mathbb{R}^{M \times L_{s}}$, possibly overlapping, where $L_s = tf$ is the number of samples within one segment. 
For each segment the MMV model \eqref{eq:MMV_model} holds and is rewritten into
\begin{align}\label{eq:MMV_seg}
\mathbf{Y}_s = \mathbf{AX}_s + \textbf{E}_s, \quad \forall s.
\end{align}
Due to a segment being stationary it is assumed that each source remains either active or non-active throughout the segment.
Thus, $\mathbf{X}_s$, consists of $k$ non-zero rows -- the active sources.

In order to characterise the source matrix with respect the amount of non-zero rows the term row sparseness is considered.  
By common definition the support of the segmented source matrix $\text{supp}(\mathbf{X}_s)$ denotes the index set of non-zero rows of $\mathbf{X}_s$.
%\begin{align*}
%\text{supp}(\mathbf{X}) = \{ j \in [N] \ : \ X_j\cdot \neq 0 \}.
%\end{align*}
%where $[N]$ is a set of integers $\lbrace 1, 2, \hdots, N \rbrace$ \cite[p. 41]{FR}.  
To count the non-zeros row of a matrix the $\ell_0$-norm is defined:
\begin{align*}
\Vert \mathbf{X} \Vert_0 := \text{card}(\text{supp}(\mathbf{X})),
\end{align*}
where the function $\text{card}(\cdot)$ gives the cardinality of the input set. 
$\textbf{X}_s$ is said to be $k$-sparse if it contains at most $k$ non-zeros rows:
\begin{align*}
\Vert \mathbf{X}_s \Vert_0 \leq k
\end{align*}
A model for the EEG measurements is now established.
From the model the aim is to recover the source matrix $\textbf{X}_s \forall s$, which gives us the separated original brain signals as intended by the problem statement.      
In the next section the solution method is presented and discussed -- outlining the remaining chapters of the thesis.    

\section{Solution Method}\label{sec:sol_met}
It is now justified that the EEG measurements can be modelled by the multiple measurement vector model defined by the system of linear equations \eqref{eq:MMV_seg}, including an additional noise.
By the problem statement cf. chapter \ref{ch:problemstatement} the aim is to recover the source vector $\textbf{X}$, in the case where the number of sensors is less than the number of sources, $M < N$. That is recovering $\textbf{X}$ from an under-determined linear system. Therefore, the solution must be found in the infinite solution space, provided that one solution exists, thus simple linear algebra can not be used. 
%%% er det her godt? eller nødvendig overhoved?
However, by considering numerical methods such as mathematical optimization it is possible restrict the solution by some constraint and then find the unique optimal solution which respect to a defined cost relative to the solution.
%%%
The theory of compressive sensing dictates a framework for solving an under-determined system when $\textbf{X}$ is known to have non-zeros entries. Specifically a unique solution $\textbf{X}$ can be found when $\textbf{X}$ is $M$-sparse, cf. theorem \ref{th:CS_A} in appendix \ref{app_sec:CS}. When $\textbf{A}$ is unknown, as it is in the current case, the concept of dictionary learning can be used to determine $\textbf{A}$, again under the assumption that $\textbf{X}$ is $M$-sparse.  

As discussed in chapter \ref{ch:motivation} the aim of this thesis is to overcome the limitation of fewer sources than measurements, which is the limitation of compressive sensing.  

A method to overcome this limitation, with respect to learning $\textbf{A}$, is the Covariance-domain dictionary learning (Cov-DL) method\cite{Balkan2015}, introduced in chapter \ref{ch:motivation}. The method manage to leverage the increased dimensionality of the covariance domain in order to the allow the theory of compressive sensing to apply to an under-determined system.     
%problem to covariance domain to increase the dimensionality of the problem -- instead the number of activations $k$ is now limited by $k \leq \frac{M(M+1)}{2}$ which allows us to have $k \geq M$.
%A problem comes with the transformation into the covariance-domain.
%With the recovering process taking place in the covariance-domain only the mixing matrix $\mathbf{A}$ is recoverable as it can transformed back to the time-domain without losing its characteristics. 
However, this method does only apply to the process of learning $\textbf{A}$\todo{as the relation between $\textbf{X}$ and the corresponding $\delta$ is unknown(?) det kan vi ikke skrive her kan vi?}, hence a different approach is necessary to recover $\textbf{X}$.
%As the activations are increase in the covariance-domain, the source matrix $\mathbf{X}$ recovered are not the true source matrix as it cannot be transformed back to the time-domain without losing the sparsity and therefore not be a unique recovering.

For recovering $\mathbf{X}$, given both $\textbf{Y}$ and $\textbf{A}$, the method Multiple sparse Bayesian learning (M-SBL), introduced in chapter \ref{ch:motivation}, is considered.
A method which ensure that the sparsity holds(?).
O. Balkan \cite{Balkan2014} did also, in 2014, proposed a method which could identify the sources, in the time-domain, by creating a likelihood which ensure the wanted sparsity of the source matrix $\mathbf{X}$ and controlled by some variance. This method is called multiple sparse Bayesian learning (M-SBL) and takes advantage of a Bayesian approach. 
In \cite{Balkan2014} a variance dependent log-likelihood which has been induce by a empirical prior that ensure sparsity of the likelihood has been constructed to be minimised with respect to the variance. 
From the log-likelihood an estimate for the source matrix $\mathbf{X}$ is drawn with respect to the support set $S$ which has been influence by variance used in the minimization\todo{evt. tilføj afslutning der siger hvad der kommer i kapitlerne?}.

%With these two methods the recovering of $\mathbf{A}$ and $\mathbf{X}$ can be done from the EEG measurements $\mathbf{Y}$ uniquely.



%fjernet indhold:
%The number of $k$ active source in the source vector $\mathbf{x} \in \mathbb{R}^N$ can be view as non-zeros -- the non-activations are described by zeros. 
%By using this joint information it is possible to recover $\mathbf{X}_s$ from fewer measurements \cite[p. 43]{CS}.

