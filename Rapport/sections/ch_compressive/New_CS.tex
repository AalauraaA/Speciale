\section{System of Linear Equations}\label{sec:SMV}
Let $\mathbf{y} \in \mathbb{R}^M$ be some vector. By basic linear algebra $\mathbf{y}$ can be described as a linear combination of a coefficient matrix $\mathbf{A} \in \mathbb{R}^{M \times N}$ and some scalar vector $\mathbf{x} \in \mathbb{R}^N$ such that
\begin{align}\label{eq:SMV_model}
\mathbf{y} = \mathbf{Ax},
\end{align}
%The measurement vector $\mathbf{y}$ consist of $M$ measurements, $\mathbf{x}$ is an unknown vector of $N$ elements, and the coefficient matrix $\mathbf{A}$ models the linear measurement process column-wise.
Let $\mathbf{y}$ and $\mathbf{A}$ be known, then  
\ref{eq:SMV_model} makes a system of $M$ linear equations with $N$ unknowns, referred to as a linear system. 

To solve the linear system \ref{eq:SMV_model} with respect to $\textbf{x}$ one must look at the three different cases that can occur, depending on the relation between the number of equations $M$ and the number of unknowns $N$.
For $M = N$, the system has in general one unique solution, provided that a solution exist(?).  
If the square coefficient matrix $\mathbf{A}$ has full rank the solution can be found by inverting $\mathbf{A}$.
%-- $\mathbf{A}$ consists of linearly independent columns or rows -- 
\begin{align*}
\mathbf{x} = \mathbf{A}^{-1} \mathbf{y}.
\end{align*}

For $M > N$ the system is over-determined. In general there is no solution to an over-detemined. An/The exception occur when the system contains enough linearly dependent equations.    
For $M < N$ the system is under-determined. There exist infinitely many solutions to an under-determined system, provided that one solution exist\todo{note: skal vi nævne løsnings metoder for underdetermined system her?} \cite[p. ix]{CS}.  

Consider now $\mathbf{y} \in \mathbb{R}^M$ as observed measurements from $M$ EEG sensors at time $t$. 
The linear system \ref{eq:SMV_model} is then considered as a single measurement vector (SMV) model.
Remember from chapter \ref{ch:motivation} that EEG measurements basically is a mixture of original brain signals affected by volumen condiction.  
Modelling the EEG measurements by the SMV model embody the following assumptions/interpretations. $\textbf{x}$ is seen as the original brain signal sources, each entry representing the signal of one source. 
Thus, $\textbf{x}$ is $\textbf{A}$ referred to as the source vector. $N$ is considered the maximum number of sources, however zero-entries may occur. Hence let $k$ denote the number of non-zero entries in $\textbf{x}$ referred to as the active sources at time $t$.   
The coefficient matrix $\textbf{A}$ models the volume conduction by mapping the source vector from $\mathbb{R}^N$ to $\mathbb{R}^M$. $\textbf{A}$ is $\textbf{A}$ referred to a the mixing matrix.             

\section{Multiple Measurement Vector Model of EEG}\label{sec:MMV}
In practise EEG measurements are sampled over time by a certain sample frequency. 
Thus multiple EEG measurement vectors are achieved.
Let $L$ be the total number of samples. Now the  
the SMV model is expanded to include $L$ measurement vectors:
\begin{align}\label{eq:MMV_model}
\mathbf{Y} = \mathbf{AX}+\textbf{E},
\end{align}
now $\mathbf{Y} \in \mathbb{R}^{M \times L}$ is the observed measurement matrix, $\mathbf{X} \in \mathbb{R}^{N \times L}$ is the source matrix, and $\mathbf{A} \in \mathbb{R}^{M \times N}$ is the mixing matrix. 
Furthermore $\textbf{E} \in \mathbb{R}^{M \times L}$ is consider an additional noise matrix, to be expected from psychical measurements.  
The model is now referred to as the multiple measurement vector (MMV) model.
As for \eqref{eq:SMV_model} the solution set of the linear system \eqref{eq:MMV_model} depends on the relation between $N$ and $M$ \cite[p. 42]{CS}. 

In chapter \ref{ch:motivation} it is specified that the case of more sources than sensors, $N>M$, is the case of interest in this thesis.  
%flytte sidste del?

\subsection{Segmentation}
In\todo{man ikke teoritisk modellere de non-stationære tilfælde?} chapter \ref{ch:motivation} is it argued that EEG measurements are only stationary with small segments. 
Hence the following segmentation is considered.   
%As the EEG measurements are non-stationary it can be hard to say something about the activations in the source matrix $\mathbf{X}$ as the sources fluctuates between be active and non-active -- depending on the situation the EEG measurements are measured in.
%Therefore it is of interest that the EEG measurements becomes stationary.
%This could possibly be done by segmentation of the EEG measurements as the measurements becomes stationary in short time period.

Let $f$ be the sample frequency of the observed EEG measurements $\mathbf{Y}$ and let $s$ denoted a segment index.
As such the observed EEG measurements can be divided into segments  $\mathbf{Y}_s \in \mathbb{R}^{M \times L_s}$, possibly overlapping. Here $L_s = t_{s}f$ where $t_s$ is the length of the segment in seconds\todo{her er t\_s ikke konstant med afhængig af index s, ikke?}. For each segment the MMV model \eqref{eq:MMV_model} holds and is rewritten into
\begin{align}\label{eq:MMV_seg}
\mathbf{Y}_s = \mathbf{AX}_s + \textbf{E}_s, \quad \forall s.
\end{align}
Due to a segment being stationary it is assumed that each source remains either active or non-active throughout the segment -- a non-active source gives a zero-row in $\textbf{X}$.
Thus, $\mathbf{X}_s$, consists of $k$ non-zero rows -- the active sources.

.....

The support of the segmented source matrix $\mathbf{X}_s$ denotes the index set of non-zero rows of $\mathbf{X}_s$. 
As mentioned before, $\mathbf{X}_s$ has at most $k$ non-zero rows, the non-zero values occur in common location for all columns. 
By using this joint information it is possible to recover $\mathbf{X}_s$ from fewer measurements \cite[p. 43]{CS}.

\section{Solution Methods}\label{sec:sol_met}
For the EEG measurements the MMV model is an under-determined linear system with less sensors than sources, $M < N$, and therefore a solution must be found in the infinitely solution space, provided one solution exists -- simple linear algebra can not be used.

O. Balkan \cite{Balkan2015} proposed in 2015 a method which can recover a mixing matrix $\mathbf{A}$ from a under-determined segmented MMV model. 
The method is called covariance-domain dictionary learning (Cov-DL).
The background of this method take a part of compressive sensing. 
Compressive sensing is method which can recover the unknown signals and mixing matrix from under-determined linear systems. 
With this method a new term is introduce as sparsity. Sparsity can be view as the number of activation within the sources $k$ -- the number of non-zeros indices in $\mathbf{x}$.
One thing which is necessary for compressive sensing is the the number of activations $k$ is limited by the number of sensors, $k \leq M$, to ensure uniquely recovery of $\mathbf{x}$.
This is not always the case in the EEG measurements as more activations than sensors occurs when using low-density equipment to measure the EEG measurements.
To overcome this problem O. Balkan transform the compressive sensing problem to covariance domain to increase the dimensionality of the problem -- instead the number of activations $k$ is now limited by $k \leq \frac{M(M+1)}{2}$ which allows us to have $k \geq M$.

O. Balkan \cite{Balkan2014} did also, in 2014, proposed another method which could identify the sources of a non-segmented MMV model. 
This method is called multiple sparse bayesian learning (M-SBL).
The method take advantage of a Bayesian approach as O. Balkan constructs a variance dependent log-likelihood which as been induce by a empirical prior that ensure sparsity of the likelihood. 
By minimisation of the variance of the log-likelihood an estimate for the source matrix can be drawn from the log-likelihood.


fjernet indhold:
From chapter \ref{ch:motivation} the linear system of interest consists of an observed EEG measurement vector $\mathbf{y} \in \mathbb{R}^M$ with $M$ sensors and a unknown source vector $\mathbf{x} \in \mathbb{R}^N$ with $N$ sources. 
The matrix $\mathbf{A} \in \mathbb{R}^{M \times N}$ is the mixing matrix.
As described in chapter \ref{ch:motivation}, the case of interest is an under-determined linear system, $M < N$, as the EEG measurement have less sensors than sources -- hence a solution has to be found within the infinite solution set. 
For the source vector $\mathbf{x}$ the activation of the sources is denoted by $k$ and there are at most $k \leq N$ activations in the EEG measurements. 
To find the solution of the linear system \eqref{eq:SMV_model} the number of activations $k$ must also be included in the recovering.

The number of $k$ active source in the source vector $\mathbf{x} \in \mathbb{R}^N$ can be view as non-zeros -- the non-activations are described by zeros. To count non-zeros entries of a vector the $\ell_0$-norm can be used:
\begin{align*}
\Vert \mathbf{x} \Vert_0 := \text{card}(\text{supp}(\mathbf{x})).
\end{align*}
The function $\text{card}(\cdot)$ gives the cardinality of the input and the support of $\mathbf{x}$, consisting of the non-zeros entries, is given as
\begin{align*}
\text{supp}(\mathbf{x}) = \{ j \in [N] \ : \ x_j \neq 0 \},
\end{align*} 
where $[N]$ is a set of integers $\lbrace 1, 2, \hdots, N \rbrace$ \cite[p. 41]{FR}. 

