\section{System of Linear Equations}\label{sec:SMV}
Let $\mathbf{y} \in \mathbb{R}^M$ be some vector. By basic linear algebra $\mathbf{y}$ can be described as a linear combination of a coefficient matrix $\mathbf{A} \in \mathbb{R}^{M \times N}$ and some scalar vector $\mathbf{x} \in \mathbb{R}^N$ such that
\begin{align}\label{eq:SMV_model}
\mathbf{y} = \mathbf{Ax},
\end{align}
%The measurement vector $\mathbf{y}$ consist of $M$ measurements, $\mathbf{x}$ is an unknown vector of $N$ elements, and the coefficient matrix $\mathbf{A}$ models the linear measurement process column-wise.
Let $\mathbf{y}$ and $\mathbf{A}$ be known, then  
\ref{eq:SMV_model} makes a system of $M$ linear equations with $N$ unknowns, referred to as a linear system. 

To solve the linear system \ref{eq:SMV_model} with respect to $\textbf{x}$ one must look at the three different cases that can occur, depending on the relation between the number of equations $M$ and the number of unknowns $N$.
For $M = N$, the system has in general one unique solution, provided that a solution exist(?).  
If the square coefficient matrix $\mathbf{A}$ has full rank the solution can be found by inverting $\mathbf{A}$.
%-- $\mathbf{A}$ consists of linearly independent columns or rows -- 
\begin{align*}
\mathbf{x} = \mathbf{A}^{-1} \mathbf{y}.
\end{align*}

For $M > N$ the system is over-determined. In general there is no solution to an over-detemined system. An/The exception occur when the system contains a sufficient amount of linearly dependent equations.    
For $M < N$ the system is under-determined. There exist infinitely many solutions to an under-determined system, provided that one solution exist\todo{note: skal vi nævne løsnings metoder for underdetermined system her?} \cite[p. ix]{CS}.  

Consider now $\mathbf{y} \in \mathbb{R}^M$ as the observed measurements from $M$ EEG sensors at time $t$. 
The linear system \ref{eq:SMV_model} is then considered as a single measurement vector (SMV) model.
Remember from chapter \ref{ch:motivation} that EEG measurements basically is mixture of original brain signals affected by volume conduction.  
Modelling the EEG measurements by the SMV model embody the following assumptions/interpretations. $\textbf{x}$ is seen as the original brain signal sources, each entry representing the signal of one source. 
Thus, $\textbf{x}\in \mathbb{R}^N$ is referred to as the source vector. $N$ is considered the maximum number of sources, however zero-entries may occur. Let $k$ denote the number of non-zero entries in $\textbf{x}$, referred to as the active sources at time $t$.   
The coefficient matrix $\textbf{A}$ models the volume conduction by mapping the source vector from $\mathbb{R}^N$ to $\mathbb{R}^M$. $\textbf{A}$ is referred to as the mixing matrix.             

\section{Multiple Measurement Vector Model of EEG}\label{sec:MMV}
In practise EEG measurements are sampled over time by a certain sample frequency. 
Thus multiple EEG measurement vectors are achieved.
Let $L$ be the total number of samples. Now the  
the SMV model is expanded to include $L$ measurement vectors:
\begin{align}\label{eq:MMV_model}
\mathbf{Y} = \mathbf{AX}+\textbf{E},
\end{align}
now $\mathbf{Y} \in \mathbb{R}^{M \times L}$ is the observed measurement matrix, $\mathbf{X} \in \mathbb{R}^{N \times L}$ is the source matrix, and $\mathbf{A} \in \mathbb{R}^{M \times N}$ is the mixing matrix. 
Furthermore $\textbf{E} \in \mathbb{R}^{M \times L}$ is consider an additional noise matrix, to be expected from psychical measurements.  
The model is now referred to as the multiple measurement vector (MMV) model.
As for \eqref{eq:SMV_model} the solution set of the linear system \eqref{eq:MMV_model} depends on the relation between $N$ and $M$ \cite[p. 42]{CS}. 

In chapter \ref{ch:motivation} it is specified that the case of more sources than sensors, $N>M$, is the case of interest in this thesis.  
%flytte sidste del?

\subsection{Segmentation}
In\todo{man kan ikke teoretisk modellere de non-stationære tilfælde med samme model?} chapter \ref{ch:motivation} it is argued that EEG measurements are only stationary within small segments. 
Hence the following segmentation is considered.   
%As the EEG measurements are non-stationary it can be hard to say something about the activations in the source matrix $\mathbf{X}$ as the sources fluctuates between be active and non-active -- depending on the situation the EEG measurements are measured in.
%Therefore it is of interest that the EEG measurements becomes stationary.
%This could possibly be done by segmentation of the EEG measurements as the measurements becomes stationary in short time period.

Let $f$ be the sample frequency of the observed EEG measurements $\mathbf{Y}$ and let $t_s$ be the length of a segment. 
Here $s$ is the segment index. 
As such the observed EEG measurements can be divided into stationary segments  $\mathbf{Y}_s \in \mathbb{R}^{M \times L_{s}}$, possibly overlapping, where $L_s = t_{s}f$ \todo{her er ts ikke konstant med afhængig af index s, ikke?}. 
For each segment the MMV model \eqref{eq:MMV_model} holds and is rewritten into
\begin{align}\label{eq:MMV_seg}
\mathbf{Y}_s = \mathbf{AX}_s + \textbf{E}_s, \quad \forall s.
\end{align}
Due to a segment being stationary it is assumed that each source remains either active or non-active throughout the segment.
Thus, $\mathbf{X}_s$, consists of $k$ non-zero rows -- the active sources.

In order to characterise the source matrix with respect the amount of non-zero rows the term row sparseness is considered.  
By common definition the support of the segmented source matrix $\text{supp}(\mathbf{X}_s)$ denotes the index set of non-zero rows of $\mathbf{X}_s$.
%\begin{align*}
%\text{supp}(\mathbf{X}) = \{ j \in [N] \ : \ X_j\cdot \neq 0 \}.
%\end{align*}
%where $[N]$ is a set of integers $\lbrace 1, 2, \hdots, N \rbrace$ \cite[p. 41]{FR}.  
To count the non-zeros row of a matrix the $\ell_0$-norm is defined:
\begin{align*}
\Vert \mathbf{X} \Vert_0 := \text{card}(\text{supp}(\mathbf{X})),
\end{align*}
where the function $\text{card}(\cdot)$ gives the cardinality of the input set. 
$\textbf{X}_s$ is said to be $k$-sparse if it contains at most $k$ non-zeros rows:
\begin{align*}
\Vert \mathbf{X}_s \Vert_0 \leq k
\end{align*}
A model for the EEG measurements is now established.
From the model the aim is to recover the source matrix $\textbf{X}_s \forall s$, which gives us the separated original brain signals as intended by the problem statement.      
In the next section our solution method is presented and discussed before the algorithms are established in the next chapters.    

\section{Solution Methods}\label{sec:sol_met}
It is now justified that the EEG measurements can be modelled by the multiple measurement vector model defined by the system of linear equations \eqref{eq:MMV_seg}, including an additional noise.
From the problem statement cf. chapter \ref{ch:problemstatement} it is given that the aim is to recover the source vector $\textbf{X}$, in the case where the number of sensors is less than the number of sources, $M < N$. That is recovering $\textbf{X}$ from an under-determined system. Therefore, a solution must be found in the infinitely solution space, provided that one solution exists -- simple linear algebra can not be used.

       


O. Balkan \cite{Balkan2015} proposed in 2015 a method which can recover a mixing matrix $\mathbf{A}$ from a under-determined segmented MMV model. 
The method is called covariance-domain dictionary learning (Cov-DL).
The background of this method take a part of compressive sensing. 
Compressive sensing is method which can recover the unknown signals and mixing matrix from under-determined linear systems. 
With this method a new term is introduce as sparsity. Sparsity can be view as the number of activation within the sources $k$ -- the number of non-zeros indices in $\mathbf{x}$.
One thing which is necessary for compressive sensing is the the number of activations $k$ is limited by the number of sensors, $k \leq M$, to ensure uniquely recovery of $\mathbf{x}$.
This is not always the case in the EEG measurements as more activations than sensors occurs when using low-density equipment to measure the EEG measurements.
To overcome this problem O. Balkan transform the compressive sensing problem to covariance domain to increase the dimensionality of the problem -- instead the number of activations $k$ is now limited by $k \leq \frac{M(M+1)}{2}$ which allows us to have $k \geq M$.

O. Balkan \cite{Balkan2014} did also, in 2014, proposed another method which could identify the sources of a non-segmented MMV model. 
This method is called multiple sparse bayesian learning (M-SBL).
The method take advantage of a Bayesian approach as O. Balkan constructs a variance dependent log-likelihood which as been induce by a empirical prior that ensure sparsity of the likelihood. 
By minimisation of the variance of the log-likelihood an estimate for the source matrix can be drawn from the log-likelihood.


fjernet indhold:


The number of $k$ active source in the source vector $\mathbf{x} \in \mathbb{R}^N$ can be view as non-zeros -- the non-activations are described by zeros. 

By using this joint information it is possible to recover $\mathbf{X}_s$ from fewer measurements \cite[p. 43]{CS}.

