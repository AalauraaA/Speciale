\section{Single Measurement Vector Model}
Some measurement vector $\mathbf{y} \in \mathbb{R}^M$ can be described as a linear combinations of a coefficient matrix $\mathbf{A} \in \mathbb{R}^{M \times N}$ and some vector $\mathbf{x} \in \mathbb{R}^N$ such that
\begin{align}\label{eq:SMV_model}
\mathbf{y} = \mathbf{Ax}.
\end{align}
The measurement vector $\mathbf{y}$ consist of $M$ measurements, $\mathbf{x}$ is an unknown vector of $N$ elements, and the coefficient matrix $\mathbf{A}$ models the linear measurement process column-wise. 
\ref{eq:SMV_model} makes a system of linear equations with $M$ equations and $N$ unknowns and will be referred to as a linear system for the rest of the master thesis. In other literature, the linear system is also referred to as a single measurement vector (SMV) model.

To solve the linear system one must look at the different cases that occurs for the number of $M$ equations and $N$ unknowns.
In the case where the number of equations equal the number of unknowns, $M = N$, the coefficient matrix $\mathbf{A}$ becomes a square matrix. 
A solution for the linear system, $\mathbf{x}^\ast$, can be found, provided that a solution exist, if the square coefficient matrix $\mathbf{A}$ has full rank -- $\mathbf{A}$ consists of linearly independent columns or rows -- as the square matrix can be inverted and the solution is achieved by 
$$
\mathbf{x}^\ast = \mathbf{A}^{-1} \mathbf{y}.
$$
Such linear system, when $M = N$, is called determined and there will exist a unique solution, provided that a solution exist. 
In the cases where $M > N$ and $M < N$ the linear system is called over-determined and under-determined, respectively. 
For an over-determined system there does not exist a solution and for under-determined systems there exist infinitely many solutions, provided that one solution exist \cite[p. ix]{CS} \todo{Her skal vi være opmærksom på at dette ikke altid gælder. At der godt kan være en løsning i over-determined tilfælde (I følge Rasmus) - Laura}.

From chapter \ref{ch:motivation} the linear system of interest consists of a observed EEG measurement vector $\mathbf{y} \in \mathbb{R}^M$ with $M$ sensors and a unknown source vector $\mathbf{x} \in \mathbb{R}^N$ with $N$ sources. 
The matrix $\mathbf{A} \in \mathbb{R}^{M \times N}$ is the mixing matrix.
As described in chapter \ref{ch:motivation}, the case of interest is an under-determined linear system, $M < N$, as the EEG measurement have less sensors than sources -- hence a solution has to be found within the infinite solution set. 
For the source vector $\mathbf{x}$ the activation of the sources is denoted by $k$ and there are at most $k \leq N$ activations in the EEG measurements. 
To find the solution of the linear system \eqref{eq:SMV_model} the number of activations $k$ must also be included in the recovering.

The number of $k$ active source in the source vector $\mathbf{x} \in \mathbb{R}^N$ can be view as non-zeros -- the non-activations are described by zeros. To count non-zeros entries of a vector the $\ell_0$-norm can be used:
\begin{align*}
\Vert \mathbf{x} \Vert_0 := \text{card}(\text{supp}(\mathbf{x})).
\end{align*}
The function $\text{card}(\cdot)$ gives the cardinality of the input and the support of $\mathbf{x}$, consisting of the non-zeros entries, is given as
\begin{align*}
\text{supp}(\mathbf{x}) = \{ j \in [N] \ : \ x_j \neq 0 \},
\end{align*} 
where $[N]$ is a set of integers $\lbrace 1, 2, \hdots, N \rbrace$ \cite[p. 41]{FR}. 

\todo{Nu nævner vi jo herover at vi skal finde en løsning i det uendelige løsningsrum men vi går egenligt ikke videre med det da vi i stedet introducer k og MMV. Går der for lang til fra at det er nævnt til at man kommer med et bud? Vi kommer jo først til at nævne det i næste sektion "solution method".}

\section{Multiple Measurement Vector Model}\label{sec:MMV}
As the EEG measurements vary over time multiple EEG measurement vectors are achieved. 
Therefore, to provided a practical use of the linear system the linear system is expanded to include the multiple measurement vectors, such that the linear system now becomes a multiple measurement vector (MMV) model:
\begin{align}\label{eq:MMV_model}
\mathbf{Y} = \mathbf{AX}+\textbf{E},
\end{align}
where $\mathbf{Y} \in \mathbb{R}^{M \times L}$ is the observed measurement matrix, $\mathbf{X} \in \mathbb{R}^{N \times L}$ is the source matrix, and $\mathbf{A} \in \mathbb{R}^{M \times N}$ is the mixing matrix. 
By expanding the system to include multiple measurement vectors some noise will occur. 
This is added to the linear system as a noise matrix $\textbf{E} \in \mathbb{R}^{M \times L}$. 
$L$ denotes the number of observed EEG measurement vectors each consisting of $M$ measurements, that is $L$ samples given. 
For $L = 1$ the linear system \eqref{eq:MMV_model} will just be the SMV model \eqref{eq:SMV_model}.
As for the linear system \eqref{eq:SMV_model} the MMV model \eqref{eq:MMV_model} is under-determined with $M < N$ and $k < M$ \cite[p. 42]{CS}.

As the EEG measurements are non-stationary it can be hard to say something about the activations in the source matrix $\mathbf{X}$ as the sources fluctuates between be active and non-active.
Therefore it is of interest that the EEG measurements becomes stationary.
This could possibly be done by segmentation of the EEG measurements as the measurements becomes stationary in short time period.

Let $f$ be the sample frequency of the observed EEG measurements $\mathbf{Y}$ and let $s$ denoted a segment index \todo{f er samples pr sek., L er antal sampels i alt, Ls er antal samples pr segment og ts er længen pr segment i sekunder}.
As such the observed EEG measurements can be divided into segments $\mathbf{Y}_s \in \mathbb{R}^{M \times t_s f}$, possibly overlapping, where $t_s$ is the length of the segments in seconds. For each segment the MMV model \eqref{eq:MMV_model} still holds and is rewritten into
\begin{align}\label{eq:MMV_seg}
\mathbf{Y}_s = \mathbf{AX}_s + \textbf{E}_s, \quad \forall s.
\end{align}
As the segmented EEG measurement matrix $\mathbf{Y}_s$ now consists of stationary measurement, the segmented source matrix $\mathbf{X}_s$, consisting of the vectors $\mathbf{x}_{s1}, \dots, \mathbf{x}_{sL}$ which have been stacked column-wise, will now consists of at most $k$ non-zero rows -- the activations of sources. 
The support of segmented source matrix $\mathbf{X}_s$ denotes the index set of non-zero rows of $\mathbf{X}_s$. 
As mentioned before, $\mathbf{X}_s$ has at most $k$ non-zero rows, the non-zero values occur in common location for all columns. 
By using this joint information it is possible to recover $\mathbf{X}_s$ from fewer measurements \cite[p. 43]{CS}.

\section{Solution Methods}\todo{Mangler dette afsnit - Laura}
For the EEG measurements the MMV model is an under-determined linear system with less sensors than sources, $M < N$, and therefore a solution must be found in the infinitely solution space, provided one solution exists.

O. Balkan \cite{Balkan2015} proposed in 2015 a method which can recover a mixing matrix $\mathbf{A}$ from a under-determined segmented MMV model.
The background of this method is take a part of compressive sensing. 
Compressive sensing is method which can recover the unknown signals and mixing matrix from under-determined linear systems. 
With this method a new term is introduce as sparsity. Sparsity can be view as the number of activation within the sources $k$ -- the number of non-zeros indices in $\mathbf{x}$.
One thing which is necessary for compressive sensing is the the number of activations $k$ is limited by the number of sensors, $k \leq M$, to ensure uniquely recovery of $\mathbf{x}$.
This is not always the case in the EEG measurements as more activations than sensors occurs when using low-density equipment to measure the EEG measurements.
To overcome this problem O. Balkan transform the compressive sensing problem to covariance domain to increase the dimensionality of the problem -- instead the number of activations $k$ is now limited by $k \leq \frac{M(M+1)}{2}$ which allows us to have $k \geq M$.

O. Balkan \cite{Balkan2014} did also proposed another method which could identify the sources of the MMV model \eqref{eq:MMV_model} -- the non-segmented linear system.



