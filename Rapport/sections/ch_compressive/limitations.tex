\section{Limitations of compressive sensing}
Through this chapter the concept of sparse signal recovery has been explained. 
The essential limitation of signal recovery from an under-determined system is that $k\leq M$ is necessary in order to uniquely recover the $k$-sparse signal $\textbf{X}\in \mathbb{R}^N$ from the measurements $\textbf{Y} \in \mathbb{R}^M$. 
That is the number of measurements must be greater than the number of active sources within the signal to be recovered.  
Similarly it is not possible to recover the true dictionary $\textbf{A}$ by dictionary learning methods if $k>M$. 
Because in that case any random dictionary of full rank can be used to create $\textbf{y}$ from $\geq M$ basis vectors \cite[p. 30]{phd2015}. 
\\
When considering source recovery from EEG measurements, described in section \ref{sec:EEG}, it is not reasonable to assume that $k<M$ and especially not in the case of low density EEG measurements. 
This motivates the next two chapters where the possibility of source recovery for $k>M$ is explored. 
The methods, proposed recently by O. Balkan, are taking advantage of the covariance domain and.
\todo{skal dette argumenteres yderligere, som værende uafhængig af motivations kapitlet?}  



