\chapter{Conclusion 2}
The main purpose of this thesis was to prove a state of the art result for solving the EEG inverse problem, with respect to the original source signals, in the under-determined case.
Secondary the method is sought improved from a practical perspective.

A main algorithm was proposed, based on the state of the art methods covariance-domain dictionary learning (COV-DL) \cite{Balkan2015} and multiple sparse Bayesian learning (M-SBL) \cite{Balkan2014}, for recovery of respectively the mixing matrix and the source signals, from the EEG inverse problem. 
From the initial implementation verification of the COV-DL method, the method was found to fail. Based on brief analysis of the error, it is concluded that the scientific article proposing the method did not have a sufficient degree of reproducibility.    
The implementation verification of the M-SBL method was successful when the true mixing matrix was provided. From this it can be concluded that the corresponding scientific article provided a sufficient degree of reproducibility.  

To replace the recovery of the mixing matrix from COV-DL within, the main algorithm, a fixed matrix was chosen, based on empirical tests.
By this modification of the main algorithm and the corresponding tests it can not be expected that the main algorithm manage to provide a sufficient recovery of the source signals. The performance of the resulting main algorithm applied to synthetic data was found to be both insufficient and fluctuating, indicating an unreliability which complicates any useful conclusion.      
Though, with respect to investigating the extent of the resulting main algorithm, the performance was tested on real EEG data. 
The results from the recovery of source signals, in three cases of varying degree of over-completeness, was compared to the solutions provided by independent component analysis (ICA) on the complete system. From the analysis of the results a potentially reliable recovery of sources was seen for the complete system, while for the over complete cases which is of interest, the performance can not be evaluated as sufficient. Thus as expected, it must be concluded that the main algorithm do not provide a sufficient recovery of the source signals. Furthermore the extent of the algorithm can not be determined based on the fluctuation results. 
An alternative test was conducted as an attempt to analysis the recovered source signal from a different perspective. From this analysis no significant founding was discovered, thus the above conclusion is preserved. 

With respect to the practical perspective, the issue of the unknown number of active source relative to the total number of sources was investigate through empirical tests. A method was proposed with respect to identification of non-active source after the recovery. The method showed potential when applied to results provided by use of the true mixing matrix. However, when increasing the over completeness as desired       difficulties arise. Thus it is concluded that the method do not provide a sufficient estimate of the number of active sources. But the found potential suggest that further work on the method may allow the possibility of reaching a reliable estimate.  

Overall, it is concluded that a recreation of the specified state of the art methods for source recovery was not successfully provided by the proposed main algorithm. Furthermore the proposed alternative to the estimation of the mixing matrix, did not provide reliable results hence the conclusion remains. 
Secondary it is concluded that a potential is seen with respect using the M-SBL method in practice, considering the unknown number of active sources.      
 



