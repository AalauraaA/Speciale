\subsection{Under-determined System}\label{sec:cov1}
When $\widetilde{M} < N$ the transformed model \eqref{eq:cov1} makes an under-determined system. 
This is similar to the original MMV model \eqref{eq:MMV_model} being under-determined when $M < N$. 
Thus, from the theory of compressive sensing, it is again possible to solve the under-determined system if a certain sparsity is withheld, namely $\boldsymbol{\delta}_s$ being $\widetilde{M}$-sparse.
Assuming the sufficient sparsity on $\boldsymbol{\delta}_s$ is withheld it is possible to learn the dictionary matrix of the covariance-domain $\mathbf{D}_s$. 
This can be done by traditional dictionary learning methods applied to the measurements represented in the covariance-domain $\text{vec}\left(\widehat{\boldsymbol{\Sigma}}_{\mathbf{Y}_s}\right)$ for all segments $s$.

\subsubsection{Dictionary Learning}\label{sec:dictionarylearning}
As mentioned, within the theory of compressive sensing the matrix $\mathbf{A}$ is referred to as a dictionary matrix. 
When the dictionary matrix is not known a priori it is essential how to choose the dictionary matrix in order to achieve the best recovery, of a sparse vector $\mathbf{x}$ from the observed measurements $\mathbf{y}$. 
This is clarified from the proof of theorem \ref{th:CS_A} in appendix \ref{app_sec:CS}. 
One choice is a pre-constructed dictionary. 
In many cases the use of a pre-constructed dictionary results in simple and fast algorithms for reconstruction of $\mathbf{x}$ \cite{Elad_book}. 
However, a pre-constructed dictionary is typically fitted to a specific kind of data. 
For instance the discrete Fourier transform or the discrete wavelet transform are used especially for sparse representation of images \cite{Elad_book}. 
Hence the results of using such dictionaries depend on how well they fit the data of interest, which is establishing a certain limitation. 

The alternative option is to consider an adaptive dictionary based on a set of training data that resembles the data of interest. 
For this purpose learning methods are considered to empirically construct a dictionary. 
There exist several dictionary learning algorithms. One is the K-SVD algorithm which was presented in 2006 by Elad et al. and found to outperform pre-constructed dictionaries, when computational cost is of secondary interest \cite{Elad2006}. 
The concept of the K-SVD algorithm is introduced here, and the more detailed algorithm is to be found in appendix \ref{app_sec:K-SVD_alg}. 

Consider, from the general MMV model \eqref{eq:MMV_model}, the measurement matrix $\mathbf{Y} \in \mathbb{R}^{M \times L}$ consisting of measurement vectors $\lbrace \mathbf{y}_j \rbrace_{j=1}^L$. Let the set of measurement vectors make a set of $L$ training examples each forming a linear system
\begin{align*}
\mathbf{y}_j = \mathbf{A} \mathbf{x}_j.
\end{align*}
From the linear system one can learn a suitable dictionary $\hat{\mathbf{A}}$, and the sparse representation of the source matrix $\hat{\mathbf{X}} \in \mathbb{R}^N$ with the source vectors $\lbrace \hat{\mathbf{x}}_j \rbrace_{j=1}^L$.
For a known sparsity constraint $k$ dictionary learning can be defined by the following optimization problem. 
\begin{align}\label{eq:SVD1}
\min_{\mathbf{A}, \mathbf{X}} \sum_{j=1}^{L} \Vert \mathbf{y}_j - \mathbf{A} \mathbf{x}_j \Vert_2^2 \quad \text{subject to} \quad \Vert \mathbf{x}_j \Vert_0 \leq k, \ 1 \leq j \leq L,
\end{align}
where both $\mathbf{A}$ and $\mathbf{x}_j$ are quantities to be determined.
Learning the dictionary by the K-SVD algorithm consists of joint solving of the optimization problem with respect to $\mathbf{A}$ and $\mathbf{X}$. 
An initial $\mathbf{A}_0 = [\mathbf{a}_1, \dots, \mathbf{a}_N]$ is chosen and the corresponding $\mathbf{X}_0 = [\mathbf{x}_1, \dots, \mathbf{x}_L]$ is determined, where $\mathbf{x}_j = [x_{1j}, \dots, x_{Nj}]^T$. Then, for each iteration an update rule is applied to every column of $\mathbf{A}_0$. That is updating first $\mathbf{a}_j$ for $j = 1, \dots, N$ and then the corresponding row $\mathbf{x}_{i\cdot}$ where $i = j$. 
More details on the K-SVD algorithm are found in appendix \ref{app_sec:K-SVD_alg}. 
The uniqueness of the dictionary $\hat{\mathbf{A}}$ depends on the recovery sparsity condition. As clarified earlier in section \ref{sec:sol_met} the recovery of a unique solution $\mathbf{X}^\ast$ is only possible if $k < M$ \cite{phd2015}.
%The dictionary learning algorithm K-SVD is a generalisation of the well known K-means clustering also referred to as vector quantization. In K-means clustering a set of $K$ vectors is learned referred to as mean vectors. Each signal sample is then represented by its nearest mean vector. That corresponds to the case with sparsity constraint $k = 1$ and the representation reduced to a binary scalar $x = \lbrace 1, 0 \rbrace$. Further instead of computing the mean of $K$ subsets the K-SVD algorithm computes the SVD factorisation of the $K$ different sub-matrices that correspond to the $K$ columns of $\textbf{A}$.


\subsubsection{Application of Dictionary Learning}
By the establishment of a dictionary learning algorithm, the transformed mixing matrix $\mathbf{D}_s$ from \eqref{eq:cov1} can be learned. 
Remember that \eqref{eq:cov1} is a single vector model, thus in order to make training samples for learning $\mathbf{D}_s$ a further segmentation is needed.
This is segmentation of $\mathbf{Y}_s$ indexed by $s'$. 
For convenience segment index $s$ will be omitted through out this chapter, as the same theory applies to all segments $s$.
Hence, $\mathbf{Y}_{s'}$ referrers to one segment within the outer segment of measurements $\mathbf{Y}_s$. 
   
The transformed and vectorized measurements $\text{vec} \left( \widehat{\boldsymbol{\Sigma}}_{\mathbf{Y}_{s'}} \right), \forall s'$ now makes the training data set for learning $\mathbf{D}$. 
As such each segment $s'$ provides one training sample. 
Thus, the number of available training samples, denoted $L_{s'}$, depends on the chosen length of the segments. In practice this will vary with respect to the total amount of available data. 

K-SVD is applied to the transformed model \eqref{eq:cov1} and $\hat{\mathbf{D}}$ is found. Then it is possible to estimate the mixing matrix $\mathbf{A}$ that generated $\mathbf{D}$ through the known relation 
\begin{align*}
\mathbf{d}_j = \text{vec}(\mathbf{a}_j \mathbf{a}_j^T).
\end{align*}
For each column $\mathbf{d}_j$ for $j = 1, \dots, N$ the following optimization problem is solved with respect to the corresponding column $\mathbf{a}_j$ of the mixing matrix.
\begin{align*}
\min_{\mathbf{a}_j} \| \mathbf{d}_j -\text{vec}\left(\mathbf{a}_j \mathbf{a}_j^T\right) \|_2^2, 
\end{align*}
equivalent to 
\begin{align}
\min_{\mathbf{a}_j} \| \text{vec}^{-1}(\mathbf{d}_j) - \mathbf{a}_j \mathbf{a}_j^T\|_2^2. \label{eq:opt_DL1}
\end{align}
From \cite{Balkan2015} the global minimizer to \eqref{eq:opt_DL1} is given as $\mathbf{a}^{\ast}_j=\sqrt{\lambda_j} \mathbf{b}_j$, without further details or a source.  
Here $\lambda_j$ is the largest eigenvalue of $\text{vec}^{-1}(\mathbf{d}_j)$, where
\begin{align*}
\text{vec}^{-1}(\mathbf{d}_j) = 
\begin{bmatrix}
d_{11} & d_{12} & \cdots & d_{1N} \\
d_{21} & d_{22} & \cdots & d_{2N} \\
\vdots & \vdots & \ddots & \vdots \\
d_{N1} & d_{N2} & \cdots & d_{NN}
\end{bmatrix}, \quad j =1, \dots, N
\end{align*}
and $\mathbf{b}_j$ is the corresponding eigenvector.

From this result each column of the mixing matrix $\mathbf{A}$ can be estimated. 
Hence, it is possible to determine the mixing matrix in the case where the measurements transformed into the covariance-domain makes an under-determined system.
Provided however that the necessary sparsity constraint of $\boldsymbol{\delta}$ being $\widetilde{M}$-sparse is withheld. 
Remember $\widetilde{M} := \frac{M(M+1)}{2}$ thus $M < k$ is allowed and the original sparsity constraint, $\mathbf{X}$ being $M$-sparse, is relaxed. 





