\subsection{Under-determined system}\label{sec:cov1}
When $N > \widetilde{M}$ the transformed model \eqref{eq:cov1} makes an under-determined system.   
This is similar to the original MMV model \eqref{eq:MMV_model} being under-determined  when $N > M$. 
Thus, it is from the theory of compressive sensing again possible to solve the under-determined system if a certain sparsity is withhold. 
Namely $\boldsymbol{\delta}_s$ being $\widetilde{M}$-sparse.
Assuming the sufficient sparsity on $\boldsymbol{\delta}_s$ is withhold it is possible to learn the dictionary matrix of the covariance domain $\mathbf{D}$ by traditional dictionary learning methods applied to the observations represented in the covariance domain $\text{vec}(\widehat{\boldsymbol{\Sigma}}_{\mathbf{Y}_s})$ for all segments $s$.

\subsubsection{Dictionary Learning}\label{sec:dictionarylearning}
Within the theory of compressive sensing the matrix $\textbf{A}$ is referred to as a dictionary matrix, as it determines how a sparse vector $\textbf{x}$ is transformed to the original non-sparse signal. 
When the dictionary is not known i prior it is essential how to choose the the dictionary matrix in order to achieve the best recovery, of the sparse vector $\mathbf{x}$ from the measurements $\mathbf{y}$. 
This is clarified from the proof of theorem \ref{th:CS_A} in appendix \ref{app_sec:CS}. 
One choice is a pre-constructed dictionary. 
In many cases the use of a pre-constructed dictionary results in simple and fast algorithms for reconstruction of $\mathbf{x}$ \cite{Elad_book}. 
However, a pre-constructed dictionary is typically fitted to a specific kind of data. 
For instance the discrete Fourier transform or the discrete wavelet transform are used especially for sparse representation of images \cite{Elad_book}. 
Hence the results of using such dictionaries depend on how well they fit the data of interest, which is creating a certain limitation. 

The alternative option is to consider an adaptive dictionary based on a set of training data that resembles the data of interest. 
For this purpose learning methods are considered to empirically construct a fixed dictionary which can take part in the application. 
There exist several dictionary learning algorithms. One is the K-SVD algorithm which was presented in 2006 by Elad et al. and found to outperform pre-constructed dictionaries, when computational cost is of secondary interest \cite{Elad2006}. The concept of the K-SVD algorithm is introduced, and the more detailed algorithm is to be found in appendix \ref{app_sec:K-SVD_alg}. 
Consider the measurement matrix $\mathbf{Y} \in \mathbb{R}^M$ consisting of measurement vectors $\lbrace \mathbf{y}_j \rbrace_j^L$ making a set of $L$ training examples forming a linear system
\begin{align*}
\mathbf{y}_j = \mathbf{A} \mathbf{x}_j.
\end{align*}
from which one can learn a suitable dictionary $\hat{\mathbf{A}}$, and the sparse representation of the source matrix $\hat{\mathbf{X}} \in \mathbb{R}^N$ with the source vectors $\lbrace \hat{\mathbf{x}}_j \rbrace_j^L$.
For a known sparsity constraint $k$ the dictionary learning can be defined by the following optimisation problem. 
\begin{align}\label{eq:SVD1}
\min_{\mathbf{A}, \mathbf{X}} \sum_{j=1}^{L} \Vert \mathbf{y}_j - \mathbf{A} \mathbf{x}_j \Vert_2^2 \quad \text{subject to} \quad \Vert \textbf{x}_j \Vert_1 \leq k, \ 1 \leq j \leq L.
\end{align}
where both $\textbf{A}$ and $\textbf{x}_j$ are variables to be determined.
Learning the dictionary by the K-SVD algorithm constitute joint solving of the optimization problem with respect to $\mathbf{A}$ and $\mathbf{X}$ respectively. An initial $\textbf{A}_0 = [\textbf{a}_0,\hdots,\textbf{a}_N]$ and the corresponding $\textbf{X}_0$ is determined. Then, for each iteration an update rule is applied to each column of $\textbf{A}_0$, that is updating first $\textbf{a}_j$ and then the corresponding row $\textbf{x}_i$. More details on the K-SVD algorithm is found in appendix \ref{app_sec:K-SVD_alg}. 
The uniqueness of $\mathbf{A}$ depends on the recovery sparsity condition. As clarified earlier in \ref{sec:sol_met} the recovery of a unique solution $\mathbf{X}^\ast$ is only possible if $k < M$ \cite{phd2015}.
%The dictionary learning algorithm K-SVD is a generalisation of the well known K-means clustering also referred to as vector quantization. In K-means clustering a set of $K$ vectors is learned referred to as mean vectors. Each signal sample is then represented by its nearest mean vector. That corresponds to the case with sparsity constraint $k = 1$ and the representation reduced to a binary scalar $x = \lbrace 1, 0 \rbrace$. Further instead of computing the mean of $K$ subsets the K-SVD algorithm computes the SVD factorisation of the $K$ different sub-matrices that correspond to the $K$ columns of $\textbf{A}$.


\subsubsection{Application of dictionary learning}
By the establishments of an dictionary learning algorithm it is now used to learn the transformed dictionary matrix $\textbf{D}$ in \eqref{eq:cov1}. Here the transformed and vectorised measurements $\left\{ \text{vec}\left( \hat{\Sigma}_\textbf{Y}\right), \forall s\right\}$ makes the training dataset. By this note that each segment the original measurement sample constitute only one sample in the covariance domain. Thus the number of training samples depends on the length of a segment.     
When K-SVD is applied and $\mathbf{D}$ is found it is possible to estimate the mixing matrix $\mathbf{A}$ that generated found $\textbf{D}$ through the relation 
\begin{align*}
\mathbf{d}_j = \text{vec}(\mathbf{a}_j \mathbf{a}_j^T).
\end{align*}
Here each column is found from the optimisation problem 
\begin{align*}
\min_{\textbf{a}_j} \| \text{vec}^{-1}(\textbf{d}_j) -\textbf{a}_j\textbf{a}_j^T\|_2^2, 
\end{align*}
for which the global minimizer is $\mathbf{a}^{\ast}_j=\sqrt{\lambda_j} \textbf{b}_j$\todo{redegørelse for resultatet her skal laves}. Here $\lambda_j$ is the largest eigenvalue of $\text{vec}^{-1}(\textbf{d}_j)$,
\begin{align*}
\text{vec}^{-1}(\textbf{d}_j) = 
\begin{bmatrix}
d_{11} & d_{12} & \cdots & d_{1N} \\
d_{21} & d_{22} & \cdots & d_{2N} \\
\vdots & \vdots & \ddots & \vdots \\
d_{N1} & d_{N2} & \cdots & d_{NN}
\end{bmatrix}, \quad j \in [N]
\end{align*}
and $\textbf{b}_j$ is the corresponding eigenvector.

By this each column of the mixing matrix $\textbf{A}$ can be estimated hence it is possible to determine the mixing matrix in the case where the measurements transformed into the covariance domain makes an under-determined system, but the necessary sparsity constraint, $\boldsymbol{\delta}_s$ being $\widetilde{M}$-sparse (instead of $M$-sparse), is withhold.    




