\section{Covariance Domain Representation}\label{sec:cov}
Consider a single sample vector $\textbf{y}_i\in \mathbb{R}^{M}$, containing EEG measurements. 
The covariance of $\textbf{y}_i$ is defined as
\begin{align*}
\boldsymbol{\Sigma}_{\textbf{y}_i}=\mathbb{E}[(\textbf{y}_i-\mathbb{E}[\textbf{y}_i])(\textbf{y}_i-\mathbb{E}[\textbf{y}_i])^T],
\end{align*}
where $\mathbb{E}[\cdot]$ is the expected value operator. 
Let $\textbf{Y}_{s}=\left[\textbf{y}_1, \hdots ,\textbf{y}_{L_s}\right]$ be the observed measurement matrix containing all samples of segment $s$.
Furthermore, assume that all sample vectors $\textbf{y}_i$ within one segment have zero mean and the same distribution.  
Then $\mathbf{Y}_s \in \mathbb{R}^{M \times L_s}$ is described in the covariance-domain by the sample covariance $\widehat{\boldsymbol{\Sigma}}$. The sample covariance is defined as the empirical covariance among the $M$ measurements across the $L_s$ samples. That is a $M \times M$ matrix $\widehat{\boldsymbol{\Sigma}}_{\mathbf{Y}_s} = [\sigma_{jk}]$ with entries 
\begin{align*}
\sigma_{jk}= \frac{1}{L_s}\sum_{i=1}^{L_s} y_{ji} y_{ki}.
\end{align*}
Using matrix notation the sample covariance of $\mathbf{Y}_s$ can be written as
\begin{align*}
\widehat{\boldsymbol{\Sigma}}_{\mathbf{Y}_s} = \frac{1}{L_s} \mathbf{Y}_s \mathbf{Y}_s^T.
\end{align*} 
Similar, the source matrix $\mathbf{X}_s$ can be described in the covariance domain by the sample covariance matrix:
\begin{align*}
\widehat{\boldsymbol{\Sigma}}_{\mathbf{X}_s} &= \frac{1}{L_s} \mathbf{X}_s \mathbf{X}_s^T = \boldsymbol{\Lambda}_s + \boldsymbol{\varepsilon}. 
\end{align*}
The second equality comes from the assumption of the sources within $\mathbf{X}_s$ being uncorrelated. By uncorrelated sources $\mathbf{X}_s$ the sample covariance matrix is asumed to be nearly diagonal. Thus it can be written as $\boldsymbol{\Lambda}_s + \boldsymbol{\varepsilon}$ where $\boldsymbol{\Lambda}_s$ is a diagonal matrix consisting of the diagonal entries of $\widehat{\boldsymbol{\Sigma}}_{\mathbf{X}_s}$ and $ \boldsymbol{\varepsilon}_s$ is a non-diagonal matrix with entries close to zero representing the estimation error \cite{Balkan2015}.

Each segment is now modelled in the covariance-domain.
\begin{widepage}
\begin{align} 
\widehat{\boldsymbol{\Sigma}}_{\mathbf{Y}_s}= \frac{1}{L_s}\mathbf{Y}_s \mathbf{Y}_s^T &=  \frac{1}{L_s}\left( \mathbf{A}_s \mathbf{X}_s + \mathbf{E}_s \right) \left( \mathbf{A}_s \mathbf{X}_s + \mathbf{E}_s\right)^T \nonumber \\ 
&= \frac{1}{L_s} (\textbf{A}_s\textbf{X}_s)(\textbf{A}_s\textbf{X}_s)^T +\frac{1}{L_s}\textbf{E}_s \textbf{E}_s^T + \frac{1}{L_s}\textbf{E}_s (\textbf{A}_s\textbf{X}_s)^T + \frac{1}{L_s}\textbf{A}_s\textbf{X}_s \textbf{E}_s^T  \nonumber \\
&=\frac{1}{L_s} \textbf{A}_s\textbf{X}_s \textbf{X}_s^T \textbf{A}_s^T +\frac{1}{L_s} \textbf{E}_s \textbf{E}_s^T +\frac{1}{L_s} \textbf{E}_s \textbf{X}_s^T \textbf{A}_s^T +\frac{1}{L_s} \textbf{A}_s\textbf{X}_s \textbf{E}_s^T  \nonumber \\
&= \textbf{A}_s(\boldsymbol{\Lambda}_s +\boldsymbol{\varepsilon}_s) \textbf{A}_s^T +\frac{1}{L_s} \textbf{E}_s \textbf{E}_s^T + \frac{1}{L_s}\textbf{E}_s \textbf{X}_s^T \textbf{A}_s^T + \frac{1}{L_s}\textbf{A}_s\textbf{X}_s \textbf{E}_s^T \nonumber \\
&= \textbf{A}_s \boldsymbol{\Lambda}_s \textbf{A}_s^T + \textbf{A}_s \boldsymbol{\varepsilon}_s \textbf{A}_s^T + \frac{1}{L_s}\textbf{E}_s \textbf{E}_s^T +\frac{1}{L_s} \textbf{E}_s \textbf{X}_s^T \textbf{A}_s^T + \frac{1}{L_s}\textbf{A}_s\textbf{X}_s \textbf{E}_s^T \label{eq:noise1} \\
&= \textbf{A}_s \boldsymbol{\Lambda}_s \textbf{A}_s^T + \widetilde{\textbf{E}}_s \label{eq:noise2}
\end{align}
\end{widepage}
From \eqref{eq:noise1} to \eqref{eq:noise2} all terms where noise, $\boldsymbol{\varepsilon}_s$ and $\mathbf{E}_s$, is included, are aggregated in a joint noise term $\widetilde{\textbf{E}}_s$. 
Next, the expression \eqref{eq:noise2} is rewritten through a vectorization. 
Because the covariance matrix $\widehat{\boldsymbol{\Sigma}}_{\mathbf{Y}_s}$ is symmetric it is sufficient to vectorize only the lower triangular part, including the diagonal. 
For this purpose the function $\text{vec}(\cdot)$ is defined to map a symmetric $M \times M$ matrix into a vector of size $\widetilde{M}$ by row-wise vectorization of the lower triangular part. The increased dimension $\widetilde{M}$ becomes 
\begin{align}
\widetilde{M} := \frac{M(M+1)}{2}.
\end{align}
Furthermore, let $\text{vec}^{-1}: \mathbb{R}^{\widetilde{M}} \rightarrow \mathbb{R}^{M\times M}$ be the inverse function for devectorisation. 

Let $\textbf{a}_i$ be the $i$-th column of $\textbf{A}_s$, then the matrix product in \eqref{eq:noise2} can be written in sum form where $\boldsymbol{\Lambda}_{s_{ii}}$ is the $ii$-th entry of $\boldsymbol{\Lambda}_s$. 
\begin{align}
\widehat{\boldsymbol{\Sigma}}_{\mathbf{Y}_s} &= \sum_{i=1}^{N} \textbf{a}_i \boldsymbol{\Lambda}_{s_{ii}} \textbf{a}_i^{T} + \widetilde{\textbf{E}}_s, \quad \boldsymbol{\Lambda}_{s_{ii}} \label{eq:cov_domain}
\end{align}
Applying $\text{vec}(\cdot)$ to \eqref{eq:cov_domain} results in the following expression, which concludes the transformation of model \eqref{eq:MMV_seg} into the covariance-domain. 
\begin{align}
\text{vec}\left( \widehat{\boldsymbol{\Sigma}}_{\mathbf{Y}_s} \right) &= \sum_{i=1}^N \text{vec}(\mathbf{a}_i \mathbf{a}_i^T) \boldsymbol{\Lambda}_{s_{ii}} + \text{vec}( \widetilde{\textbf{E}}_s) \nonumber \\
&= \sum_{i=1}^N \mathbf{d}_i \boldsymbol{\Lambda}_{s_{ii}} + \text{vec}( \widetilde{\textbf{E}}_s) \nonumber \\
&= \mathbf{D}_s \boldsymbol{\delta}_s + \text{vec}( \widetilde{\textbf{E}}_s), \quad \forall s. \label{eq:cov1}
\end{align}
Here $\boldsymbol{\delta}_s \in \mathbb{R}^{N}$ contains the diagonal entries of the source sample covariance matrix $\boldsymbol{\Lambda}_s$
and the matrix $\mathbf{D}_s \in \mathbb{R}^{\widetilde{M} \times N}$ consists of the columns $\mathbf{d}_j = \text{vec}(\mathbf{a}_j \mathbf{a}_j^T)$. Note that $\mathbf{D}$ and $\boldsymbol{\delta}_s$ are unknown while $\text{vec}\left( \widehat{\boldsymbol{\Sigma}}_{\mathbf{Y}_s} \right)$ is known from the observed measurements.
By this transformation to the covariance-domain, one segment is now represented by s single measurement model with $\widetilde{M}$ ''measurements''. 

It has been shown that this transformed model allows for identification of $k \leq \widetilde{M}$ active sources \cite{Pal2015}, which is a much weaker sparsity constraint than the original sparsity constraint $k \leq M$. 
The purpose of the Cov-DL algorithm is to leverage this transformed model to find the dictionary $\mathbf{A}_s$ from $\mathbf{D}_s$ still allowing for $k \leq \widetilde{M}$ active sources to be recovered. 
That is the number of active sources are allowed to exceed the number of sensors as intended.

\section{Recovery of the Mixing Matrix}
The goal is now to learn first $\textbf{D}_s$ and then the associated mixing matrix $\textbf{A}_s$. 
Two methods are considered relying on the relation between $M$ and $N$. 
For now the noise vector is ignored.


\subsection{Under-determined System}\label{sec:cov1}
When $\widetilde{M} < N$ the transformed model \eqref{eq:cov1} makes an under-determined system. 
This is similar to the original MMV model \eqref{eq:MMV_model} being under-determined when $M < N$. 
Thus, from the theory of compressive sensing, it is again possible to solve the under-determined system if a certain sparsity is withheld, namely $\boldsymbol{\delta}_s$ being $\widetilde{M}$-sparse.
Assuming the sufficient sparsity on $\boldsymbol{\delta}_s$ is withheld it is possible to learn the dictionary matrix of the covariance domain $\mathbf{D}_s$. 
This can be done by traditional dictionary learning methods applied to the measurements represented in the covariance-domain $\text{vec}\left(\widehat{\boldsymbol{\Sigma}}_{\mathbf{Y}_s}\right)$ for all segments $s$.

\subsubsection{Dictionary Learning}\label{sec:dictionarylearning}
As mentioned, within the theory of compressive sensing the matrix $\mathbf{A}$ is referred to as a dictionary matrix. 
When the dictionary matrix is not known a priori it is essential how to choose the dictionary matrix in order to achieve the best recovery, of a sparse vector $\mathbf{x}$ from the observed measurements $\mathbf{y}$. 
This is clarified from the proof of theorem \ref{th:CS_A} in appendix \ref{app_sec:CS}. 
One choice is a pre-constructed dictionary. 
In many cases the use of a pre-constructed dictionary results in simple and fast algorithms for reconstruction of $\mathbf{x}$ \cite{Elad_book}. 
However, a pre-constructed dictionary is typically fitted to a specific kind of data. 
For instance the discrete Fourier transform or the discrete wavelet transform are used especially for sparse representation of images \cite{Elad_book}. 
Hence the results of using such dictionaries depend on how well they fit the data of interest, which is establishing a certain limitation. 

The alternative option is to consider an adaptive dictionary based on a set of training data that resembles the data of interest. 
For this purpose learning methods are considered to empirically construct a dictionary. 
There exist several dictionary learning algorithms. One is the K-SVD algorithm which was presented in 2006 by Elad et al. and found to outperform pre-constructed dictionaries, when computational cost is of secondary interest \cite{Elad2006}. 
The concept of the K-SVD algorithm is introduced here, and the more detailed algorithm is to be found in appendix \ref{app_sec:K-SVD_alg}. 

Consider, from the general MMV model \eqref{eq:MMV_model}, the measurement matrix $\mathbf{Y} \in \mathbb{R}^{M \times L}$ consisting of measurement vectors $\lbrace \mathbf{y}_j \rbrace_{j=1}^L$. Let the set of measurement vectors make a set of $L$ training examples each forming a linear system
\begin{align*}
\mathbf{y}_j = \mathbf{A} \mathbf{x}_j.
\end{align*}
From the linear system one can learn a suitable dictionary $\hat{\mathbf{A}}$, and the sparse representation of the source matrix $\hat{\mathbf{X}} \in \mathbb{R}^N$ with the source vectors $\lbrace \hat{\mathbf{x}}_j \rbrace_{j=1}^L$.
For a known sparsity constraint $k$ dictionary learning can be defined by the following optimization problem. 
\begin{align}\label{eq:SVD1}
\min_{\mathbf{A}, \mathbf{X}} \sum_{j=1}^{L} \Vert \mathbf{y}_j - \mathbf{A} \mathbf{x}_j \Vert_2^2 \quad \text{subject to} \quad \Vert \mathbf{x}_j \Vert_0 \leq k, \ 1 \leq j \leq L,
\end{align}
where both $\mathbf{A}$ and $\mathbf{x}_j$ are quantities to be determined.
Learning the dictionary by the K-SVD algorithm consists of joint solving of the optimization problem with respect to $\mathbf{A}$ and $\mathbf{X}$. 
An initial $\mathbf{A}_0 = [\mathbf{a}_1, \dots, \mathbf{a}_N]$ is chosen and the corresponding $\mathbf{X}_0 = [\mathbf{x}_1, \dots, \mathbf{x}_L]$ is determined, where $\mathbf{x}_j = [x_{1j}, \dots, x_{Nj}]^T$. Then, for each iteration an update rule is applied to every column of $\mathbf{A}_0$. That is updating first $\mathbf{a}_j$ for $j = 1, \dots, N$ and then the corresponding row $\mathbf{x}_{i\cdot}$ where $i = j$. 
More details on the K-SVD algorithm are found in appendix \ref{app_sec:K-SVD_alg}. 
The uniqueness of the dictionary $\hat{\mathbf{A}}$ depends on the recovery sparsity condition. As clarified earlier in section \ref{sec:sol_met} the recovery of a unique solution $\mathbf{X}^\ast$ is only possible if $k < M$ \cite{phd2015}.
%The dictionary learning algorithm K-SVD is a generalisation of the well known K-means clustering also referred to as vector quantization. In K-means clustering a set of $K$ vectors is learned referred to as mean vectors. Each signal sample is then represented by its nearest mean vector. That corresponds to the case with sparsity constraint $k = 1$ and the representation reduced to a binary scalar $x = \lbrace 1, 0 \rbrace$. Further instead of computing the mean of $K$ subsets the K-SVD algorithm computes the SVD factorisation of the $K$ different sub-matrices that correspond to the $K$ columns of $\textbf{A}$.


\subsubsection{Application of Dictionary Learning}
By the establishment of a dictionary learning algorithm, the transformed mixing matrix $\mathbf{D}_s$ from \eqref{eq:cov1} can be learned. 
Remember that \eqref{eq:cov1} is a single vector model, thus in order to make training samples for learning $\mathbf{D}_s$ a further segmentation is needed.
This is segmentation of $\mathbf{Y}_s$ indexed by $s'$. 
For convenience segment index $s$ will be omitted through out this chapter, as the same theory applies to all segments $s$.
Hence, $\mathbf{Y}_{s'}$ referrers to one segment within the outer segment of measurements $\mathbf{Y}_s$. 
   
The transformed and vectorized measurements $\text{vec} \left( \widehat{\boldsymbol{\Sigma}}_{\mathbf{Y}_{s'}} \right), \forall s'$ now makes the training dataset for learning $\mathbf{D}$. 
As such each segment $s'$ provides one training sample. 
Thus, the number of available training samples, denoted $L_{s'}$, depends on the chosen length of the segments. In practice this will vary with respect to the total amount of available data. 

K-SVD is applied to the transformed model \eqref{eq:cov1} and $\hat{\mathbf{D}}$ is found. Then it is possible to estimate the mixing matrix $\mathbf{A}$ that generated $\mathbf{D}$ through the known relation 
\begin{align*}
\mathbf{d}_j = \text{vec}(\mathbf{a}_j \mathbf{a}_j^T).
\end{align*}
For each column $\mathbf{d}_j$ for $j = 1, \dots, N$ the following optimization problem is solved with respect to the corresponding column $\mathbf{a}_j$ of the mixing matrix.
\begin{align*}
\min_{\mathbf{a}_j} \| \mathbf{d}_j -\text{vec}\left(\mathbf{a}_j \mathbf{a}_j^T\right) \|_2^2, 
\end{align*}
equivalent to 
\begin{align}
\min_{\mathbf{a}_j} \| \text{vec}^{-1}(\mathbf{d}_j) - \mathbf{a}_j \mathbf{a}_j^T\|_2^2. \label{eq:opt_DL1}
\end{align}
From \cite{Balkan2015} the global minimizer to \eqref{eq:opt_DL1} is given as $\mathbf{a}^{\ast}_j=\sqrt{\lambda_j} \mathbf{b}_j$, without further details or a source.  
Here $\lambda_j$ is the largest eigenvalue of $\text{vec}^{-1}(\mathbf{d}_j)$, where
\begin{align*}
\text{vec}^{-1}(\mathbf{d}_j) = 
\begin{bmatrix}
d_{11} & d_{12} & \cdots & d_{1N} \\
d_{21} & d_{22} & \cdots & d_{2N} \\
\vdots & \vdots & \ddots & \vdots \\
d_{N1} & d_{N2} & \cdots & d_{NN}
\end{bmatrix}, \quad j =1, \dots, N
\end{align*}
and $\mathbf{b}_j$ is the corresponding eigenvector.

From this result each column of the mixing matrix $\mathbf{A}$ can be estimated. 
Hence, it is possible to determine the mixing matrix in the case where the measurements transformed into the covariance-domain makes an under-determined system.
Provided however that the necessary sparsity constraint of $\boldsymbol{\delta}$ being $\widetilde{M}$-sparse is withheld. 
Remember $\widetilde{M} := \frac{M(M+1)}{2}$ thus $M < k$ is allowed and the original sparsity constraint, $\mathbf{X}$ being $M$-sparse, is relaxed. 






\input{sections/ch_cov-DL/Cov-DL2.tex}

\section{Pseudo Code of the Cov-DL Algorithm}\label{seg:alg_cov}
\begin{algorithm}[H]
\caption{Cov-DL}
\begin{algorithmic}[1]
           \Procedure{Cov-DL}{$\textbf{Y}_s$}    
			\For{$s' \gets 1,\hdots, L_{s'}$}			
				\State$\textbf{y}_{\text{cov}_{s'}} = \text{vec}\left( \widehat{\boldsymbol{\Sigma}}_{\mathbf{Y}_{s'}} \right)$	
			\EndFor			
			\State$\textbf{Y}_{\text{cov}} = \{\textbf{y}_{\text{cov}_{s'}}\}_{s'=1}^{L_{s'}}$
			\State
			\If{$N \geq \widetilde{M}$}		
			\Procedure{K-SVD}{$\textbf{Y}_{\text{cov}}$}
			\State$\text{returns} \ \textbf{D} \in \mathbb{R}^{\widetilde{M}\times N}$
			\EndProcedure
			\For{$j \gets 1, \hdots, N$}
			\State$\textbf{T} = \text{vec}^{-1}(\textbf{d}_j)$            
			\State$\lambda_j\gets \max\{\text{eigenvalue}(\textbf{T})\}$
			\State$\textbf{b}_j \gets \ \text{eigenvector}(\lambda_j)$
			\State$\textbf{a}_j \gets \sqrt{\lambda_j}\textbf{b}_j$
			\EndFor
			\State$\textbf{A} = \{\textbf{a}_j\}_{j=1}^N$
			\EndIf
			\State
			\If{$N < \widetilde{M}$}
				\Procedure{PCA}{$\textbf{Y}_{\text{cov}}$}
				\State$\text{returns} \ \textbf{U}\in \mathbb{R}^{\widetilde{M}\times N}$
				\EndProcedure
				\Procedure{Min. $\textbf{A}$ in }{$\Vert  \textbf{D}(\textbf{D}^T\textbf{D})^{-1}\textbf{D}^T - \textbf{U}(\textbf{U}^T\textbf{U})^{-1}\textbf{U}^T \Vert_{F}^{2}$}
				\State$\text{returns}\ \textbf{A}= \{\textbf{a}_j\}_{j=1}^{N}$
				\EndProcedure
			\EndIf
           \EndProcedure
        \end{algorithmic} 
        \label{alg:Cov1}
\end{algorithm}

\section{Remarks}
Through this chapter the theoretical aspects of the Cov-DL method proposed by \cite{Balkan2015} have been investigated in order to create algorithm \ref{alg:Cov1} from which the implementation of Cov-DL will be based. Furthermore the following remarks are considered with respect to the implementation.   

The length of each time segment $s$ has to be defined with respect to the assumption of the signals being stationary. However, it can not be assured that the assumption is withhold for every segment and this will introduce a source of error. 
This must be taken into account in the preprocessing part for the implementation of Cov-DL when the EEG measurements are divided into segments.

For each segment a further segmentation is conducted into segments $s'$, each serving as one sample in the covariance-domain. Here the number of samples $L_{s'}$, depending on the chosen length, is most likely to influence the estimated dictionary. This is assuming that more training data will provide better results. Here a certain trade off may be considered. Longer segments $s'$ lead to better sample covariance representation but also a fewer number of training samples. Opposite, too short segments $s'$ might compromise the sample covariance-domain representation, thus the number of training sample will increase but the training samples might not be as representative.                 
This trade off must be taken into account during the implementation of Cov-DL.  
Furthermore, overlapping segments might be an option for potential improvement of the Cov-DL method.