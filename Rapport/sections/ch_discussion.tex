\chapter{Discussion}
%In this chapter the methods used throughout the thesis is discussed based achieved results.  
%......
%Through this study the main algorithm have been proposed with the purpose of recovering the sources signals from the inverse EEG problem. 
%The main algorithm is based upon two state of the art methods proposed by O. Balkan \cite{Cov}\cite{M-SBL}, intended for recovery of respectively the mixing matrix and the source matrix.  
%Thus, the primary purpose of the study the was understand and implement the two methods into one application to be used on EEG measurements - the same purpose for which O. Balkan intended the methods. 

%%%% short summary of results
%Based on the articles the applied theory was studied and an implementation was conducted. 
%From the verification of the implementation is found that the COV-DL method is not successful. 
%By not having a successful estimate of the mixing matrix the recovery of the source signals is compromised. 
%The M-SBL method was proven successful on synthetic data when the true mixing matrix was given.

%% I believe that the results is more or less being mentioned under this so maybe this section should be an over all conclusion?   

% 1. A restatement of your research question followed by a statement about your findings.
% A short summary of our obtain results
The purpose of this thesis was to investigate whether it was possible, based on two state of the art methods covariance-domain dictionary learning (Cov-DL) and multiple sparse Bayesian learning (M-SBL) proposed by O. Balkan \cite{Cov}\cite{M-SBL}, to reproduce the methods and results for localization and identification of sources from scalp EEG measurements. 
It was found that this task was not easy and the combination of Cov-DL and M-SBL, the main algorithm, did not successfully solve the inverse EEG problem of recovering a mixing matrix and a source matrix from the EEG measurements. 
From the verification of the implementation of Cov-DL it was found that it does not successfully recover an estimate for the mixing matrix $\mathbf{A}$ for the simulated data sets. 
By not having a successful estimate of the mixing matrix the recovery of the sources is compromised when using the M-SBL.
However, with the real mixing matrix $\mathbf{A}_{\text{true}}$ the M-SBL recovers an estimate for the source matrix $\mathbf{X}$ almost identical to the real source matrix for the simulated data sets.
Instead Cov-DL was replaced by a fixed mixing matrix which was then used in the main algorithm.

% 2. Relate your findings to the issues in the introduction -- similiarities, difference, commens and trends
%%%% discussion of the articles of which the project is based.
In chapter \ref{ch:implementation} it was concluded that the implementation of the Cov-DL method is not successful.
The issue was located in the definition of the optimization problem determining the columns of the mixing matrix $\mathbf{A}$.  
This questions the reproducibility of the scientific article \cite{Cov} which has been used as the main source. 
The article \cite{cov} did not provide any code or implementation specifications. 
Likewise it was not possible to recreate or access the exact data which was use to provide the results presented in the article. 
Thus the intention was never to recreate the exact results from the article but rather to prove the conclusion that the method manage to provide results of a certain success rate.    
One could argue that testing the implementation on the same data would lead to a different conclusion. 
However, this was sought approached by the simulated data which was created with inspiration from the article.
%	description of the procedure in the article
%	X are simulated by AR process to be super gaussian, A and Y 	is genereated by the tools in EEGLAB i matlab, sample rate 		100 hz ans segments of 2 seconds. They consider three cases 	M=N=k=32, N/M = 2 N=k=64(COV-DL2) and M=8 N=40 k=vary such 		that N/M=5 and k>m,(COV-DL1). 
%	here they base the result on the rate of recovered sources. 	comparing with two ICA methods. however i am not sure how 		the compute X from from COV-DL A they find. "The accuracy 		of the result is measured as the ratio of the number of 		scalp maps(one scalp map corresponds to one source thus one colunm in A) that are recovered
%	(having correlation higher than 0.99 with true scalp maps)"  

From \ref{ch:implementation} it was concluded that the M-SBL succeeded to estimate the source matrix when applied to the simulated data sets and given the true mixing matrix. 
%The method was not tested further on the recovered mixing matrix from Cov-DL.
However, an alternative to the estimate of the mixing matrix was sought. 
Through empirical test, which is discussed later, a fixed mixing matrix was chosen to replace the estimate from COV-DL. 
With the fixed mixing matrix the M-SBL manage to estimate the source matrix but with a significant higher error compare to the use of the true mixing matrix.
Regarding the reproducibilite of the article \cite{M-SBL} the results indicates that the provided information about M-SBL have been sufficient.
However, this article did, as \cite{Cov}, not provide any code or data disabling the possibility of recreating the exact results. In \cite{M-SBL} tests were conducted on random simulations of $\textbf{A}$ and $\textbf{X}$ with various noise level added.   
Thus, due to no counter arguments, the tests of M-SBL in this thesis were conducted on the simulated data sets allready created for the tests of COV-DL, with inspiration from \cite{Cov}.

% 3. Write about unexpected findings
% 4. state you major conclusion and present theortical and pratical implications of study
%%%% verificerings methode 
To test the performance of the main algorithm the first test was conducted on simulated stochastic data resembling real EEG measurements. 
Inspired by \cite{cov} the sources was synthesised by independent auto-regressive processes. 
The true mixing matrix however was simply generated by a Gaussian distribution. 
This choice was based on a lack of information to point in different directions. 
Instead of the mixing matrix being chosen as a stochastic matrix a deterministic matrix could have been chosen instead. 
The choice of the stochastic true mixing matrix of the simulated data sets could affect the results when testing the fixed alternatives for the mixing matrix estimated by COV-DL. 
From the test, cf. section \ref{seg:??} it was found that a Gaussian fixed mixing matrix, generated with same specifications as the true mixing matrix, did not lead to the best recovery of the sources matrix.
which went against the natural expectation -- the better estimate of $\textbf{A}$ the better estimate of $\textbf{X}$. Here a fixed mixing matrix with Gaussian distribution of higher variance provided the best result. It is here one could argue that the stochastic true mixing matrix had an influence to the results.      
%Therefore, other true mixing matrix was not tested as it seem to not have any influence on achieving a better performing when finding the sources from M-SBL. 
Furthermore, the chose of error measurement might also be reflected in the results. Not being sufficient towards the purpose. 

The common choice of error measurement throughout the thesis was the mean-square error (MSE). The MSE measures the performance of an estimate with respect to the true value, by comparing each element and summing the squared error. Hence the MSE is providing a measure of how far the estimate is from the true value. This was considered an sufficient error measurement for evaluation of COV-DL and M-SBL.
However, one challenge when using MSE is to set a tolerance defining the when an estimate is considered successful. It could be argued that set tolerance should vary with respect to the data of interest.
Though, it was not chosen to evaluate to the algorithms with respect to success rate, by which the above challenge was avoided, replaced by more soft evaluation based on the MSE.  
A different choice of error measurement could have been the use of correlation among variables of the estimate the ground truth. With respect to the comparison of the main algorithm to ICA as the ground truth, to be discussed next, using the correlations as the evaluation might have overcome the scaling issue with respect to ICA.    

For the test of the main algorithm on real EEG measurements the choice of evaluating the performance by considering the ICA solution as the ground truth was crucial.   
First of all the foundation for the evaluation was not ideal.  
It is an issue that the performance of the main algorithm on the simulated data was not as good expected, presumably due to the estimate of $\textbf{A}$ been replaced by a fixed matrix. 
Thus it is not reasonable to trust the results when the algorithm is applied to data for which the true results are not known at all. 
However, it can be argued that comparing the obtained result to the best known solution, in this case provided by ICA, a small error will indicated an acceptable performance from the main algorithm. 
The arguments accounting for the use of ICA was discussed in section \ref{seg:??}. However, an unreliability will be present as the ICA algorithm is limited to $M = N$. The true $N$ is unknown, thus ICA do not guaranteed to find all the active sources.
Furthermore, the comparison of the main algorithm to ICA was found to be compromised. The localisation and phase of the active sources is not necessary the same for the two estimates, which distorts the MSE between the two matrices to an unknown degree.
In despite of this issue the comparison was still performed by MSE, which suggest a potential to improve the found performance. 

Due to the possible unreliability of the performance evaluation with respect to ICA, an alternative test was considered. 
However, from the alpha wave analysis no new conclusion was made. The expected behaviour was not observed with respect to an increased amount of alpha frequency for the test subject having closed eyes. 
The behaviour was more or less fluctuating over time. However exceptions from the expected behaviour was also found for the raw measurement. This indicates that different approaches, with respect to measuring the power within the alpha band, should be considered.
The advantage of this test approach in general is that one see past the challenge of recovering the exact source signal, but rather consider the practical usage of the source separation. For instance, when considering the usage of the source signals within the a hearing aid, cf. section \ref{seg:??}, the amount of active source signals might be more significant that an exact recovery.             

% 5. Discuss your implications of the study for future research
The last issue addressed in this thesis was estimate the number of active sources $k$ relative to the maximum number of sources $N$. Through out the implementation of the main algorithm in chapter \ref{ch:implementation} it was argued that setting $k = N$ would not compromise the results. 
In fact perhaps lead to better estimates as fewer sources must be recovered and therefore reduced the errors.
This is supported by \cite{cov} where the same assumption is used, as no sparsity constraint is considered for Cov-DL. 
Likewise for M-SBL, providing $k$ would only reduced possible errors within localisation of the sources.
However, by setting $k = N$ one must assume that $k$ sources are active, thus no less that $k$ sources are estimated. 
Hence, to justify this definition of $k$ one must have a qualified guess with respect to the true $k$. 
However, this is the issue which is addressed in the problem statement, as this is not possible in practice. 

To address this exact issue an investigation with respect to estimation of $k$ was conducted, cf. chapter \ref{ch:estimation_k}. 
Due to interesting empirical observation it was chosen to analyses the source signal resulting from the main algorithm when $k = N = \widetilde{M}$. 
That is estimating the maximum number of active sources under the hypothesis that the false estimates where to be identified among the true estimates. By false estimate there is referred to a non-zero estimate of a zero row. 
From visual observation of results from the simulated stochastic data set a potential was seen as the false estimates appears as copies of the true estimates. 
Identification of these false estimates was sought by use of the replicates of the true estimates.
However, this was not found successful in the desired cases where $M<k$. 
Here the false estimates manages to diverge more from the true estimates. 
Furthermore, the identification method was found to have trouble separating the true estimate from the corresponding replicates.
Alternative methods could have been considered with respect to estimation of $k$. 
One obvious approach is to consider the estimate of $k$ which could be provided from M-SBL, if a $k$ was not given as a input to the algorithm. 
However, one could argue that an essential limitation is still present as the mixing matrix is needed for the executing of M-SBL and the number of active sources $k$ is a needed input to the Cov-DL. Furthermore the success rate of such estimation of $k$ was not provided in the article  \cite{M-SBL} of which the article was based. As the performance presented in the article was obtained by providing $k$ to the algorithm, cf. \ref{seg:??}.   

Throughout this chapter the choices which ware found essential toward the obtained conclusions have been discussed and alternative choices have been considered.  
None of these alternatives are sought investigated in the thesis and but they will serve essential point to be considered if further work on the main algorithm where to be conducted.


      
        


       
 