\chapter{Discussion}
%In this chapter the methods used throughout the thesis is discussed based achieved results.  
%......
%Through this study the main algorithm have been proposed with the purpose of recovering the sources signals from the inverse EEG problem. 
%The main algorithm is based upon two state of the art methods proposed by O. Balkan \cite{Cov}\cite{M-SBL}, intended for recovery of respectively the mixing matrix and the source matrix.  
%Thus, the primary purpose of the study the was understand and implement the two methods into one application to be used on EEG measurements - the same purpose for which O. Balkan intended the methods. 

%%%% short summary of results
%Based on the articles the applied theory was studied and an implementation was conducted. 
%From the verification of the implementation is found that the COV-DL method is not successful. 
%By not having a successful estimate of the mixing matrix the recovery of the source signals is compromised. 
%The M-SBL method was proven successful on synthetic data when the true mixing matrix was given.

%% I believe that the results is more or less being mentioned under this so maybe this section should be an over all conclusion?   

% 1. A restatement of your research question followed by a statement about your findings.
% A short summary of our obtain results
The purpose of this thesis was to investigate whether it was possible, based on two state of the art methods covariance-domain dictionary learning (Cov-DL) and multiple sparse Bayesian learning (M-SBL) proposed by O. Balkan \cite{Cov}\cite{M-SBL}, to reproduce the methods and results for localization and identification of sources from scalp EEG measurements. 
It was found that this task was not easy and the combination of Cov-DL and M-SBL, the main algorithm, did not successfully solve the inverse EEG problem of recovering a mixing matrix and a source matrix from the EEG measurements. 
From the verification of the implementation of Cov-DL it was found that it does not successfully recover an estimate for the mixing matrix $\mathbf{A}$ for the simulated data sets. 
By not having a successful estimate of the mixing matrix the recovery of the sources is compromised when using the M-SBL.
However, with the real mixing matrix $\mathbf{A}_{\text{true}}$ the M-SBL recovers an estimate for the source matrix $\mathbf{X}$ almost identical to the real source matrix for the simulated data sets.
Instead Cov-DL was replaced by a fixed mixing matrix which was then used in the main algorithm.

% 2. Relate your findings to the issues in the introduction -- similiarities, difference, commens and trends
%%%% discussion of the articles of which the project is based.
In chapter \ref{ch:implementation} it was concluded that the implementation of the Cov-DL method is not successful.
The issue was located in the definition of the optimization problem determining the columns of the mixing matrix $\mathbf{A}$.  
This questions the reproducibility of the scientific article \cite{Cov} which has been used as the main source. 
The article \cite{cov} did not provide any code or implementation specifications. 
Likewise it was not possible to recreate or access the exact data which was use to provide the results presented in the article. 
Thus the intention was never to recreate the exact results from the article but rather to prove the conclusion that the method manage to provide results of a certain success rate.    
One could argue that testing the implementation on the same data would lead to a different conclusion. 
However, this was sought approached by the simulated data which was created with inspiration from the article.
%	description of the procedure in the article
%	X are simulated by AR process to be super gaussian, A and Y 	is genereated by the tools in EEGLAB i matlab, sample rate 		100 hz ans segments of 2 seconds. They consider three cases 	M=N=k=32, N/M = 2 N=k=64(COV-DL2) and M=8 N=40 k=vary such 		that N/M=5 and k>m,(COV-DL1). 
%	here they base the result on the rate of recovered sources. 	comparing with two ICA methods. however i am not sure how 		the compute X from from COV-DL A they find. "The accuracy 		of the result is measured as the ratio of the number of 		scalp maps(one scalp map corresponds to one source thus one colunm in A) that are recovered
%	(having correlation higher than 0.99 with true scalp maps)"  

From \ref{ch:implementation} it was concluded that the M-SBL finds good estimates for the source matrix when using the simulated data sets and the true mixing matrix. 
The method was not tested further on the recovered mixing matrix from Cov-DL.
However, an alternative was sought to the estimate of the mixing matrix and the M-SBL was then tested with a fixed mixing matrix drawn from a normal distribution. 
With the fixed mixing matrix the M-SBL still recovers a source matrix but with a higher difference from the true mixing matrix.
That means the the reproducibilite of the artcile \cite{M-SBL} providing the information about M-SBL indeed succeed.
However, as this article did as \cite{Cov} not provided any code or data as well and the tests was therefore conducted on the simulated data sets created with inspiration from \cite{Cov}.
As the two articles \cite{M-SBL} and \cite{Cov} have the same writer it was concluded that the same data set was used in both papers and therefore the simulated data sets created purpose to test the main algorithm can be used on both methods.


% 3. Write about unexpected findings

% 4. state you major conclusion and present theortical and pratical implications of study
%%%% verificerings methode 
To test the performance of the main algorithm the first test was conducted on simulated stochastic data which was suppose to resemble real EEG measurements. 
Inspired by the article \cite{cov} the sources was synthesised by independent auto-regressive processes. 
The true mixing matrix however was simply generated by a Gaussian distribution. 
This choice was based on a lack of information to point in different directions. 
Instead of the mixing matrix being chosen as a stochastic matrix a deterministic matrix could have been chosen instead. 
The choice of the true mixing matrix of the simulated data sets could affect the results when testing fixed alternative for  mixing matrix instead of Cov-DL estimate. 
However, from the test it was found that the Gaussian fixed mixing matrix did not lead to the best recovering even with the true mixing matrix being Gaussian. Therefore, other true mixing matrix was not tested as it seem to not have any influence on achieving a better performing when finding the sources from M-SBL. 

%This could have made a chance with respect to the results regarding the choice of fixed mixing matrix to replace the missing estimate... here fixed alternative options was presented and tested on several simulated data sets... not sure what chance deterministic could do.
%- perhaps discuss the mse measure here instead. 
From the test of fixed mixing matrix it was expected that the fixed matrix with generate from the same Gaussian distribution as the true mixing matrix would lead to the best recovery out of all fixed matrices. 
However, this was not the case one could assume perhaps the error measurement could not calculate the correct error. 
The choice for a measurement of error was the mean-square error (MSE) which measures the performance of estimator regarding the true estimate. 
As the Cov-DL and M-SBL gives an approximately estimates this was concluded to be a good choice to measure the performance of the main algorithm. 

For the test of the main algorithm on real EEG measurements the choice of performance measure has been crucial. 
First of all the starting point for the test is not ideal \todo{Forstår ikke denne sætning - L}. 
It is an issue that the performance on the simulated data was not as expected, due to one estimate been replaced by an fixed matrix. 
Thus it is not reasonable to trust the results when the algorithm is applied to data for which the true results are not known at all. 
However, it can be argued that comparing the obtained result to the best known solution, in this case ICA, similar results will indicated an acceptable performance from the main algorithm. 
Despite the performance of the main algorithm on simulated data there is always an unreliability when comparing to the results by ICA. 
As the ICA algorithm is limited to $M = N$, and the true $N$ is unknown, ICA do not guaranteed to find all the active sources.
 
Furthermore, the comparison of the main algorithm to ICA was found to not be ideal due the localisation and phase of the active sources is not necessary the same for the two estimates, which to compromises the MSE between the two matrices.
In despite of this issue the comparison was still performed by MSE, which suggest that the result might be better than it appears by the MSE. 

Due to the possible reliability of the performance evaluation with respect to ICA, an alternative test was considered. 
With the alpha wave analysis....\todo{Tilføj her alpha waves -L} 

% 5. Discuss your implications of the study for future research
The last issue of thesis was to make an estimation of active sources $k$ relative to the maximum number of sources $N$. Through out the implementation of the main algorithm in chapter \ref{ch:implementation} it was argued that setting $k = N$ would not compromise the results. 
In fact perhaps lead to better estimates as fewer sources must be recovered and therefore reduced the errors.
This is supported by \cite{cov} where the same assumption is used, as no sparsity constraint is considered for Cov-DL. 
Likewise for M-SBL, providing $k$ would only reduced possible errors within localisation of the sources.
However, by setting $k = N$ one must assume that $k$ sources are active, thus no less that $k$ sources are estimated. 
Hence, to justify this definition of $k$ one must have a qualified guess with respect to the true $k$. 
However, this is the issue which is addressed in the problem statement, as this is not possible in practice. 
%During the test of the main algorithm on real EEG measurements, $k = N$ was provided by the number of active source estimated by ICA, indicating that the main algorithm is not usable i practice. 
To address this exact issue an investigation with respect to estimation of $k$ was conducted in chapter \ref{ch:estimation_k}. 
Due to interesting empirical observation it was chosen to analyses the source signal resulting from the main algorithm when $k = N = \widetilde{M}$. 
That is estimating the maximum number of active sources under the hypothesis that the ''false'' estimates where to be identified among the true estimates. 
From visual observation of results from the simulated stochastic dataset a potential was seen as the ''false'' estimates appears as copies of the true estimates. 
Identification of these ''false'' estimates was sought by use of the replicates of the true estimates.
However, this was not found successful in the desired cases where $M<k$. 
Here the ''false'' estimates manages to diverge more from the true estimates. 
Furthermore, the identification method was found to have trouble separating the true estimate from the corresponding replicates.
Alternative methods could have been considered with respect to estimation of $k$. 
One obvious approach is to consider the estimate of $k$ which could be provided from M-SBL, if a $k$ was not given as a input to the algorithm. 
However, one could argue that an essential limitation is still present as the mixing matrix is needed for the executing of M-SBL and the number of active sources $k$ is a needed input to the Cov-DL. 
However, none of these alternative are sought investigated in the thesis and would be a step to think about for further work on the implementation of the main algorithm.
%- Though this would also be a similar issue within the method we invastigated right? 

\todo{Måske en lille kort opsummering eller en eller anden form for afslutning - L}

      
        


       
 