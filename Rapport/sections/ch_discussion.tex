\chapter{Discussion}
In this chapter the methods used throughout the thesis is discussed based achieved results.  
...... 


Through this study the main algorithm have been proposed with the purpose of recovering the sources signals from the inverse EEG problem. 
The main algorithm is based upon two state of the art methods proposed by O. Balkan \cite{Cov}\cite{M-SBL},
intended for recovery of respectively the mixing matrix and the source signals.  
Thus, the primary purpose of the study the was understand and implement the two methods into one application to be used on EEG measurements - the same purpose for which O. Balkan intended the methods. 

%%%% short summary of results
Based on the articles the applied theory was studied and an implementation was conducted. From the verification of the implementation is found that the COV-DL method is not successful. By not having a successful estimate of the mixing matrix the recovery of the source signals is compromised. The M-SBL method was proven successful on synthetic data when the true mixing matrix was given
%% I believe that the results is more or less being mentioned under this so maybe this section should be an over all conclusion?   

%%%% discussion of the articles of which the project is based.
In chapter \ref{ch:implementation} it is concluded that the implementation of the COV-DL method is not successful.
The issue was located to the definition of the optimization problem determining the columns of the mixing matrix.  
This questions the reproducibility of the scientific article which has been used as the main source. 
The article \cite{cov} did not provide any code or implementation specifications. 
Likewise it was not possible to recreate or access the exact data which was use to provide the results presented in the article. 
Thus the intention was never to recreate the exact results from the article but rather to prove the conclusion that the method manage to provide results of a certain success rate.    
one could argue that that testing the implementation on the same data. however this was sought approached by the synthetic data which was created with inspiration from the article.
%	description of the procedure in the article
%	X are simulated by AR process to be super gaussian, A and Y 	is genereated by the tools in EEGLAB i matlab, sample rate 		100 hz ans segments of 2 seconds. They consider three cases 	M=N=k=32, N/M = 2 N=k=64(COV-DL2) and M=8 N=40 k=vary such 		that N/M=5 and k>m,(COV-DL1). 
%	here they base the result on the rate of recovered sources. 	comparing with two ICA methods. however i am not sure how 		the compute X from from COV-DL A they find. "The accuracy 		of the result is measured as the ratio of the number of 		scalp maps(one scalp map corresponds to one source thus one colunm in A) that are recovered
%	(having correlation higher than 0.99 with true scalp maps)"  

%%%% verificerings methode 
To test the performance of the main algorithm is was first tested on synthetic data which was suppose to resemple real EEG measurements, the synthetic data was. inspired by the article \cite{cov} the sources signal was synthesised by independent AR processed. The true mixing matrix however was simply generated by a normal distribution. This choice was based on a lack of information to point in different directions. instead of a stochastic matrix a deterministic one could have been chosen this might. This could have made a chance with respect to the results regarding the choice of fixed mixing matrix to replace the missing estimate... here fixed alternative options was presented and tested on several simulated data sets... not sure what chance deterministic could do.
- perhaps discuss the mse measure here instead. 

when the main algorithm where to be tested on real EEG data the choice of performance measure has been crucial. First of all the starting point for the test is not ideal. It is an issue that the performance on the synthetic data was not as expected, due to one estimate been replaced by an fixed matrix. Thus it is not reasonable to trust the results when the algorithm is applied to data for which the results are not known at all. However it can be argued that comparing the obtained result to the best known solution, in this case ICA, similar result will indicated an acceptable performance from the main algorithm. 
Despite the performance of the main algorithm on synthetic data there is always an unreliability the when comparing to the result by ICA. As the ICA algorithm used in the study is limited to N=M, and the true N is unknown, ICA is not guaranteed find all the active source signals.
 
Furthermore the comparison of the main algorithm to ICA was found to not be ideal due appendix \ref{app:ica_test} the localisation and phase of the active sources is not necessary the same for the two estimates, which to compromises the mse between the two matrices. In despite of this issue the comparison was still performed by mse, which suggest that the result might be better than it appears by the MSE. 

Due to the possible reliability of the performance evaluation with respect to ICA, an alternative test was considered...  


The secondary issue of the study is the estimation of active sources $k$ relative to the maximum number of sources $N$. through out the development of the main algorithm is was argued to that letting $k=N$ would not compromise the results. This is supported by \cite{cov} where the same assumption is used, as no sparsity constraint is considered for COV-DL. Likewise for M-SBL, providing $k$ would only reduced possible errors within localisation of the sources.
However by letting $k=N$ one assume that $k$ sources is active, thus no less that $k$ sources are estimated. Hence, to justify this definition of k one must have a qualified guess with respect to $k$. However this is the issue which is addressed in the problem statement, as this is not possible in practise. during the test of the main algorithm, $k=N$ was provided by the number of active source estimated by ICA, indicating that the main algorithm is not usable i practice. 

To addressee this exact issue an investigation with respect to estimation of k was conducted in chapter \ref{ch:estimation_k}. 
Due to interesting empirical observation it was chosen to analyses the source signal resulting from the main algorithm when $k=N=\widetilde{M}$. That is estimating the maximum number of active sources under the hypothesis that the ''false'' estimates where to be identified among the true estimates. 
From visual observation of results from synthetic dataset a potential was seen as the ''false'' estimates appears as copies of the true estimates. Identification was sought by use of the however this was not found successful in the desired cases where $M<k$. here the ''false'' estimates manages to diverge  more from the true estimates. Furthermore the identification method was found to have trouble separating the true estimate from the corresponding copies.
Alternative methods could have been considered with respect to estimation of k. One obvious approach is to consider the estimate of $k$ which is provided by M-SBL if k is not provided. However one could argue that an essential limitation is still present as the mixing matrix is needed to perform M-SBL and $k$ need to be provided to COV-DL if it was to be used prior to M-SBL. - Though this would also be a similar issue within the method we invastigated right? 

      
        


       
 