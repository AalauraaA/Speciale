\chapter{Test on EEG measurements}\label{ch:eeg_test}
Through this chapter a modification of the main algorithm will be described such that the main algorithm can handled real EEG measurements. 

For verification of the results the MSE values achieved from the EEG measurements will be compared to MSE values achieved from the use of ICA, cf. appendix. This would include some preprocessing of the used EEG measurements.

At last, from the observed results of the comparison a conclusion will close this chapter.\\
Context of the chapter:\todo{update introduction}
\begin{itemize}
\item Description of EEG data
\item Describtion of performance measure method (practise of the Comparison to ICA) concluded by a new flow diagram.
\item Test and conclusion, including the flaws of the ICA comparison and lack of good estimate of A.
\item Description of different test approach, with different pros,including integrational functionality/connectivity(or something similar which is mentioned in ch 1.)
\item test and conclusion. 
\end{itemize}

\section{Data Description}
For this thesis a set of real EEG scalp measurements has been provided, from the the department of .... at AAU. 
The data base consist of 6 set of EEG measurements. For each of three test subjects two data set is provided, one where the test subject sit still with open eyes and one similar but with closed eyes.  
For the measurements an EEG cap with $32$ sensors measuring the scalp EEG signal with sample frequency at $512$Hz\todo{tjek sample frequency ant time} over a period of $XX$ seconds, resulting in $L = XX$ samples. That is 27 channels with names and position available in EEG.chanlocs structure\todo{Matlab eller python her?}.   
Before the data base was provided each raw data set had undergone the following preprocessing.
The data was bandpass filtered between 1 and 40 Hz. Then decomposed by ICA where the independent components related to eye activity or movement was removed. Thus, for every data set 27 sensors remains. 

One data set then consist solely of the measurement matrix $\mathbf{Y} \in \mathbb{R}^{27\times ??}$. The data set is then ready to be divided in segments for which a source matrix $\hat{\textbf{X}}$ is to be estimated by the main algorithm described in chapter \ref{ch:implementation}.

\section{Test Description}
The test procedure is now described through specification of the evaluation criteria and the practical implementation of the test.
Remember that the aim of the developed main algorithm is to estimated the source matrix in the case where the number of sensors is less than the number of active source signals -- $M<k\leq N.$.

\subsection{Performance Evaluation by Comparison to ICA}
From the description of ICA used on EEG measurements, cf. section \ref{sec:ICAsolution} ICA are considered unreliable when using low-density EEG equipment where $M<32$, but for $M\geq 32$ the currently considered the most reliably method for the source estimation. However note again that the true number of sources is not known thus there is always some unreliability to the result. 

From the view that the sources found by ICA where $N=M$ is the best estimate, it is possible to let that estimate serve as a reference to be compared to the estimates achieved when $M<N$. 
In practice that is to perform ICA on a data set $\textbf{Y}\in \mathbb{R}^{M \times L}$ resulting in $\hat{\textbf{X}}_{ICA}\in \mathbb{R}^{N\times L}$ where $M=N$, then a specific number of sensors are removed from the data set $\textbf{Y}$ such that $M<N$ then and $\hat{\textbf{X}}_{Main}\in \mathbb{R}^{N\times L}$ is estimated by the main algorithm. Then the performance of the main algorithm can be measured by comparison to the $\hat{\textbf{X}}_{ICA}$ -- does the main algorithm manage to find the same active sources as ICA, but for $M<N$?

In appendix \ref{app:ica_test}the ICA algorithm is verified on synthetic data without noise. It is found that ICA manage to estimate $\textbf{X}$ almost exact, when $M=N=k$. Furthermore it is seen that for $k < N$ ICA manage to estimate the zero rows as zero. This supports that the estimate by ICA can serve as a reference.     

To compare the two estimates the MSE, cf. section \ref{sec:mse}, is used. 
However an issue arise due to the fact that ICA do not manage to localise each of the found sources. That is the row of $\hat{\textbf{X}}_{ICA}$ do not correspond to the true $\textbf{X}$. Furhtermore the ICA algorithm is invariant toward phase and amplitude. This must necessarily worsen the resulting MSE.  

The issue is cover in appendix \ref{app:ica_test}. Here a function is considered, which manage to pair and fit the rows with the lowest mutual MSE an then arrange the rows of $\hat{\textbf{X}}_{ICA}$ such that $MSE(\textbf{X},\hat{\textbf{X}}_{ICA})$ is minimised. The fitting consist of a possible phase shift and scaling of the the amplitude.   
However this function have not been applied for this test as it as .... 

\subsection{Test Setup}
The test set up is visualised in figure \ref{fig:flow2} by a flow diagram, showing the essential steps of the test. 

\begin{figure}[H]
    \centering
	\includegraphics[scale=0.8]{figures/ch_7/flow2.png}
	\caption{Flow diagram for visualisation of the test procedure for one data set. Example given for $M<N$ where \texttt{request = 1/3} result in $M=9$.}
	\label{fig:flow2}
\end{figure}

%%% describtion of the flowchart - remainings
In the flow diagram the two estimation processes are seen to run parallel but taking the same input. 
Prior to the application of ICA the input is divided in to segments. That is the same segmentation as inside the main algorithm, cf. section \ref{sec:implementation_flow}.
The size of the segments are defined due to the expected stationarity of the sources. As described in the motivation chapter \ref{ch:motivation} sources are stationary if you look at sufficiently small intervals. 
For one second intervals this is not the case with our EEG measurements and one can therefore not have a entire row (source) which laid the tolerance interval\todo{lidt i tvivl om hvad vi mener her, hvordan har nul tolerencen noget med stationarity at gøre, sætter vi ikke først rækkerne nul efter ICA?}. One could decrease the length of the segments but one must also take in mind that smaller segments lead to more segments and therefore a higher computational complexity. Instead an average is introduced. For each rows (the sources) of each segments will be average such that one source is resembled by one average value. This average value will then be compared to the interval. If the value laid inside the tolerance interval, the whole row will be set equal to zero. The sources in each segment equal to zero are removed and the source matrix will now be of size $\mathbf{X}_s \in \mathbb{R}^{k \times L_s}$

%%%%
After the segmentation the ICA is performed to every segment, returning $\hat{\textbf{X}}_{ICA s} \in \mathbb{R}^{N \times L_s}$.
Remember that the number of sensors equal the number of sources, $M = N$, but as mentioned in chapter XX, the case of interest are the active sources $k$, and by that $N = k$. 
For each source matrix $\mathbf{X}_s$ one need to find the $k$ active sources but it is not as easy as one may have though. Each entries of the sources matrix are to small to be detected from being active (non-zero) or being non-active (zeros).
Instead a tolerance is defined for which values less will be determined as zeros. As the source matrices have positive and negative values a tolerance interval, an interval around zero, must be made. Let the tolerance be defined as tol = $[10E-03, -10E-03]$ where values inside this interval is set equal to zero.

By removing the zero rows the resulting estimate is reduced to $\hat{\textbf{X}}_{ICA s} \in \mathbb{R}^{k \times L_s}$. The found $k$ is now given as input to the main algorithm. Prior to the applying the main algorithm the input data has been reduced  such that .... \todo{afslutning her}   
 
As mentioned only the sensors $M$ is known from the EEG measurement data sets but with the source matrices achieved from the ICA algorithm $k$ is now known for each segment. One now have all the information need to used the main algorithm on the EEG measurement data sets. 
\\ \\
The described test is performed on the following three cases,
\begin{itemize}
\item \textbf{Test 0}: $M = N$ , to see the best possible result achieved by the main algorithm. 
\item \textbf{Test 1}: $M < N$ every third sensor is removed. 
\item \textbf{Test 2}: $M << N$ every second sensor is removed.
\end{itemize}
For each case the test is performed on one data set -- \texttt{S1\_Cclean} 

\section{Results}


\subsection{Case 0, $M=N$}
The results are plotted for data set S1\_Cclean. The data set consist of 144 time segment with $L_s = 516$ samples and $M\_ = 27$ sensors. Figure \ref{fig:M=N_1} show $MSE\left(\hat{\mathbf{X}}_{\text{main}},\hat{\mathbf{X}}_{\text{ICA}}\right)$ for all segments $s$. ICA is applied on $\textbf{Y}_s$ specified by $M\_ = 27$ and $L_s = 516$. The main algorithm is applied on $\textbf{Y}_s$ without any reduction hence specified by $M=27$ and $L_s=516$, given $\textbf{A}_{fix}$ and $N = k$ provided from ICA.
Figure \ref{fig:M=N_1_2} show the same plot but the y-axis is specified to the interval [-10,50] for better visualization.
Furthermore is the MSE tolerance $= 5$ plotted, indicting for each segment whether the estimate $\hat{\mathbf{X}}_{\text{main}}$ is sufficiency close to $\hat{\mathbf{X}}_{\text{main}}$. It is seen that for a majority of the segments the MSE lies under the tolerance, but single outliers appears for which the MSE of the segment is remarkably increased.    
\begin{figure}[H]
\begin{widepage}
    \begin{minipage}[t]{.45\textwidth}
		\centering
		\includegraphics[width=1\linewidth]{figures/ch_7/average_mse_non_removed_ica}
	\caption{$MSE\left(\hat{\mathbf{X}}_{\text{main}},\hat{\mathbf{X}}_{\text{ICA}}\right)$ for all 144 segments}
	\label{fig:M=N_1}
    \end{minipage} 
    \hspace{0.5cm}
    \begin{minipage}[t]{.45\textwidth}
        \centering
		\includegraphics[width=1\linewidth]{figures/ch_7/average_mse_non_removed_ica_zoom.png}
	\caption{$MSE\left(\hat{\mathbf{X}}_{\text{main}},\hat{\mathbf{X}}_{\text{ICA}}\right)$ for all 144 segments. Plotted only for the y-axis interval [-10, 50] for better visualisation.}
	\label{fig:M=N_1_2}
    \end{minipage}
\end{widepage}
\end{figure}
To investigate the behaviour of a single segment figure \ref{fig:M=N_2} show the MSE value computed for each row of the two estimates of a specific segment. That is $MSE\left(\hat{\mathbf{X}}_{\text{main}_{i}},\hat{\mathbf{X}}_{\text{ICA}_{i}}\right)$ for every row $i = 1, \hdots, k$ in time segment $s=56$. Additionally figure \ref{fig_M=N} show and compare the corresponding estimates for four random chosen sources. This allows for visual comparison of the estimates relative to the corresponding MSE value seen in figure \ref{fig:M=N_2}. Note that for better visual comparison each plotted row of $\hat{\textbf{X}}_{ICA}$ is scaled with respect to the max value of the corresponding row in $\hat{\textbf{X}}_{Main}$.
From figure \ref{fig:M=N_2} it is seen that the estimate of each source result in a relative low MSE which indicate that the main algorithm has managed to estimate the same source as the ICA algorithm. In contradiction to this, figure \ref{fig:M=N_3} do not confirm that the estimates are close, as generally the two signals in one plot does not follow the same trend.   
\begin{figure}[H]
\begin{widepage}
    \begin{minipage}[t]{.45\textwidth}
\centering
\includegraphics[width=1\linewidth]{figures/ch_7/mse_non_removed_ica_timeseg55.png}
\caption{$MSE\left(\hat{\mathbf{X}}_{\text{main}_{i}},\hat{\mathbf{X}}_{\text{ICA}_{i}}\right)$ for every row $i = 1, \hdots, k$ in time segment $s=56$.}
\label{fig:M=N_2}
\end{minipage} 
\hspace{0.5cm}
\begin{minipage}[t]{.45\textwidth}
\centering
\includegraphics[width=1\linewidth]{figures/ch_7/EEG_non_removed_scaled_timeseg55S1_CClean.png}
\caption{Plot of the $k = 13$ sources from $\hat{\mathbf{X}}_{\text{main}}$ and $\hat{\mathbf{X}}_{\text{ICA}}$ from time segment $s = 56$ with $M=N$}
	\label{fig:M=N_3}
    \end{minipage}
\end{widepage}
\end{figure}
The test is repeated for every data set, and the results are summarised in table \ref{tab:case_0}. 
The achieved results will serve as reference when analysing the results of the following cases where the main algorithm is applied on data set reduced with respect to the original data set.       
\begin{table}[]
\centering
\begin{tabular}{|c|c|c|c|c|c|c|}
\hline
\multirow{2}{*}{\textbf{\begin{tabular}[c]{@{}c@{}}Case 0 \\ $M=N$\end{tabular}}} & \multicolumn{2}{c|}{test subject 1} & \multicolumn{2}{c|}{test subject 2} & \multicolumn{2}{c|}{test subject 3} \\ \cline{2-7} 
                                                                                  & Open             & Close            & Open             & Close            & Open            & Close             \\ \hline
\multicolumn{1}{|c|}{Average MSE()}                                               & 2,138            & 4.336            & 1.692            & 16.35            & 6.397           & 19693.9           \\ \hline
\begin{tabular}[c]{@{}c@{}}Segments below \\ tolerance in \%\end{tabular}          & 93             & 87            & 99             & 61             & 85            & 70              \\ \hline
\end{tabular}
\caption{Summarised results for Case 0. Test is performed on the every data set.}
\label{tab:case_0}
\end{table}	


\subsection{Case 1, $M<N$}
Here the main algorithm is applied to a data set, where the number of sensors is reduced by one third. As such ICA is applied on the original data set with segments $\textbf{Y}_s$ specified by $M_= 27$ and $L_s = 516$. The main algorithm is applied on $\textbf{Y}_s$ specified by $M=18$ and $L_s=516$, given $\textbf{A}_{fix}$ and $N = k$ provided from ICA.  
The viewed plots correspond to those of case 0, but for reduce number of sources $M<N$, hence detailed plot description is omitted here.   

From figure \ref{fig:M<N_1} and \ref{fig:M<N_1_2} it is seen that the for a majority of the segments the MSE value is close to the tolerance, but the number of outliers has increased compared to case 0, indicating the that for an increased number of segments the main algorithm do not manage to estimate enough  sources sufficiently in order to stay below the tolerance.   
\begin{figure}[H]
\begin{widepage}
    \begin{minipage}[t]{.45\textwidth}
		\centering
		\includegraphics[width=1\linewidth]{figures/ch_7/average_mse_third_removed_ica}
	\caption{$MSE\left(\hat{\mathbf{X}}_{\text{main}},\hat{\mathbf{X}}_{\text{ICA}}\right)$ for all 144 segments}
	\label{fig:M<N_1}
    \end{minipage} 
\hspace{0.5cm}
    \begin{minipage}[t]{.45\textwidth}
        \centering
		\includegraphics[width=1\linewidth]{figures/ch_7/average_mse_third_removed_ica_zoom.png}
	\caption{$MSE\left(\hat{\mathbf{X}}_{\text{main}},\hat{\mathbf{X}}_{\text{ICA}}\right)$ for all 144 segments. Plotted only for the y-axis interval [-10, 50] for better visualisation.}
	\label{fig:M<N_1_2}
    \end{minipage}
\end{widepage}
\end{figure}
\noindent 
From figure \ref{fig:M<N_2} and \ref{fig:M<N_3} showing the results of segment 56, it is seen that the MSE for each source has increased slightly compare to case 0. This supports the observation from figure \ref{fig:M<N_1_2}.      
\begin{figure}[H]
\begin{widepage}
    \begin{minipage}[t]{.45\textwidth}
\centering
\includegraphics[width=1\linewidth]{figures/ch_7/mse_third_removed_ica_timeseg55.png}
\caption{$MSE\left(\hat{\mathbf{X}}_{\text{main}_{i}},\hat{\mathbf{X}}_{\text{ICA}_{i}}\right)$ for every row $i = 1, \hdots, k$ in time segment $s=56$.}
\label{fig:M<N_2}
\end{minipage} 
\hspace{0.5cm}
\begin{minipage}[t]{.45\textwidth}
\centering
\includegraphics[width=1\linewidth]{figures/ch_7/EEG_third_removed_scaled_timeseg55S1_CClean.png}
\caption{Plot of the $k = 13$ sources from $\hat{\mathbf{X}}_{\text{main}}$ and $\hat{\mathbf{X}}_{\text{ICA}}$ from time segment $s = 56$ with $M=N$}
	\label{fig:M<N_3}
    \end{minipage}
\end{widepage}
\end{figure}
The test is repeated for every data set, and the results are summarised in table \ref{tab:case_1}. Comparing table \ref{tab:case_1} to table \ref{tab:case_0}, summarising the results of case 0, it is seen that the percentage of segments below the tolerance are decreasing, with the majority being close to 50\%, thoroughly indicating that half of the time the main algorithm do not manage to provide a sufficient estimate when $M = 2/3N$.  

\begin{table}[h]
\centering
\begin{tabular}{|c|c|c|c|c|c|c|}
\hline
\multirow{2}{*}{\textbf{\begin{tabular}[c]{@{}c@{}}Case 1 \\ $M<N$\end{tabular}}} & \multicolumn{2}{c|}{test subject 1} & \multicolumn{2}{c|}{test subject 2} & \multicolumn{2}{c|}{test subject 3} \\ \cline{2-7} 
                                                                                  & Open             & Close            & Open             & Close            & Open              & Close           \\ \hline
\multicolumn{1}{|c|}{Average MSE()}                                               & 21.64            & 8.348            & 454.7            & 91.19            & 17316.5           & 25.73           \\ \hline
\begin{tabular}[c]{@{}c@{}}Segments below \\ tol in percent\end{tabular}          & 47.1             & 61.8             & 62.6             & 47.5             & 73.8              & 52.8            \\ \hline
\end{tabular}
\caption{Summarised results for Case 1. Test is performed on the every data set.}
\label{tab:case_1}
\end{table}


\subsection{Case 2, $M<<N$}
Here the main algorithm is applied to a data set, where the number of sensors is reduced to half. As such ICA is applied on the original data set with segments $\textbf{Y}_s$ specified by $M_= 27$ and $L_s = 516$. The main algorithm is applied on $\textbf{Y}_s$ specified by $M=13$\todo{13 eller 14?} and $L_s=516$, given $\textbf{A}_{fix}$ and $N = k$ provided from ICA.  
The viewed plots correspond to those of case 0 and case 1, but for further reduce number of sources $M<<N$, hence detailed plot description is omitted.   

From figure \ref{fig:M<<N_1} and \ref{fig:M<<N_1_2} it is seen that the MSE value for each segment is more widely scatted around the tolerance, compared to case 0 and 1. Outliers where the MSE value has increased significantly do also occur, similar to case 1. This indicates that the performance of the main algorithm has decreased further, compared to case 1. 

\begin{figure}[H]
\begin{widepage}
    \begin{minipage}[t]{.45\textwidth}
		\centering
		\includegraphics[width=1\linewidth]{figures/ch_7/average_mse_second_removed_ica}
	\caption{$MSE\left(\hat{\mathbf{X}}_{\text{main}},\hat{\mathbf{X}}_{\text{ICA}}\right)$ for all 144 segments}
	\label{fig:M<<N_1}
    \end{minipage} 
\hspace{0.5cm}
    \begin{minipage}[t]{.45\textwidth}
        \centering
		\includegraphics[width=1\linewidth]{figures/ch_7/average_mse_second_removed_ica_zoom.png}
	\caption{$MSE\left(\hat{\mathbf{X}}_{\text{main}},\hat{\mathbf{X}}_{\text{ICA}}\right)$ for all 144 segments. Plotted only for the y-axis interval [-10, 50] for better visualisation.}
	\label{fig:M<<N_1_2}
    \end{minipage}
\end{widepage}
\end{figure}
\noindent 
The above indication is supported by figure \ref{fig:M<<N_2} and \ref{fig:M<<N_3} showing an general increase in MSE. However segment 56 still makes a fairly good example as the majority of the sources have achieves a MSE below the tolerance of 5. From figure \ref{fig:M<<N_3} the increased MSE do not appear visually compare to either case 1 or case 0.        
\todo{tjek source nr. i label på alle signal plottene det stemmer ikke her }
\begin{figure}[H]
\begin{widepage}
    \begin{minipage}[t]{.49\textwidth}
\centering
\includegraphics[width=1\linewidth]{figures/ch_7/mse_second_removed_ica_timeseg55_kopi.png}
\caption{$MSE\left(\hat{\mathbf{X}}_{\text{main}_{i}},\hat{\mathbf{X}}_{\text{ICA}_{i}}\right)$ for every row $i = 1, \hdots, k$ in time segment $s=56$.}
\label{fig:M<<N_2}
\end{minipage} 
\hspace{.5cm}
\begin{minipage}[t]{.49\textwidth}
\centering
\includegraphics[width=1\linewidth]{figures/ch_7/EEG_second_removed_scaled_timeseg55S1_CClean_kopi.png}
\caption{Plot of the $k = 13$ sources from $\hat{\mathbf{X}}_{\text{main}}$ and $\hat{\mathbf{X}}_{\text{ICA}}$ from time segment $s = 56$ with $M<N$}
	\label{fig:M<<N_3}
    \end{minipage}
\end{widepage}
\end{figure}

The test is repeated for every data set, and the results are summarised in table \ref{tab:case_2}. Comparing table \ref{tab:case_0} to table \ref{tab:case_1}, summarising the results of case 1, it is generally seen that the percentage of segments below the tolerance is not decreased but improved, but without getting close to the tendency from case 0.  This indicates...


\begin{table}[]
\begin{tabular}{|c|l|l|l|l|l|l|}
\hline
\multirow{2}{*}{\textbf{\begin{tabular}[c]{@{}c@{}}Case 2\\ $M<<N$\end{tabular}}} & \multicolumn{2}{l|}{test subject 1} & \multicolumn{2}{l|}{test subject 2} & \multicolumn{2}{l|}{test subject 3} \\ \cline{2-7} 
                                                                                  & Open             & Close            & Open             & Close            & Open             & Close            \\ \hline
\multicolumn{1}{|l|}{Average MSE()}                                               & 15.64            & 19.36            & 345.4            & 13.57            & 315.3            & 63.85            \\ \hline
\begin{tabular}[c]{@{}c@{}}Segments below \\ tol in percent\end{tabular}          & 82.1             & 63.8             & 40.4             & 72.9             & 63.5             & 65.8             \\ \hline
\end{tabular}
\caption{results...}
\label{tab:case_2}
\end{table}

%%%%%%%%%%%%%%%%%%%%%%%%%%%%%%%%%%%%%%%%%%%%%%%%%%%%%%%%% old
%\subsection{Test on EEG Data Set with $\frac{1}{3} M<N$}
%For this test the S1\_Cclean EEG data set will be used and case of interest is when $\frac{1}{3} M < N$, every third sensors is removed from S1\_Cclean EEG data set. The test is performance on 144 time segment with $L_s = 516$ and $M = 18$ sensors. From ICA 144 different $k$ values will represent the activation of sources. The obtain source matrices will only consists of the active sources.
%The aim of this test is to see if one still can recover the same sources as the in the case of $M=N$. It is expected that a higher MSE will be seen.
%
%Figure \ref{fig:M<N_1} visualize the average MSE values obtain from $\hat{\mathbf{X}}_{\text{main}}$ and $\hat{\mathbf{X}}_{\text{ICA}}$, one MSE value per time segment.
%\begin{figure}[H]
%    \begin{minipage}[t]{.45\textwidth}
%		\centering
%		\includegraphics[scale=0.5]{figures/ch_7/AveMSE_3M_N.png}
%	\caption{Average MSE of the main algorithm with $M=18$, $L = 516$ and a different $k$ for each time segment. The main algorithm and ICA is used on the S1\_Cclean EEG data set.}
%	\label{fig:M<N_1}
%    \end{minipage} 
%    \hfill
%    \begin{minipage}[t]{.45\textwidth}
%        \centering
%		\includegraphics[scale=0.5]{figures/ch_7/AveMSE_3M_N_zoom.png}
%	\caption{Average MSE of the main algorithm with $M=18$, $L = 516$ and a different $k$ for each time segment. The main algorithm and ICA is used on the S1\_Cclean EEG data set -- zoom version.}
%	\label{fig:M<N_1_2}
%    \end{minipage}
%\end{figure}
%\noindent 
% 
%\begin{figure}[H]
%\centering
%\includegraphics[scale=0.5]{figures/ch_7/MSE_3M_N.png}
%	\caption{MSE of the main algorithm with $M=18$, $L = 516$ and $k=13$ for time segment $s=45$. The main algorithm and ICA is used on the S1\_Cclean EEG data set.}
%\label{fig:M<N_2}
%\end{figure}
%\noindent
%From figure \ref{fig:M<N_1_2} it is seen that the average MSE of $\hat{\mathbf{X}}_{\text{main}}$ compared to $\hat{\mathbf{X}}_{\text{ICA}}$ mostly laid between 0 and 10 but a mosre higher MSE tendency can be seen compared to the MSE from $M=N$. 
%
%In figure \ref{fig:M<N_2} the MSE values of one time segment is plotted to visualize how the sources performs compared to the ICA. The time segment in interest is $s = 45$ so the segment of the 45 second in S1\_Cclean. 
%From figure \ref{fig:M<N_2} it seen that the estimation of the sources in this time segment have a higher MSE value compared to \ref{fig:M=N_2}. This verifies the expectation that the less sensors a higher MSE values will occur.
%
%The last figure illustrate the $k = 13$ sources from $\hat{\mathbf{X}}_{\text{main}}$ and $\hat{\mathbf{X}}_{\text{ICA}}$ plot on each other.
%\begin{figure}[H]
%    \centering
%	\includegraphics[scale=0.5]{figures/ch_7/Sources_3M_N.png}
%	\caption{Plot of the $k = 13$ sources from $\hat{\mathbf{X}}_{\text{main}}$ and $\hat{\mathbf{X}}_{\text{ICA}}$ from time segment $s = 45$ with $\frac{1}{3} M<N$}
%	\label{fig:M<N_3}
%\end{figure} 
%\noindent
%
%\subsection{Test on EEG Data Set with $\frac{1}{2} M<N$}
%For this test the S1\_Cclean EEG data set will be used and case of interest is when $\frac{1}{2} M < N$, every second sensors is removed from S1\_Cclean EEG data set. The test is performance on 144 time segment with $L_s = 516$ and $M = 13$ sensors. From ICA 144 different $k$ values will represent the activation of sources. The obtain source matrices will only consists of the active sources.
%The aim of this test is the same as the previous, to see if one still can recover the same sources as the in the case of $M=N$. It is still expected to see a higher MSE.
%
%Figure \ref{fig:M<<N_1} visualize the average MSE values obtain from $\hat{\mathbf{X}}_{\text{main}}$ and $\hat{\mathbf{X}}_{\text{ICA}}$, one MSE value per time segment.
%
%\begin{figure}[H]
%    \begin{minipage}[t]{.45\textwidth}
%		\centering
%		\includegraphics[scale=0.5]{figures/ch_7/AveMSE_2M_N.png}
%	\caption{Average MSE of the main algorithm with $M=13$, $L = 516$ and a different $k$ for each time segment. The main algorithm and ICA is used on the S1\_Cclean EEG data set.}
%	\label{fig:M<<N_1}
%    \end{minipage} 
%    \hfill
%    \begin{minipage}[t]{.45\textwidth}
%        \centering
%		\includegraphics[scale=0.5]{figures/ch_7/AveMSE_2M_N_zoom.png}
%	\caption{Average MSE of the main algorithm with $M=13$, $L = 516$ and a different $k$ for each time segment. The main algorithm and ICA is used on the S1\_Cclean EEG data set -- zoom version.}
%	\label{fig:M<<N_1_2}
%    \end{minipage}
%\end{figure}
%\noindent
%
%\begin{figure}[H]
%\centering
%\includegraphics[scale=0.5]{figures/ch_7/MSE_2M_N.png}
%\caption{MSE of the main algorithm with $M=13$, $L = 516$ and $k=13$ for time segment $s=45$. The main algorithm and ICA is used on the S1\_Cclean EEG data set.}
%	\label{fig:M<<N_2}
%\label{fig:M<N_2}
%\end{figure}
%\noindent
%From figure \ref{fig:M<<N_1_2} it is seen that the average MSE of $\hat{\mathbf{X}}_{\text{main}}$ compared to $\hat{\mathbf{X}}_{\text{ICA}}$ mostly laid between 0 and 10. 
%
%In figure \ref{fig:M<<N_2} the MSE values of one time segment is plotted to visualize how the sources performs compared to the ICA. The time segment in interest is $s = 45$ so the segment of the 45 second in S1\_Cclean. 
%From figure \ref{fig:M<<N_2} it seen that the estimation of the sources in this time segment have a generally low MSE value and is almost the same compared to \ref{fig:M=N_2}. 
%
%The last figure illustrate the $k = 13$ sources from $\hat{\mathbf{X}}_{\text{main}}$ and $\hat{\mathbf{X}}_{\text{ICA}}$ plot on each other.
%\begin{figure}[H]
%    \centering
%	\includegraphics[scale=0.5]{figures/ch_7/Sources_2M_N.png}
%	\caption{Plot of the $k = 13$ sources from $\hat{\mathbf{X}}_{\text{main}}$ and $\hat{\mathbf{X}}_{\text{ICA}}$ from time segment $s = 45$ with $\frac{1}{2} M<N$}
%	\label{fig:M<<N_3}
%\end{figure} 
%\noindent


\section{Conclusion}