\chapter{Test on EEG measurements}\label{ch:eeg_test}
The main algorithm was implemented and tested on simulated data in chapter \ref{ch:implementation}. 
In this chapter the main algorithm is tested on EEG measurements, for which it is intended. 
Two different approaches are considered with respect to evaluating the resulting estimates of the source signals, ICA comparison and alpha wave analysis, respectively.

At first the provided data sets with EEG measurements are described. 
Followed by a test description and an analysis of the results for both of the evaluation approaches. 
Finally, a summary is provided to highlight the conclusions.  

\section{Data Description}
For this thesis a data base of real EEG scalp measurements has been provided, from the department of electronic systems at Aalborg University. 
The data base consists of data sets of EEG measurements resulting from three test subjects. 
For each of three test subjects, two data set is provided. One where the test subject sits still with open eyes and one similar but with closed eyes, resulting in a data base with 6 data sets. 
For the measurements an EEG cap with $32$ sensors measuring the scalp EEG signal with sample frequency at $512$ Hz over a varying period time.
That is 27 channels with names and position available in \texttt{EEG.chanlocs} structure.
The data sets are specified in table \ref{tab:data_spec}.
\begin{table}[H]
\centering
\begin{tabular}{|l|l|l|l|l|l|l|}
\hline
 \multicolumn{2}{|l|}{EEG measurements } & $M\_$ & $L$    & $f_s$ & $n_{\text{seg}}$ & $L_s$ \\ \hline
1.& \texttt{S1\_Cclean} & 27  & 74161  & 512   & 144       & 516   \\ \hline
2.& \texttt{S1\_Oclean} & 27  & 63245  & 512   & 123       & 514   \\ \hline
3.& \texttt{S2\_Cclean} & 27  & 94918  & 512   & 185       & 513   \\ \hline
4.& \texttt{S2\_Oclean} & 27  & 117900 & 512   & 230       & 512   \\ \hline
5.& \texttt{S3\_Cclean} & 27  & 110060 & 512   & 214       & 514   \\ \hline
6.& \texttt{S3\_Oclean} & 27  & 114065 & 512   & 222        & 513    \\ \hline
\end{tabular}
\caption{Specifications of the available data sets of EEG measurements, including specification of the segments resulting from segmentation into segments of length $t=1$ seconds.}
\label{tab:data_spec}
\end{table}
\noindent
Before the data base was provided each raw data set had undergone the following preprocessing.
The data were bandpass filtered between 1 and 40 Hz. Then decomposed by ICA where the independent components related to eye activity or movement was removed. 
Thus, for every data set 27 sensors remains. 
One data set then consist solely of the measurement matrix $\mathbf{Y} \in \mathbb{R}^{27\times L}$.

\section{Test Description}\label{seg:main_test_description}
The test procedure is now described through specification of the evaluation criteria and the practical implementation of the test.
Remember the aim of the implemented main algorithm is to estimated the source matrix in the case where the number of sensors is less than the number of active source signals -- $M < k \leq N$.

\subsection{Performance Evaluation by Comparison to ICA}
From the description of ICA used on EEG measurements, cf. section \ref{sec:ICAsolution}, ICA is considered unreliable when using low density EEG equipment where $M < 32$. 
For $M \geq 32$ ICA is currently considered the most reliable method for the source recovery. 
However, note that the true number of sources is unknown thus there is always some unreliability to the result. 

From the view that the sources found by ICA is the best estimate, it is possible to let that estimate serve as a reference for comparison of estimates recovered from $M < N$. 
In practice that is to perform ICA on a data set $\mathbf{Y} \in \mathbb{R}^{M \times L}$ resulting in $\hat{\mathbf{X}}_{\text{ICA}} \in \mathbb{R}^{N \times L}$ where $M = N$. 
Then a specific number of sensors are removed from the data set $\mathbf{Y}$ such that $\hat{\mathbf{X}}_{\text{main}} \in \mathbb{R}^{N\times L}$ is estimated by the main algorithm  with $M < N$. 
The performance of the main algorithm can be measured by comparison to the $\hat{\mathbf{X}}_{\text{ICA}}$. 
Does the main algorithm manage to find the same active sources as ICA, but for $M < N$?

In appendix \ref{app:ica_test} the ICA algorithm is verified on simulated data without noise. 
It was found that ICA manages to estimate $\mathbf{X}$ almost exact, when $M = N = k$. 
Furthermore, it is seen that for $k < N$ ICA manages to estimate the zero rows as zero. 
This supports that the estimate by ICA can serve as a reference.     

To compare the two estimates the MSE, cf. section \ref{sec:mse}, is used. 
However, an issue arises due to the fact that ICA do not manage to localize each of the found sources. 
That is the row of $\hat{\mathbf{X}}_{\text{ICA}}$ do not correspond to the true $\mathbf{X}$. 
Furthermore, the ICA algorithm is invariant toward the phase and the amplitude. 
This must necessarily worsen the resulting MSE.  

The issue is covered in appendix \ref{app:ica_test}. 
Here a function is considered, which manage to pair and fit the rows with the lowest mutual MSE and then arrange the rows of $\hat{\mathbf{X}}_{\text{ICA}}$ such that $\text{MSE}(\mathbf{X}, \hat{\textbf{X}}_{\text{ICA}})$ is minimised. 
The fitting consists of a possible phase shift and scaling of the amplitude. 
The right optimal fit was found through a brute-force search, however this is impossible as the possible number of combinations increases as $k$ increases. 
This suggests the definition of an optimization problem minimizing the resulting MSE with respect to the combination of row indexes, possible phase and corresponding amplitude scaling. 
Unfortunately, a successful optimization was not achieved within the time scope of this thesis.
Thus, the fitting process is not applied to the results achieved from the EEG measurements in this chapter. 
This factor must be taken into account when evaluating the results.

Consider again the resulting MSE$(\hat{\mathbf{X}}_{\text{ICA}}, \hat{\mathbf{X}}_{\text{main}})$. 
To evaluate further on the question whether the same sources have been found a tolerance for the MSE is introduced. 
With MSE$(\hat{\mathbf{X}}_{\text{ICA}}, \hat{\mathbf{X}}_{\text{main}})$ being an average over the MSE of each row within one segment a low value indicate that main part of the rows makes an estimate similar the estimate from ICA. 
From this perspective a tolerance for MSE$(\hat{\mathbf{X}}_{\text{ICA}}, \hat{\mathbf{X}}_{\text{main}})$ decides whether the same sources are achieved with success. 
The tolerance is set to six due to previous observations with respect to the simulated data. Especially figure \ref{fig:AR2} indicate that an MSE below six is achievable for a system where $M << N$ with use of $\mathbf{A}_{\text{true}}$. 
It could be argued that the tolerance should be increased as the estimate of $\mathbf{A}$ is not expected to be nearly as good. 
However, this could give a distorted image of the results.    

\subsection{Test Setup}
The test setup is visualized in figure \ref{fig:flow2} by a flow diagram, showing the essential steps of the test. 
\begin{figure}[H]
    \centering
	\includegraphics[scale=1]{figures/ch_7/flow2.png}
	\caption{Flow diagram for visualization of the test procedure for one data set. Example given for $M < N$ where \texttt{request = 1/3} result in $M = 9$.}
	\label{fig:flow2}
\end{figure}
\noindent
In the flow diagram the two estimation processes are seen to run parallel but taking the same input. 
Prior to the application of ICA, the input is divided into segments. 
That is the same segmentation as inside the main algorithm, cf. section \ref{sec:implementation_flow}.
The size of the segments is defined due to the expected stationarity of the sources. 
As described in the motivation chapter \ref{ch:motivation} sources are stationary if you look at sufficiently small intervals. 
Segments at $t = 1$ second is chosen from the assumption that the brain activity can be assumed stationary within short time interval.
Furthermore, one must take in mind that shorter time interval lead to more segments and therefore a higher computational complexity. 
After the segmentation the ICA is applied to every segment, returning $\hat{\mathbf{X}}_{\text{ICA} s} \in \mathbb{R}^{M \times L_s}$.
From appendix \ref{app:ica_test} it is seen that ICA manage to estimate the non-active sources by zero rows, when no noise is present. 
When ICA is applied to the EEG measurements noise is expected. 
Thus the non-active sources defined by the average amplitude being within a tolerance interval around zero, defined by tol = $[10E-03, -10E-03]$. 
When the non-active sources are identified, they are removed and the resulting estimate is reduced to $\hat{\mathbf{X}}_{\text{ICA} s} \in \mathbb{R}^{k \times L_s}$. 
The found number of active sources $k$ is then given as input to the main algorithm where $k = N$. 
In parallel to the ICA process the input data is reduced as specified in the previous section. 
Then the main algorithm is applied to the reduced data set. 
Within the main algorithm the data are like wise divided in segments and an estimate $\hat{\mathbf{X}}_{\text{main} s} \in \mathbb{R}^{k \times L_s}$ is returned. 
Note that $\mathbf{A}_{\text{fix}}$ is given as a manual input, replacing the Cov-DL algorithm as concluded in chapter \ref{ch:implementation}.
At the end the resulting two estimates have the same dimensions which allow for $\hat{\mathbf{X}}_{\text{main} s}$ to be evaluated with respect to $\hat{\mathbf{X}}_{\text{ICA} s}$ by the MSE. 
\\ \\
The described test is performed on the following three cases,
\begin{itemize}
\item \textbf{Case 0}: $M = N$ to see the best possible result achieved by the main algorithm. 
\item \textbf{Case 1}: $M < N$ every third sensor is removed. 
\item \textbf{Case 2}: $M << N$ every second sensor is removed.
\end{itemize}

\section{Results}
For each case the test is performed on all the data sets specified in table \ref{tab:data_spec}.
The results are visualized for one data set for visual understanding.
Lastly, the results of all three data sets are compared in a table.  

The results are plotted for data set S1\_Cclean. 
The data set consist of 144 time segments with $L_s = 516$ samples and $M\_ = 27$ sensors. 
\subsection{Case 0, $M=N$}
The results are plotted for data set S1\_Cclean. The data set consist of 144 time segment with $L_s = 516$ samples and $M\_ = 27$ sensors. Figure \ref{fig:M=N_1} show $MSE\left(\hat{\mathbf{X}}_{\text{main}},\hat{\mathbf{X}}_{\text{ICA}}\right)$ for all segments $s$. ICA is applied on $\textbf{Y}_s$ specified by $M\_ = 27$ and $L_s = 516$. The main algorithm is applied on $\textbf{Y}_s$ without any reduction hence specified by $M=27$ and $L_s=516$, given $\textbf{A}_{fix}$ and $N = k$ provided from ICA.
Figure \ref{fig:M=N_1_2} show the same plot but the y-axis is specified to the interval [-10,50] for better visualization.
Furthermore is the MSE tolerance $= 5$ plotted, indicting for each segment whether the estimate $\hat{\mathbf{X}}_{\text{main}}$ is sufficiency close to $\hat{\mathbf{X}}_{\text{main}}$. It is seen that for a majority of the segments the MSE lies under the tolerance, but single outliers appears for which the MSE of the segment is remarkably increased.    
\begin{figure}[H]
\begin{widepage}
    \begin{minipage}[t]{.45\textwidth}
		\centering
		\includegraphics[width=1\linewidth]{figures/ch_7/average_mse_non_removed_ica}
	\caption{$MSE\left(\hat{\mathbf{X}}_{\text{main}},\hat{\mathbf{X}}_{\text{ICA}}\right)$ for all 144 segments}
	\label{fig:M=N_1}
    \end{minipage} 
    \hspace{0.5cm}
    \begin{minipage}[t]{.45\textwidth}
        \centering
		\includegraphics[width=1\linewidth]{figures/ch_7/average_mse_non_removed_ica_zoom.png}
	\caption{$MSE\left(\hat{\mathbf{X}}_{\text{main}},\hat{\mathbf{X}}_{\text{ICA}}\right)$ for all 144 segments. Plotted only for the y-axis interval [-10, 50] for better visualisation.}
	\label{fig:M=N_1_2}
    \end{minipage}
\end{widepage}
\end{figure}
To investigate the behaviour of a single segment figure \ref{fig:M=N_2} show the MSE value computed for each row of the two estimates of a specific segment. That is $MSE\left(\hat{\mathbf{X}}_{\text{main}_{i}},\hat{\mathbf{X}}_{\text{ICA}_{i}}\right)$ for every row $i = 1, \hdots, k$ in time segment $s=56$. Additionally figure \ref{fig_M=N} show and compare the corresponding estimates for four random chosen sources. This allows for visual comparison of the estimates relative to the corresponding MSE value seen in figure \ref{fig:M=N_2}. Note that for better visual comparison each plotted row of $\hat{\textbf{X}}_{ICA}$ is scaled with respect to the max value of the corresponding row in $\hat{\textbf{X}}_{Main}$.
From figure \ref{fig:M=N_2} it is seen that the estimate of each source result in a relative low MSE which indicate that the main algorithm has managed to estimate the same source as the ICA algorithm. In contradiction to this, figure \ref{fig:M=N_3} do not confirm that the estimates are close, as generally the two signals in one plot does not follow the same trend.   
\begin{figure}[H]
\begin{widepage}
    \begin{minipage}[t]{.45\textwidth}
\centering
\includegraphics[width=1\linewidth]{figures/ch_7/mse_non_removed_ica_timeseg55.png}
\caption{$MSE\left(\hat{\mathbf{X}}_{\text{main}_{i}},\hat{\mathbf{X}}_{\text{ICA}_{i}}\right)$ for every row $i = 1, \hdots, k$ in time segment $s=56$.}
\label{fig:M=N_2}
\end{minipage} 
\hspace{0.5cm}
\begin{minipage}[t]{.45\textwidth}
\centering
\includegraphics[width=1\linewidth]{figures/ch_7/EEG_non_removed_scaled_timeseg55S1_CClean.png}
\caption{Plot of the $k = 13$ sources from $\hat{\mathbf{X}}_{\text{main}}$ and $\hat{\mathbf{X}}_{\text{ICA}}$ from time segment $s = 56$ with $M=N$}
	\label{fig:M=N_3}
    \end{minipage}
\end{widepage}
\end{figure}
The test is repeated for every data set, and the results are summarised in table \ref{tab:case_0}. 
The achieved results will serve as reference when analysing the results of the following cases where the main algorithm is applied on data set reduced with respect to the original data set.       
\begin{table}[]
\centering
\begin{tabular}{|c|c|c|c|c|c|c|}
\hline
\multirow{2}{*}{\textbf{\begin{tabular}[c]{@{}c@{}}Case 0 \\ $M=N$\end{tabular}}} & \multicolumn{2}{c|}{test subject 1} & \multicolumn{2}{c|}{test subject 2} & \multicolumn{2}{c|}{test subject 3} \\ \cline{2-7} 
                                                                                  & Open             & Close            & Open             & Close            & Open            & Close             \\ \hline
\multicolumn{1}{|c|}{Average MSE()}                                               & 2,138            & 4.336            & 1.692            & 16.35            & 6.397           & 19693.9           \\ \hline
\begin{tabular}[c]{@{}c@{}}Segments below \\ tolerance in \%\end{tabular}          & 93             & 87            & 99             & 61             & 85            & 70              \\ \hline
\end{tabular}
\caption{Summarised results for Case 0. Test is performed on the every data set.}
\label{tab:case_0}
\end{table}	


\subsection{Case 1, $M<N$}
Here the main algorithm is applied to a data set, where the number of sensors is reduced by one third. As such ICA is applied on the original data set with segments $\textbf{Y}_s$ specified by $M_= 27$ and $L_s = 516$. The main algorithm is applied on $\textbf{Y}_s$ specified by $M=18$ and $L_s=516$, given $\textbf{A}_{fix}$ and $N = k$ provided from ICA.  
The viewed plots correspond to those of case 0, but for reduce number of sources $M<N$, hence detailed plot description is omitted here.   

From figure \ref{fig:M<N_1} and \ref{fig:M<N_1_2} it is seen that the for a majority of the segments the MSE value is close to the tolerance, but the number of outliers has increased compared to case 0, indicating the that for an increased number of segments the main algorithm do not manage to estimate enough  sources sufficiently in order to stay below the tolerance.   
\begin{figure}[H]
\begin{widepage}
    \begin{minipage}[t]{.45\textwidth}
		\centering
		\includegraphics[width=1\linewidth]{figures/ch_7/average_mse_third_removed_ica}
	\caption{$MSE\left(\hat{\mathbf{X}}_{\text{main}},\hat{\mathbf{X}}_{\text{ICA}}\right)$ for all 144 segments}
	\label{fig:M<N_1}
    \end{minipage} 
\hspace{0.5cm}
    \begin{minipage}[t]{.45\textwidth}
        \centering
		\includegraphics[width=1\linewidth]{figures/ch_7/average_mse_third_removed_ica_zoom.png}
	\caption{$MSE\left(\hat{\mathbf{X}}_{\text{main}},\hat{\mathbf{X}}_{\text{ICA}}\right)$ for all 144 segments. Plotted only for the y-axis interval [-10, 50] for better visualisation.}
	\label{fig:M<N_1_2}
    \end{minipage}
\end{widepage}
\end{figure}
\noindent 
From figure \ref{fig:M<N_2} and \ref{fig:M<N_3} showing the results of segment 56, it is seen that the MSE for each source has increased slightly compare to case 0. This supports the observation from figure \ref{fig:M<N_1_2}.      
\begin{figure}[H]
\begin{widepage}
    \begin{minipage}[t]{.45\textwidth}
\centering
\includegraphics[width=1\linewidth]{figures/ch_7/mse_third_removed_ica_timeseg55.png}
\caption{$MSE\left(\hat{\mathbf{X}}_{\text{main}_{i}},\hat{\mathbf{X}}_{\text{ICA}_{i}}\right)$ for every row $i = 1, \hdots, k$ in time segment $s=56$.}
\label{fig:M<N_2}
\end{minipage} 
\hspace{0.5cm}
\begin{minipage}[t]{.45\textwidth}
\centering
\includegraphics[width=1\linewidth]{figures/ch_7/EEG_third_removed_scaled_timeseg55S1_CClean.png}
\caption{Plot of the $k = 13$ sources from $\hat{\mathbf{X}}_{\text{main}}$ and $\hat{\mathbf{X}}_{\text{ICA}}$ from time segment $s = 56$ with $M=N$}
	\label{fig:M<N_3}
    \end{minipage}
\end{widepage}
\end{figure}
The test is repeated for every data set, and the results are summarised in table \ref{tab:case_1}. Comparing table \ref{tab:case_1} to table \ref{tab:case_0}, summarising the results of case 0, it is seen that the percentage of segments below the tolerance are decreasing, with the majority being close to 50\%, thoroughly indicating that half of the time the main algorithm do not manage to provide a sufficient estimate when $M = 2/3N$.  

\begin{table}[h]
\centering
\begin{tabular}{|c|c|c|c|c|c|c|}
\hline
\multirow{2}{*}{\textbf{\begin{tabular}[c]{@{}c@{}}Case 1 \\ $M<N$\end{tabular}}} & \multicolumn{2}{c|}{test subject 1} & \multicolumn{2}{c|}{test subject 2} & \multicolumn{2}{c|}{test subject 3} \\ \cline{2-7} 
                                                                                  & Open             & Close            & Open             & Close            & Open              & Close           \\ \hline
\multicolumn{1}{|c|}{Average MSE()}                                               & 21.64            & 8.348            & 454.7            & 91.19            & 17316.5           & 25.73           \\ \hline
\begin{tabular}[c]{@{}c@{}}Segments below \\ tol in percent\end{tabular}          & 47.1             & 61.8             & 62.6             & 47.5             & 73.8              & 52.8            \\ \hline
\end{tabular}
\caption{Summarised results for Case 1. Test is performed on the every data set.}
\label{tab:case_1}
\end{table}


\subsection{Case 2, $M<<N$}
Here the main algorithm is applied to a data set, where the number of sensors is reduced to half. As such ICA is applied on the original data set with segments $\textbf{Y}_s$ specified by $M_= 27$ and $L_s = 516$. The main algorithm is applied on $\textbf{Y}_s$ specified by $M=13$\todo{13 eller 14?} and $L_s=516$, given $\textbf{A}_{fix}$ and $N = k$ provided from ICA.  
The viewed plots correspond to those of case 0 and case 1, but for further reduce number of sources $M<<N$, hence detailed plot description is omitted.   

From figure \ref{fig:M<<N_1} and \ref{fig:M<<N_1_2} it is seen that the MSE value for each segment is more widely scatted around the tolerance, compared to case 0 and 1. Outliers where the MSE value has increased significantly do also occur, similar to case 1. This indicates that the performance of the main algorithm has decreased further, compared to case 1. 

\begin{figure}[H]
\begin{widepage}
    \begin{minipage}[t]{.45\textwidth}
		\centering
		\includegraphics[width=1\linewidth]{figures/ch_7/average_mse_second_removed_ica}
	\caption{$MSE\left(\hat{\mathbf{X}}_{\text{main}},\hat{\mathbf{X}}_{\text{ICA}}\right)$ for all 144 segments}
	\label{fig:M<<N_1}
    \end{minipage} 
\hspace{0.5cm}
    \begin{minipage}[t]{.45\textwidth}
        \centering
		\includegraphics[width=1\linewidth]{figures/ch_7/average_mse_second_removed_ica_zoom.png}
	\caption{$MSE\left(\hat{\mathbf{X}}_{\text{main}},\hat{\mathbf{X}}_{\text{ICA}}\right)$ for all 144 segments. Plotted only for the y-axis interval [-10, 50] for better visualisation.}
	\label{fig:M<<N_1_2}
    \end{minipage}
\end{widepage}
\end{figure}
\noindent 
The above indication is supported by figure \ref{fig:M<<N_2} and \ref{fig:M<<N_3} showing an general increase in MSE. However segment 56 still makes a fairly good example as the majority of the sources have achieves a MSE below the tolerance of 5. From figure \ref{fig:M<<N_3} the increased MSE do not appear visually compare to either case 1 or case 0.        
\todo{tjek source nr. i label på alle signal plottene det stemmer ikke her }
\begin{figure}[H]
\begin{widepage}
    \begin{minipage}[t]{.49\textwidth}
\centering
\includegraphics[width=1\linewidth]{figures/ch_7/mse_second_removed_ica_timeseg55_kopi.png}
\caption{$MSE\left(\hat{\mathbf{X}}_{\text{main}_{i}},\hat{\mathbf{X}}_{\text{ICA}_{i}}\right)$ for every row $i = 1, \hdots, k$ in time segment $s=56$.}
\label{fig:M<<N_2}
\end{minipage} 
\hspace{.5cm}
\begin{minipage}[t]{.49\textwidth}
\centering
\includegraphics[width=1\linewidth]{figures/ch_7/EEG_second_removed_scaled_timeseg55S1_CClean_kopi.png}
\caption{Plot of the $k = 13$ sources from $\hat{\mathbf{X}}_{\text{main}}$ and $\hat{\mathbf{X}}_{\text{ICA}}$ from time segment $s = 56$ with $M<N$}
	\label{fig:M<<N_3}
    \end{minipage}
\end{widepage}
\end{figure}

The test is repeated for every data set, and the results are summarised in table \ref{tab:case_2}. Comparing table \ref{tab:case_0} to table \ref{tab:case_1}, summarising the results of case 1, it is generally seen that the percentage of segments below the tolerance is not decreased but improved, but without getting close to the tendency from case 0.  This indicates...


\begin{table}[]
\begin{tabular}{|c|l|l|l|l|l|l|}
\hline
\multirow{2}{*}{\textbf{\begin{tabular}[c]{@{}c@{}}Case 2\\ $M<<N$\end{tabular}}} & \multicolumn{2}{l|}{test subject 1} & \multicolumn{2}{l|}{test subject 2} & \multicolumn{2}{l|}{test subject 3} \\ \cline{2-7} 
                                                                                  & Open             & Close            & Open             & Close            & Open             & Close            \\ \hline
\multicolumn{1}{|l|}{Average MSE()}                                               & 15.64            & 19.36            & 345.4            & 13.57            & 315.3            & 63.85            \\ \hline
\begin{tabular}[c]{@{}c@{}}Segments below \\ tol in percent\end{tabular}          & 82.1             & 63.8             & 40.4             & 72.9             & 63.5             & 65.8             \\ \hline
\end{tabular}
\caption{results...}
\label{tab:case_2}
\end{table}

%%%%%%%%%%%%%%%%%%%%%%%%%%%%%%%%%%%%%%%%%%%%%%%%%%%%%%%%% old
%\subsection{Test on EEG Data Set with $\frac{1}{3} M<N$}
%For this test the S1\_Cclean EEG data set will be used and case of interest is when $\frac{1}{3} M < N$, every third sensors is removed from S1\_Cclean EEG data set. The test is performance on 144 time segment with $L_s = 516$ and $M = 18$ sensors. From ICA 144 different $k$ values will represent the activation of sources. The obtain source matrices will only consists of the active sources.
%The aim of this test is to see if one still can recover the same sources as the in the case of $M=N$. It is expected that a higher MSE will be seen.
%
%Figure \ref{fig:M<N_1} visualize the average MSE values obtain from $\hat{\mathbf{X}}_{\text{main}}$ and $\hat{\mathbf{X}}_{\text{ICA}}$, one MSE value per time segment.
%\begin{figure}[H]
%    \begin{minipage}[t]{.45\textwidth}
%		\centering
%		\includegraphics[scale=0.5]{figures/ch_7/AveMSE_3M_N.png}
%	\caption{Average MSE of the main algorithm with $M=18$, $L = 516$ and a different $k$ for each time segment. The main algorithm and ICA is used on the S1\_Cclean EEG data set.}
%	\label{fig:M<N_1}
%    \end{minipage} 
%    \hfill
%    \begin{minipage}[t]{.45\textwidth}
%        \centering
%		\includegraphics[scale=0.5]{figures/ch_7/AveMSE_3M_N_zoom.png}
%	\caption{Average MSE of the main algorithm with $M=18$, $L = 516$ and a different $k$ for each time segment. The main algorithm and ICA is used on the S1\_Cclean EEG data set -- zoom version.}
%	\label{fig:M<N_1_2}
%    \end{minipage}
%\end{figure}
%\noindent 
% 
%\begin{figure}[H]
%\centering
%\includegraphics[scale=0.5]{figures/ch_7/MSE_3M_N.png}
%	\caption{MSE of the main algorithm with $M=18$, $L = 516$ and $k=13$ for time segment $s=45$. The main algorithm and ICA is used on the S1\_Cclean EEG data set.}
%\label{fig:M<N_2}
%\end{figure}
%\noindent
%From figure \ref{fig:M<N_1_2} it is seen that the average MSE of $\hat{\mathbf{X}}_{\text{main}}$ compared to $\hat{\mathbf{X}}_{\text{ICA}}$ mostly laid between 0 and 10 but a mosre higher MSE tendency can be seen compared to the MSE from $M=N$. 
%
%In figure \ref{fig:M<N_2} the MSE values of one time segment is plotted to visualize how the sources performs compared to the ICA. The time segment in interest is $s = 45$ so the segment of the 45 second in S1\_Cclean. 
%From figure \ref{fig:M<N_2} it seen that the estimation of the sources in this time segment have a higher MSE value compared to \ref{fig:M=N_2}. This verifies the expectation that the less sensors a higher MSE values will occur.
%
%The last figure illustrate the $k = 13$ sources from $\hat{\mathbf{X}}_{\text{main}}$ and $\hat{\mathbf{X}}_{\text{ICA}}$ plot on each other.
%\begin{figure}[H]
%    \centering
%	\includegraphics[scale=0.5]{figures/ch_7/Sources_3M_N.png}
%	\caption{Plot of the $k = 13$ sources from $\hat{\mathbf{X}}_{\text{main}}$ and $\hat{\mathbf{X}}_{\text{ICA}}$ from time segment $s = 45$ with $\frac{1}{3} M<N$}
%	\label{fig:M<N_3}
%\end{figure} 
%\noindent
%
%\subsection{Test on EEG Data Set with $\frac{1}{2} M<N$}
%For this test the S1\_Cclean EEG data set will be used and case of interest is when $\frac{1}{2} M < N$, every second sensors is removed from S1\_Cclean EEG data set. The test is performance on 144 time segment with $L_s = 516$ and $M = 13$ sensors. From ICA 144 different $k$ values will represent the activation of sources. The obtain source matrices will only consists of the active sources.
%The aim of this test is the same as the previous, to see if one still can recover the same sources as the in the case of $M=N$. It is still expected to see a higher MSE.
%
%Figure \ref{fig:M<<N_1} visualize the average MSE values obtain from $\hat{\mathbf{X}}_{\text{main}}$ and $\hat{\mathbf{X}}_{\text{ICA}}$, one MSE value per time segment.
%
%\begin{figure}[H]
%    \begin{minipage}[t]{.45\textwidth}
%		\centering
%		\includegraphics[scale=0.5]{figures/ch_7/AveMSE_2M_N.png}
%	\caption{Average MSE of the main algorithm with $M=13$, $L = 516$ and a different $k$ for each time segment. The main algorithm and ICA is used on the S1\_Cclean EEG data set.}
%	\label{fig:M<<N_1}
%    \end{minipage} 
%    \hfill
%    \begin{minipage}[t]{.45\textwidth}
%        \centering
%		\includegraphics[scale=0.5]{figures/ch_7/AveMSE_2M_N_zoom.png}
%	\caption{Average MSE of the main algorithm with $M=13$, $L = 516$ and a different $k$ for each time segment. The main algorithm and ICA is used on the S1\_Cclean EEG data set -- zoom version.}
%	\label{fig:M<<N_1_2}
%    \end{minipage}
%\end{figure}
%\noindent
%
%\begin{figure}[H]
%\centering
%\includegraphics[scale=0.5]{figures/ch_7/MSE_2M_N.png}
%\caption{MSE of the main algorithm with $M=13$, $L = 516$ and $k=13$ for time segment $s=45$. The main algorithm and ICA is used on the S1\_Cclean EEG data set.}
%	\label{fig:M<<N_2}
%\label{fig:M<N_2}
%\end{figure}
%\noindent
%From figure \ref{fig:M<<N_1_2} it is seen that the average MSE of $\hat{\mathbf{X}}_{\text{main}}$ compared to $\hat{\mathbf{X}}_{\text{ICA}}$ mostly laid between 0 and 10. 
%
%In figure \ref{fig:M<<N_2} the MSE values of one time segment is plotted to visualize how the sources performs compared to the ICA. The time segment in interest is $s = 45$ so the segment of the 45 second in S1\_Cclean. 
%From figure \ref{fig:M<<N_2} it seen that the estimation of the sources in this time segment have a generally low MSE value and is almost the same compared to \ref{fig:M=N_2}. 
%
%The last figure illustrate the $k = 13$ sources from $\hat{\mathbf{X}}_{\text{main}}$ and $\hat{\mathbf{X}}_{\text{ICA}}$ plot on each other.
%\begin{figure}[H]
%    \centering
%	\includegraphics[scale=0.5]{figures/ch_7/Sources_2M_N.png}
%	\caption{Plot of the $k = 13$ sources from $\hat{\mathbf{X}}_{\text{main}}$ and $\hat{\mathbf{X}}_{\text{ICA}}$ from time segment $s = 45$ with $\frac{1}{2} M<N$}
%	\label{fig:M<<N_3}
%\end{figure} 
%\noindent


\subsection{Summary of Results}
The main algorithm has been tested on six data sets of EEG measurement, for a varying relation between the number of sensors and sources, case 0, 1 and 2 respectively.
When the number of sensors is reduced with respect to the number of sources to be found, a significant decrease in performance was found. When comparing case 0 and 1. 
However, a corresponding decrease of performance was not found when further sensors was removed when comparing case 1 and 2. 

From the conclusions made in chapter \ref{ch:implementation} it was not expected that the main algorithm would provide successful results, without estimating $\mathbf{A}$ form the data. 
The results of case 0 do however indicate a solid estimate provided by the main algorithm, with an average percentage of successfully estimated segments at $83\%$. 

Furthermore, it is worth to note that the resulting MSE values has potential for improvement when considering optimization of the source localization of the ICA estimate, cf. appendix \ref{app:ica_test}.

\section{Alpha Wave Analysis}
As mentioned in chapter \ref{ch:motivation} the sources are classified into four group according to a dominant frequency.
For the provided EEG datasets the sources would be classified as alpha waves, with frequencies between 8 and 13 Hz.
With the results from the main algorithm it would be interesting to see how the recovered sources and the provided measurements act in the frequency domain, when filtered to be in the range of the alpha wave.

For this comparison the dataset for subject 1, \texttt{S1\_OClean} and \texttt{S1\_CClean}, with open and closed eyes EEG measurements, will be used. 
Furthermore, it is expected to see that the closed-eyes dataset have a higher amplitude in the frequency domain as it is expected that more brain activity occur when you closed your eyes \todo{reference eller noget? -L}.

\subsection{Setup}
To perform the filtering according to the alpha wave frequency range a bandpass Butterworth filter of order 5 with cutoff frequency $8$ Hz and $13$ Hz will be applied to the datasets and recovered sources from the main algorithm. This will be done in the time domain. 
To see the filtering process the datasets and recovered sources will after the filtering be transform into the frequency domain. This will be done with the fast Fourier transform (FFT).

In figure \ref{fig:dft_1} only one source was investigated in both time and frequency domain. The source of interest was recovered from the closed-eyes dataset \texttt{S1\_CClean} dataset from time segment $15$. Furthermore, the system specification used to recover the source was $M=27$ and $k=14$.
\begin{figure}[H]
\centering
\includegraphics[scale=0.28]{figures/ch_7/DFT_plot_X_timeseg15_source10.png}
\caption{Time domain and frequency plot of a recovered source signal, filtered and non-filtered, from the time segment 15.}
\label{fig:dft_1}
\end{figure}
\noindent
The first plot in figure \ref{fig:dft_1} is the recovered source signal in the time domain. The next plot is the same source signal but transformed to the frequency domain with the FFT. The plot has been scaled to only show the frequencies from 0-70 Hz and the power from 0 to 150. The third plot illustrate the frequency response of the bandpass Butterworth filter with order 5. The vertical blue lines illustrate the cutoff frequency for 8 Hz and 13 Hz.
Plot number 4 is the recovered source signal filtered with the bandpass Butterworth filter which have been plot in the time domain. The last plot is the filtered source signal from plot 4 transformed to frequency domain, again with the FFT.

From the frequencies plot it can be seen that the signal of interest has been filtered according to the alpha wave. And from the filtered source signal in the time domain, the signal resemble the alpha wave as seen in figure \ref{fig:EEG_example}.

\subsection{Results}
For the comparison to the alpha wave


The next figure \ref{fig:dft_2} shows filtered source signals and filtered measurement signals, one from the open-eyes dataset \texttt{S1\_OClean} and one from the close-eyes dataset \texttt{S1\_CClean}, from the time segment 15.
\begin{figure}[H]
\centering
\includegraphics[scale=0.28]{figures/ch_7/DFT_plot_X_and_Y_signal_timeseg15_source10.png}
\caption{Filtered signals from measurement 10 and source 10 from time segment 15 from the closed eye dataset \texttt{S1\_CClean}.}
\label{fig:dft_2}
\end{figure}
\noindent
From figure \ref{fig:dft_2} it can be seen that a different between open-eyes and close-eyes datasets exists. From this example the difference from open-eyes to close-eyes measurement signals is $0.59$. The different from open-eyes to close-eyes source signals is $0.24$.
The close-eyes dataset for this example has a higher amplitude just as expected. 

Lets now investigate if this behaviour continues when look at several signals. Figure \ref{fig:dft_3} is a sum of all signals in one time segment from the figure \ref{fig:dft_2} instead of only one signal.
\begin{figure}[H]
\centering
\includegraphics[scale=0.28]{figures/ch_7/DFT_plot_X_and_Y_matrix_timeseg15.png}
\caption{Filtered measurement matrix and source matrix for time segment 15. The rows of the matrices have been sum together.}
\label{fig:dft_3}
\end{figure}
\noindent
The difference from the open-eyes to close-eyes measurement signals is $1.99$. The different from open-eyes to close-eyes source signals is $0.43$. The same behaviour as seen in figure \ref{fig:dft_2} also occur in this case. Furthermore, the summed signals still resemble the alpha wave. 


\begin{figure}[H]
\begin{widepage}
    \begin{minipage}[t]{.49\textwidth}
\centering
\includegraphics[width=1\linewidth]{figures/ch_7/DFT_Y_Difference.png}
\caption{The average difference between the measurements of the open and closed eyes datasets for 100 time segments.}
\label{fig:dft4}
\end{minipage} 
\hspace{.5cm}
\begin{minipage}[t]{.49\textwidth}
\centering
\includegraphics[width=1\linewidth]{figures/ch_7/DFT_X_Difference.png}
\caption{The average difference between the recovered sources of the open and closed eyes datasets for 100 time segments.}
	\label{fig:dft5}
    \end{minipage}
\end{widepage}
\end{figure}
\noindent


