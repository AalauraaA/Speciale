\chapter{Estimation of the Number of Active Sources}\label{ch:estimation_k}
In this chapter the issue of unknown number of active sources $k$ is considered. 
The aim is to investigate the possibility of identifying an estimation of a non-active source signal from $\hat{\textbf{X}}_{\text{main}}$ when the true $k$ is not provided to the main algorithm. 
Instead of providing the true $k$ one let $k = N$.
As such one ask the main algorithm for $N$ active source signals, but there are only $k < N$ active source signals. 
At first the possibilities are investigated on synthetic data set, cf. section \ref{sec:dataset} and afterwards on a real EEG data set.   

\section{Empirical Test on Synthetic Data}
Figure \ref{fig:ktest1} visualizes the estimate $\hat{\mathbf{X}}_{\text{main}}$ given $k=N$ and the true $\textbf{A}$, resulting from a stochastic data set specified by $M = N = 8$, $k = 4$ and $L = 1000$. 
As seen in section \ref{sec:testMsbl_stoch} the case of $M = N$ should be solved almost exact by the M-SBL algorithm with true $\mathbf{A}$ given. 
From the figure it is seen that the estimates of the zero rows have amplitudes close to zero.
This distinguishes them from the remaining source estimates which are seen to be almost exact. 
Due to the estimates of the zero rows being this close to zero they do not affect the MSE. 
Thus, the MSE do not indicate flaws within the estimate. 
Furthermore, it is seen that the estimates of the zero rows form a scaled copy of one of the exact estimates. 
These observations indicate a potential for distinguishing the estimates of zero rows and hence determine $k$.     
\begin{figure}[H]
    \centering
	\includegraphics[scale=0.5]{figures/ch_estimate/k_test1.png}
	\caption{Each plot shows one row of the estimate $\hat{\mathbf{X}}_{\text{main}}$ where $k=N$ and true $\mathbf{A}$ is given, compared to the corresponding true row in $\mathbf{X}$. The MSE is $1.196E-29$. Only samples in the interval $[0,100]$ are visualized.}
	\label{fig:ktest1}
\end{figure}
\noindent
Consider now the desired case where $M < N$.
Figure \ref{fig:ktest3} visualizes the estimate $\hat{\mathbf{X}}_{\text{main}}$ given $k=N$ and the true $\textbf{A}$, resulting from a stochastic data set specified by $M = 6$, $N = 8$, $k = 4$ and $L = 1000$.
From figure \ref{fig:ktest3} it is seen that the estimates of the zero rows is not as close to zero as in figure \ref{fig:ktest1}. 
Thus, this can not be used as the indicator. 
However, the estimates of the zero rows still appears as a scaled replica of an estimate of a non-zero row. 
\begin{figure}[H]
	\centering
	\includegraphics[scale=0.5]{figures/ch_estimate/k_test3.png}
	\caption{Each plot shows one row of the estimate $\hat{\mathbf{X}}_{\text{main}}$ using true $\mathbf{A}$, compared to the corresponding true row in $\mathbf{X}$. The MSE is $0.344$. Only samples in the interval $[0,100]$ are visualized.}
	\label{fig:ktest3}
\end{figure}
\noindent
A replica in this case is not considered an exact copy but a signal with similar trends over time.
One attempt to locate the zero rows is to compare each row of $\hat{\mathbf{X}}_{\text{main}}$ to every other row by the MSE, in order to check if it appears more than one time.
Two rows are considered replicas if their mutual MSE is below a tolerance equal to 1. 
This operation is performed on the estimated source matrix $\hat{\mathbf{X}}_{\text{main}}$ visualized in figure \ref{fig:ktest3}.
The operation gives the results displayed in table \ref{tab:replica1}.
From table \ref{tab:replica1} it is seen that source signals of row 2, 4, 6 and 8 are found to appear more than one time. These row indexes correspond to the zero rows of $\mathbf{X}$ as intended. 
This indicates the possibility of locating the zero rows from the estimate $\hat{\mathbf{X}}_{\text{main}}$ without providing the true $k$ as an input.  
\begin{table}[H]
\center
\begin{tabular}{|l|l|l|l|l|l|l|l|l|}
\hline
Row index   & 1 & 2 & 3 & 4 & 5 & 6 & 7 & 8 \\ \hline
\# replicas & 1 & 3 & 1 & 2 & 1 & 4 & 1 & 3 \\ \hline
\end{tabular}
\caption{Number of replicas for each row in $\hat{\mathbf{X}}_{\text{main}}$ of figure \ref{fig:ktest3} based on the tolerance MSE $< 1$. }
\label{tab:replica1}
\end{table}
\noindent
It is expected that the precision must depend on the chosen tolerance for the mutual MSE. 
For comparison table \ref{tab:replica2}, \ref{tab:replica3} and \ref{tab:replica4} show the result from a tolerance of $0.5$, $1.5$ and $2$ respectively. 
It is observed that a tolerance of $0.5$ and $2$ results in a different conclusion with respect to the number of zero rows -- being respectively 2 and 6. 
From this it is clear that the tolerance is difficult to define and will affect the conclusion of the results.  
\begin{table}[H]
\center
\begin{tabular}{|l|l|l|l|l|l|l|l|l|}
\hline
Row index   & 1 & 2 & 3 & 4 & 5 & 6 & 7 & 8 \\ \hline
\# replicas & 1 & 1 & 1 & 1 & 1 & 2 & 1 & 2 \\ \hline
\end{tabular}
\caption{Number of replicas for each row in $\hat{\mathbf{X}}_{\text{main}}$ of figure \ref{fig:ktest3} based on the tolerance MSE $< 0.5$.}
\label{tab:replica2}
\end{table}
\noindent
\begin{table}[H]
\center
\begin{tabular}{|l|l|l|l|l|l|l|l|l|}
\hline
Row index   & 1 & 2 & 3 & 4 & 5 & 6 & 7 & 8 \\ \hline
\# replicas & 1 & 4 & 1 & 3 & 1 & 4 & 1 & 3 \\ \hline
\end{tabular}
\caption{Number of replicas for each row in $\hat{\mathbf{X}}_{\text{main}}$ of figure \ref{fig:ktest3} based on the tolerance MSE $< 1.5$.}
\label{tab:replica3}
\end{table}
\noindent
\begin{table}[H]
\center
\begin{tabular}{|l|l|l|l|l|l|l|l|l|}
\hline
Row index   & 1 & 2 & 3 & 4 & 5 & 6 & 7 & 8 \\ \hline
\# replicas & 2 & 6 & 1 & 4 & 2 & 4 & 1 & 4 \\ \hline
\end{tabular}
\caption{Number of replicas for each row in $\hat{\mathbf{X}}_{\text{main}}$ of figure \ref{fig:ktest3} based on the tolerance MSE $< 2$.}
\label{tab:replica4}
\end{table}
\noindent
The results so far have relied on the true mixing matrix $\mathbf{A}$ given as an input to the main algorithm, due to the conclusion of chapter \ref{ch:implementation} where the estimate of $\mathbf{A}$ is abandoned. 
Thus, the results is conditioned on an exact estimate of $\mathbf{A}$ which this thesis does not manage to provide.

Now the investigations are repeated but with use of the main algorithm utilizing $\hat{\mathbf{A}}_{\text{fix}}$, cf. section \ref{sec:test_base}.
Figure \ref{fig:ktest5} shows the estimate $\hat{\mathbf{X}}_{\text{main}}$ given $k=N$ and $\hat{\textbf{A}}_{\text{fix}}$, resulting from a stochastic data set specified by $M = 6$, $N = 8$, $k = 4$ and $L = 1000$.
As expected, according to the results from section \ref{sec:Main_test}, it is generally seen from figure \ref{fig:ktest5} that every row of the estimate is less accurate as the result is based on $\hat{\mathbf{A}}_{\text{fix}}$ instead of the true $\mathbf{A}$. 
\begin{figure}[H]
\centering
\includegraphics[scale=0.5]{figures/ch_estimate/k_test5.png}
\caption{Each plot shows one row of the estimate $\hat{\mathbf{X}}_{\text{main}}$ using $\mathbf{A}_{\text{norm2}}$, compared to the corresponding true row in $\mathbf{X}$. The MSE is $128.7$. Only samples in the interval $[0,100]$ are visualized.}
\label{fig:ktest5}
\end{figure}
\noindent
Table \ref{tab:replica5} shows the corresponding replica count with an MSE tolerance at 1. 
From the table it is seen that 7 out of the 8 rows are zero rows, while the true number is 4 rows. 
This could indicate that the tolerance is set to high. 
Table \ref{tab:replica6} show the replica count for an MSE tolerance at 0.5. 
From table \ref{tab:replica6} it is seen that the number of replicas is reduced.
However, it still does not results in the right number of zero rows.  
\begin{table}[H]
\center
\begin{tabular}{|l|l|l|l|l|l|l|l|l|}
\hline
Row index   & 1 & 2 & 3 & 4 & 5 & 6 & 7 & 8 \\ \hline
\# replicas & 3 & 5 & 2 & 2 & 4 & 1 & 4 & 5 \\ \hline
\end{tabular}
\caption{Number of replicas for each row in $\hat{\mathbf{X}}_{\text{main}}$ of figure \ref{fig:ktest5} based on the tolerance MSE $< 1$.}
\label{tab:replica5}
\end{table}
\noindent
\begin{table}[H]
\center
\begin{tabular}{|l|l|l|l|l|l|l|l|l|}
\hline
Row index   & 1 & 2 & 3 & 4 & 5 & 6 & 7 & 8 \\ \hline
\# replicas & 2 & 2 & 2 & 2 & 1 & 1 & 1 & 3 \\ \hline
\end{tabular}
\caption{Number of replicas for each row in $\hat{\mathbf{X}}_{\text{main}}$ of figure \ref{fig:ktest5} based on the tolerance MSE $< 0.5$.}
\label{tab:replica6}
\end{table}
\noindent
From the observations made through this investigation, based on synthetic data, the following conclusions are made.
From figure \ref{fig:ktest3} and table \ref{tab:replica1} a potential is found with respect to identifying the zero rows within the estimate, implying the desired estimate of $k$ -- conditioned by an exact estimate of $\textbf{A}$. 
Here the zero rows are identified as the rows of the estimate for which similar signals appear in other rows indicating that no new estimate has been computed. 
From figure \ref{fig:ktest5} and tables \ref{tab:replica5} and \ref{tab:replica6} $\hat{\mathbf{A}}_{\text{fix}}$ is utilized in the main algorithm, as it will be when applied to real EEG data. 
Here it has not been possible to identify the zero rows correctly, based on the replica count. 
Thus, it must be concluded that the method is not reliable when the estimate is computed by the main algorithm. 
However, it is essential that a potential was found under ideal conditions.

To finish the investigation, the replica count method has been applied to the estimation of real EEG measurements. This is done due to the possibility of seeing a different behavior from the real EEG measurements compared to the synthetic data. 

\section{Real EEG Measurements}
Let's look at estimation of active sources, $k$, on the real EEG measurements. 
For this estimation one can not compare the estimation to the real sources as in the previous section.
Instead, one applies the estimation tool, looking for replicas, on the real EEG measurements and makes a conclusion from the observed result.

For the estimation one will only look at one time segment to verify that the estimation of active sources can be applied on real EEG measurements. The estimation will be performed on the S1\_Cclean EEG data set with $\frac{1}{2} M < N$ -- every second sensor is removed. The specification on this data set are $M=13$ and the time segment of interest is $s = 10$. 
For the estimation one must give an initial guess for the number of active sources. Remember that $k>M$ and $k<N$.  
$k = 17$ has been chosen as a choice for how many active sources exists in time segment $s = 10$. 

Figure \ref{fig:eeg_k} visualize the recovered source matrix $\hat{\mathbf{X}}$ from time segment $s=10$ for $M=13$ and $k = 17$.
\begin{figure}[H]
    \centering
	\includegraphics[scale=0.5]{figures/ch_estimate/eeg_k_timeseg_10.png}
	\caption{The recovered source matrix $\hat{\mathbf{X}}$ with $k=17$ active source for the EEG data set with $M=13$.}
	\label{fig:eeg_k}
\end{figure}
\noindent
From figure \ref{fig:eeg_k} one see that all $k$ active sources are visible and there seem not to be any visible indication of a active source being non-active. 

From the list of replicates from figure \ref{fig:eeg_k} only four sources are the 'real' active source while the rest are replicates of those four. The 'real' active sources can be found in row 2, 5, 16 and 17. This leads to the conclusion that for time segment $s=10$ with system specification $M=13$ for the S1\_Cclean EEG data set only have $k=4$ active sources.



