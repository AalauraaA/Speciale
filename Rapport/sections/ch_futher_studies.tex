\chapter{Further Studies}
In this chapter some further studies regarding this thesis and the implemented main algorithm will be discussed.

As described in chapter \ref{ch:implementation} the main algorithm do not manage to successfully recover the mixing matrix and source matrix from the simulated data sets. And by that lead to a non successful recovering of unknown sources from the real EEG measurements. 
It was discussed in chapter \ref{ch:discussion} that in Cov-DL the optimisation process did not success.
This could be further investigated in form of writing the optimisation problem differently or chose a different optimisation process. 
Another aspect discussed in \ref{ch:discussion}, was that the methods behind the main algorithm was unreproducible and therefore was not the best fit for this thesis to seek in recovering sources. 
To make the main algorithm more successful an alternative method to finding the mixing matrix must be researched. Such method could be the low resolution electrical tomograph (LORETA) which localises electrical activity inside the brain or the minimum norm estimates (MNE) which reconstruct the activity on the cortical surface \cite{??}.

Another point of interest could be to alter the view on the EEG measurements. In this thesis the purpose was to identify and localised active sources from the EEG measurements with the main algorithm. 
The article, \cite{??}, present an alternative method to make used of the recovered information from the EEG measurement. They suggested that by focusing on recovering of eye movements, a directional beam could be generated and moved towards the speaking of interest. That is a directional beam influence by the recovered eye activity from EEG measurements.
As this thesis only is interesting in finding all sources from a provided EEG measurement data base, this could be a change of focus. Instead, the focus will be to look at the EEG measurement produced by eye movements. For this case another EEG measurements data base must be provided or created as the one used in chapter \ref{ch:eeg_test} is without the EEG measurements constructed from the eye movement and surrounding muscle movements.

%A last thing that could interesting to work further on is to make the main algorithm into a real-time application such as this could a useful algorithm within the hearing aids industry. 
%One of the issue with this kind of application would be to make it low in time consumption as it would not be effective in short-time conversations if the it take 1 minutes to read and analyse the EEG measurements.

