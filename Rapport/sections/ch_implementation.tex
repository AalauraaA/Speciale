\chapter{Baseline Algorithm}\label{ch:implementation}
From the start of this thesis to the start of this chapter a motivation behind investigation the identification of sources from EEG measurement with more sources than sensors as been exploit. From a linear multiple measurement vector (MMV) model
\begin{align*}
\mathbf{Y} = \mathbf{AX},
\end{align*}
theory and methods behind recovering the mixing matrix $\mathbf{A}$ and the source matrix $\mathbf{X}$ have been investigated and researched. This lead to two algorithms -- covariance-domain dictionary learning (Cov-DL) algorithm and multiple sparse Bayesian learning (M-SBL) algorithm -- which recover a mixing matrix and a source matrix from the EEG measurements with more sources than sensors, $N > M$.

The goal of this chapter is to describe how the two found algorithms can be implemented into one algorithm -- the baseline algorithm -- and what one must have in mind when combining the two algorithms. Furthermore, some tests of the baseline algorithm, with different data sets of different type of data, will be performed with the purpose to investigate how good the recovering process is. 

The chapter will begin with a discussion of the choice for the parameters $M$, $N$ and $k$ as in the realistic case with real EEG measurements the number of active sources within the brain $k$ are unknown -- which parameters lead to a good recovering?
\todo[inline]{lyder det ikke lidt som som vi skal vælge de bedste N,M og k her?, vi skal lige have styr på hvad vi kalder parameter og variable osv. overordnet så er $\textbf{X}$ vel de model parameter vi skal finde, mens A-init og cov-seg samt N og k er variable vi kan justere, mens M variable vi ikke kan justere?}

\section{Parameter Choice}
\begin{enumerate}
\item Describe why we want to set $N = k$, logical discussion. No results yet to back up the assuming.
	\begin{itemize}
	\item The amount of $N$ is unknown as it change for every brain
	\item Therefore, we do not know the amount of activation inside the brain, but it those there are of interest in this thesis.
	\end{itemize}
\item Talk about the parameter choice for $M$ and $N$ -- a low-density system have what? $M =  8$ and $N = 16$ or $N = 32$.
\end{enumerate}

   
\section{Tests}
\begin{enumerate}
\item Introduction to the section. What do we want and why. First we want to test A and the X. At last we want to test the complete system.
\item Introduce how we will compare the tests and how we measure the quality -- Mean Squared Error (MSE).
\item Subsection with the error measure
\item Subsection with a introduction to the data sets -- AR and Toy Example data sets.
\item Subsection with the performing of tests
	\begin{itemize}
	\item Talk about the choice for a initial A in the Cov-DL when D is under-complete. Random uniform distribution? or a Gaussian distribution. Show the errors of the real A and estimated A.
	\item With the choice for A we now tests the choice for the number of segmentation in Cov-DL.
	\item Test X with the found A. Make a comparison of the error for X and A in one plot.
	\end{itemize}
\end{enumerate}
   
\subsection{Error}
To evaluate our baseline algorithm we will perform some test s with knowledge of some real mixing matrices $\mathbf{A}$ and source matrices $\mathbf{X}$ to be compared with our estimated mixing matrices $\hat{\mathbf{A}}$ and source matrices $\hat{\mathbf{X}}$. For the comparison we will look at the differences between the real and estimated matrices. For this task a mean squared error (MSE) method has been chosen. The MSE measure the average squared difference between some estimated value and the actual value. The MSE has the property of always being positive because of the randomness introduced in the method.
The MSE is given by
\begin{align*}
\text{MSE} = \frac{1}{T} \sum_{i=1}^T (G_i - \hat{G}_i)^2,  
\end{align*}
where $T = \{m,n\}$ the numbers of rows of $\mathbf{A}$ or $\mathbf{X}$, $G_i = \{ \mathbf{A}_i, \mathbf{X}_i\}$ is a measurement row of the actual matrix and $\hat{G}_i = \{\hat{\mathbf{A}}_i,\hat{\mathbf{X}}_i\}$ is an estimated row of the estimated matrix.
The MSE is view as a measure of the quality of an estimator, in this case of how M-SBL and Cov-DL perform. 
If the MSE is a large value this implies that the the estimated data values are dispersed widely around its mean while a small MSE value implies that the estimated data is closely dispersed around the mean. Usually, a small MSE value indicate a good estimator but the value cannot be to small as this would indicate that the data has been overfitted. 
A good MSE would be depending on how the data is scattered as wildely scattered data may lead to a MSE not close to zero but it would still be the a good measure for the estimator.


\subsection{Simulation of Data Sets}


\subsubsection{Toy Example}
The first data set which also is the simple and non-realistic data set is used for confirmation that the two presented algorithm work.
The data set is constructed from four different signals: 
\begin{itemize}
\item[-] $\sin(2t)$
\item[-] $\sin(4t)$
\item[-] a sawtooth wave with period $2 \pi t$
\item[-] a sign signal of $\sin(3t)$
\end{itemize}
with $t$ being a time index defined in the interval $[0,10]$ with $L$ samples. Each of the four signal are randomly drawn and used to construct a source matrix $\mathbf{X}$ of size $k \times L$. 
With a source matrix $\mathbf{X}$ a mixing matrix $\mathbf{A}$ of size $M \times k$ is randomly generated from a Gaussian distribution and then normalised. By multiplying the source matrix and the mixing matrix the measurement matrix $\mathbf{Y}$ is achieved and this complete the toy example data set.

An illustration of the data set can be seen in figure \ref{fig:mix}. The data set is constructed for $M = 3$, $k = 4$ and $L = 1000$.
\begin{figure}[H]
\centering
\includegraphics[scale=0.5]{figures/chapter6/Mix_Data_m3_n4_k4_L1000.png}
\label{fig:mix}
\caption{All the signal of the toy example data set for $M = 3$, $k=4$ and $L=1000$.}
\end{figure}
\noindent

\subsubsection{Autoregressive}
The second data set illustrate a more realistic data set.
The data set is constructed from different autoregressive processes: 
\begin{itemize}
\item[-] $\mathbf{x}^{t} = \mathbf{a1}^{t-1} \cdot \mathbf{x}^{t-1} + \mathbf{a1}^{t-2} \cdot \mathbf{x}^{t-2} + \mathbf{w1}_t$
\item[-] $\mathbf{x}^{t+1} = \mathbf{a2}_t \cdot \mathbf{x}_{t-1} + \mathbf{a}_t \cdot \mathbf{x}_t + \mathbf{w}_t$
\item[-] $\mathbf{x}_{t+1} = \mathbf{a}_t \cdot \mathbf{x}_{t-2} + \mathbf{a}_t \cdot \mathbf{x}_t + + \mathbf{a}_t \cdot \mathbf{x}_{t-1} \mathbf{w}_t$
\item[-] $\mathbf{x}_{t+1} = \mathbf{a}_t \cdot \mathbf{x}_t + \mathbf{a}_t \cdot \mathbf{x}_{t-3} + \mathbf{w}_t$
\end{itemize}
with $t$ being a time index defined in the interval $[0,10]$ with $L$ samples. Each of the four signal are randomly drawn and used to construct a source matrix $\mathbf{X}$ of size $k \times L$. 
With a source matrix $\mathbf{X}$ a mixing matrix $\mathbf{A}$ of size $M \times k$ is randomly generated from a Gaussian distribution and then normalised. By multiplying the source matrix and the mixing matrix the measurement matrix $\mathbf{Y}$ is achieved and this complete the toy example data set.

An illustration of the data set can be seen in figure \ref{fig:mix}. The data set is constructed for $M = 3$, $k = 4$ and $L = 1000$.
\begin{figure}[H]
\centering
\includegraphics[scale=0.5]{figures/chapter6/AR_Data_m3_n4_k4_L1000.png}
\label{fig:mix}
\caption{All the signal of the toy example data set for $M = 3$, $k=4$ and $L=1000$.}
\end{figure}
\noindent



\subsection{Tests}

\section{Algorithm Implementation}


\begin{enumerate}
\item Introduction to the section
\item Insert the flow diagram and describe how the two algorithm are connected.
\item Perhaps end the section with some description of the complexity of the code.
\end{enumerate}

\begin{figure}[H]
\centering
\includegraphics[scale=0.8]{figures/ch_6/baseline_flowchart.png}
\label{fig:flow}
\caption{Flowchart illustrating the implementation of the baseline algorithm. Remember that $\widetilde{M}=\frac{M(M+1)}{2}$.}
\end{figure}
\noindent


\section{Test of the Baseline Algorithm}\label{sec:test_base}
In this section the performance of the final baseline algorithm is tested.  
 
\subsection{Alternative to Estimate $\hat{\textbf{A}}$}  
As concluded the Cov-DL algorithm do not recover a sufficient estimate of the mixing matrix $\mathbf{A}$, therefore a different approach is necessary. 

Replacing the insufficient estimate by a fixed estimate $\hat{\mathbf{A}}_{\text{fix}}$ is one immediately solution. 
This choice is supported by the observations from Cov-DL2 where $\mathbf{A}_{\text{ini}}$ matrix provides an estimate which is happens to be a least as good as the one provided by Cov-DL. 
Thus the challenge is now to determine a fixed matrix for which its characteristics resembles those of the true mixing matrix. 
However, from chapter \ref{ch:motivation} it is clear that no specific characteristic of the mixing matrix is known, which supports the choice of an random matrix of Gaussian distribution or similar, as it was chosen for the initial guess $\mathbf{A}_{\text{ini}}$ for the estimate.   
From this perspective three fixed mixing matrices are defined, by drawing each entry from a specified distribution: 
\begin{itemize}
\item[] $\hat{\mathbf{A}}_{\text{uni}} \sim \mathcal{U}(-1,1)$
\item[] $\hat{\mathbf{A}}_{\text{norm}} \sim \mathcal{N}(0, 2)$ 
\item[] $\hat{\mathbf{A}}_{\text{gauss}} \sim \mathcal{N}(0,1)$                                           
\end{itemize}
<<<<<<< HEAD

Note that the third matrix $\hat{\textbf{A}}_{gauss}$ is generated the same way as the true mixing matrix of the stochastic data sets. thus it is expected to have the lowest MSE when compared to the true $\textbf{A}$. However, it is of interest to investigate whether is the best estimate of $\textbf{A}$ which provide the best estimate of $\textbf{X}$.   

A different option regarding a fixed estiamtion of $\textbf{A}$ is to utilize the ICA algorithm, described in appendix \ref{app:ICA}. By the ICA algorithm it is possible to solve the EEG inverse problem for both $\textbf{A}$ and $\textbf{X}$, in the case where $k \leq M$.
Consider a simulation of a stochastic data set specified by $N = k = M$. Solving the system by ICA yields an estimate of $\textbf{A}$. Now reduce the data set $\textbf{Y}$ such that $M \leq k$. Similar the estimate of $\textbf{A}$ is reduced by removing the same rows as in $\textbf{Y}$, this yields the an estimate $\hat{\textbf{A}}_{ICA}$ which can be used as a fixed input to M-SBL along with the corresponding reduced $\textbf{Y}$.

The four different fixed estimates $\hat{\mathbf{A}}$ are tested on stochastic data sets specified by $M = 10$, $N = k = 16$ and $L = 1000$. 
To get the average performance 50 different simulations are conducted with the same specifications, each system $\textbf{X}$ is estimated from each of the four fixed estimates of $\textbf{A}$\footnote{note that for each of the $50$ repetitions  four new $\hat{\textbf{A}}_{\text{fix}}$ are fixed}, and the MSE is are computed. The resulting averaged $MSE_{\textbf{A}}$ and $MSE_{\textbf{X}}$ are visualised  in figure \ref{fig:vary_A}, for each of the four $\hat{\mathbf{A}}_{\text{fix}}$. 
=======
Note that the third matrix $\hat{\mathbf{A}}_{\text{gauss}}$ is generated the same way as the true mixing matrix of the stochastic data sets. 
Thus it is expected to have the lowest MSE when compared to the true mixing matrix $\mathbf{A}$. 
However, it is of interest to investigate whether it is the best estimate of $\mathbf{A}$ which provide the best estimate of $\mathbf{X}$.   

A different option regarding a choice for a fixed $\hat{\mathbf{A}}$ is to utilize the ICA algorithm, described in appendix \ref{app:ICA}. 
By the ICA algorithm it is possible to solve the EEG inverse problem for both $\mathbf{A}$ and $\mathbf{X}$, in the case where $k \leq M$.
Consider a simulation of a stochastic data set specified by $N = k = M$. 
Solving the system by ICA yields an estimate of $\mathbf{A}$. 
Now reduce the data set $\mathbf{Y}$ such that $M \leq k$. 
Similar the estimate of $\mathbf{A}$ is reduced by removing the same rows as in $\mathbf{Y}$, this yields the an estimate $\hat{\mathbf{A}}_{\text{ICA}}$ which can be used as a fixed input to M-SBL along with the corresponding reduced $\mathbf{Y}$.

The four different fixed estimates $\hat{\mathbf{A}}$ are tested on stochastic data sets specified by $M = 10$, $N = k = 16$ and $L = 1000$, where the estimate $\hat{\mathbf{A}}_{\text{ICA}}$ has been reduced to $M = 10$.
To get an average performance 50 different simulations are conducted with the same specifications, each system $\mathbf{X}$ is estimated from each of the four fixed estimates of $\mathbf{A}$\footnote{Note that for each of the 50 repetitions four new $\hat{\mathbf{A}}_{\text{fix}}$ are fixed.}, and the MSE are computed. 
The resulting averaged $\text{MSE}(\mathbf{A}, \hat{\mathbf{A}}_{\text{fix}})$ and $\text{MSE}(\mathbf{X}, \hat{\mathbf{X}})$ are visualised in figure \ref{fig:vary_A}, for each of the four $\hat{\mathbf{A}}_{\text{fix}}$. 
>>>>>>> f42aa842a82c9b66210ce7b6f62d5a2b6b2542e6
Furthermore, the plotted values are found in table \ref{tab:fixed}.
\begin{figure}[H]
\centering
\includegraphics[scale=0.5]{figures/ch_6/A_fix.png}
\caption{Average MSE values for each of the four fixed mixing matrix $\hat{\mathbf{A}}_{\text{fix}}$ resulting from a stochastic data set specified by $M=10$, $N=k=16$ and $L=1000$.}
\label{fig:vary_A}
\end{figure}
\noindent

\begin{table}[H]
\centering
\begin{tabular}{|c|c|c|c|c|}
\hline
 & $\hat{\mathbf{A}}_{\text{uni}}$ & $\hat{\mathbf{A}}_{\text{norm}}$	 & $\hat{\mathbf{A}}_{\text{gauss}}$ & $\hat{\mathbf{A}}_{\text{ICA}}$ \\
\hline
<<<<<<< HEAD
MSE$_\mathbf{A}$ & 1.314 & 5.021 & 1.935 & 1.033 \\
\hline
MSE$_\mathbf{X}$ & 70.74 & 17.36 & 40.58 & 231.4 \\
=======
$\text{MSE}(\mathbf{A}, \hat{\mathbf{A}}_{\text{fix}})$ & 1.330 & 4.907 & 1.976 & 1.060 \\
\hline
$\text{MSE}(\mathbf{X}, \hat{\mathbf{X}})$ & 89.70 & 21.33 & 35.93 & 148.0 \\
>>>>>>> f42aa842a82c9b66210ce7b6f62d5a2b6b2542e6
\hline
\end{tabular}
\caption{Average MSE values resulting from stochastic data set specified by $M=10$, $N=k=16$ and $L=1000$ with a fixed estimate of the mixing matrix $\hat{\mathbf{A}}_{\text{fix}}$.}
\label{tab:fixed}
\end{table}
\noindent
From table \ref{tab:fixed} and figure \ref{fig:vary_A} it is first of all seen that relation between the MSE of $\textbf{A}$ and $\textbf{X}$ is not as expected, as the lowest $\text{MSE}(\mathbf{A}, \hat{\mathbf{A}}_{\text{fix}})$ results in the highest $\text{MSE}(\mathbf{X}, \hat{\mathbf{X}})$ and so forth. 
The lowest $\text{MSE}(\mathbf{A}, \hat{\mathbf{A}}_{\text{fix}})$ is achieved by using $\hat{\mathbf{A}}_{\text{ICA}}$, which confirms that the ICA algorithm manage to estimate $\mathbf{A}$ when $k \leq M$. 
However, as this do not result in the best estimate of $\mathbf{X}$ a different choice of $\hat{\mathbf{A}}$ is still considered. 
The lowest $\text{MSE}(\mathbf{X}, \hat{\mathbf{X}})$ is achieved by use of $\hat{\mathbf{A}}_{\text{norm}}$, which resulted in the largest $\text{MSE}(\mathbf{A}, \hat{\mathbf{A}}_{\text{fix}})$. 
      
As the main interest in this thesis is to identify and localize the active sources of EEG measurements a low $\text{MSE}(\mathbf{X}, \hat{\mathbf{X}})$ is more desirable than a low $\text{MSE}(\mathbf{A}, \hat{\mathbf{A}}_{\text{fix}})$. 
Furthermore, a disadvantage of using $\hat{\mathbf{A}}_{\text{ICA}}$ is the limitations in practice when $k = M$ is not possible.     
From these observation a fixed estimate of the mixing matrix drawn from a normal distribution with mean 0 and variance 2, $\hat{\mathbf{A}}_{\text{norm}}$, is chosen as the the alternative estimate of $\mathbf{A}$, and is used throughout the thesis.   

\subsection{Performance Test of Final Baseline Algorithm}
In order to evaluate the performance of the resulting baseline algorithm tests are conducted on several simulated stochastic data sets with different specification. The aim is to see how the relationship between $N$ and $M$ affect the performance, in other words how robust the algorithm is towards low density measurements. 
The baseline algorithm is tested on simulated stochastic data sets specified by $M=8$, $L=1000$, $k=N$ with $N$ in the range $N = [M+1,\hdots,36]$, as such $k<\widetilde{M}$ are withhold insuring a solution.
For each value of $N$ 10 different data sets are simulated and solved, and the average $\textbf{X}_{MSE}$ are used as the result. 
The results are plotted in figure \ref{fig:varyN1}.
      
\begin{figure}[H]
    \centering
	\includegraphics[scale=0.5]{figures/ch_6/varyN1.png}
	\caption{ plot of $\textbf{X}_{MSE}$ when the baseline algorithm is use on simulated stochastic data sets specified by $M = 8$, $L=1000$ and $k = N$ for $N = M+1, \hdots , 36$. Average over 10 repetitions for each $N$.}
	\label{fig:varyN1}
\end{figure}

From figure \ref{fig:varyN1} it is seen that the $\textbf{X}_{MSE}$ lies in an interval from 4 to 14. However no clear trend appears in the plot. This suggest first that it is not an representative average which have been plotted, thus the test is repeated with more repetitions for value of $N$. The new result is seen in figure \ref{fig:varyN2}

\begin{figure}[H]
    \centering
	\includegraphics[scale=0.5]{figures/ch_6/varyN2.png}
	\caption{Plot of $\textbf{X}_{MSE}$ when the baseline algorithm is use on simulated stochastic data sets specified by $M = 8$, $L=1000$ and $k = N$ for $N = M+1, \hdots , 36$. Average over 500 repetitions for each $N$.}
	\label{fig:varyN2}
\end{figure}  
Figure \ref{fig:varyN2} confirms the result of the first test. Thus it must be that average behaviour which is seen. This suggest that the performance of the baseline algorithm is not affected the relation between $M$ and $N$.
However, this assumption is counter intuitive and it is a contradiction to the results seen in figure \ref{fig:AR1} and \ref{fig:AR2}, where the true $\textbf{A}$ was utilised. Thus the choice of the alternative estimate $\hat{\textbf{A}}_{norm}$ might have influenced the results negatively.  
Furthermore it is worth to notice the relative large interval of the $\textbf{X}_{MSE}$ suggesting a vary high variance within the resulting $\textbf{X}_{MSE}$, which add a certain unreliability to the results.    






\section{Conclusion}
Opsumering/konklusion af kapitel -- henvisning/led hen til test af rigtig data


