\section{Conclusion}
Through this chapter the implementation process has been described, followed by verification tests of the two main elements of the baseline algorithm, respectively the COV-DL algorithm and the M-SBL algorithm. 
From the test of M-SBL on stochastic data sets it was verified that the algorithm provide the expected output and from $\textbf{X}_{MSE}$, and the corresponding visual comparison, the estimate was found to be sufficient. 
The verification of M-SBL was conditioned on the true mixing matrix $\textbf{A}$ as input, to not let the precision of the estimate $\hat{\textbf{A}}$ affect the results.  
Furthermore the possibilities of letting $k = N$ was discussed. Either $N$ nor $k$ is known in practise, but one has to provide the best guess for both $N$ and $k$ to the algorithm in order to provide corresponding number of source signals. By letting $k=N$ one only has to guess the maximal number of active sources and not the relation between active and non active sources, which is considered easier.  
Considering the consequences within the M-SBL, letting $k=N$ will reduce the chance of dislocation, which is seen as an advantage. Furthermore test on the deterministic data confirmed that the estimated active sources was not degraded. Thus it is confirmed that letting $k = N$ is sufficient, and will be used when test the algorithm on real EEG measurements.               

From the verification test of COV-DL, providing the estimate $\hat{\textbf{A}}$, it was that COV-DL did not manage to provide a sufficient result. 
Is was confirmed that the COV-DL resulted in the expected output relative to the implementation, but the output did not comply with the theoretically expected result. 
Thus it is concluded that the theory provided by source \cite[phd2015] was misinterpreted, suggesting partly that the degree of reproducibility of the paper have not been sufficient. 
Due to the time scope of the thesis this issue is not investigated further. 
However, as the estimate of $\textbf{A}$ resulting from COV-DL is crucial in order to estimate the sources signal from real EEG data, it was chosen that the best possible alternative to original estimate must be used, in order to pursue the remaining elements of the thesis. Then, the missing estimate must be taking into account when evaluating the final results.
Different suggestions for an alternative estimate of $\textbf{A}$ was proposed an evaluated by the resulting $\textbf{X}_{MSE}$. Here the it was found that fixed estimated $\hat{\textbf{A}}_{norm}$ generated from a normal distribution with mean $= 0$ and variance $=2$ provided the best result, when tested on stochastic data sets resembling real EEG measurements.          
      
Lastly the performance of the final baseline algorithm was tested on stochastic data sets. Here tests were performed on varying N in order to investigate performance relative to the relation between M and N. For each value of N repetitions was conducted and the average $\textbf{X}_{MSE}$ was evaluated. The $\textbf{X}_{MSE}$ was found to lie within an interval from 2 to 25, without any characteristic trend relative to the increasing N. From this is it concluded that the performance do not rely on the relation between N and M. Despite that this was indicated by the tests where the true $\textbf{A}$ was utilised. 
Thus the lack of a precise estimate of $\textbf{A}$ might influence the final results. 

Overall, the implementation of the baseline algorithm is approved. However the performance is not as good as expected. From this the baseline algorithm is ready  to be tested on real EEG measurements in order to evaluated the performance with respect to the problem statement of this thesis. This test is specified and conducted in the next chapter.           


