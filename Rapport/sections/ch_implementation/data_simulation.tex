\section{Data Simulation}\label{sec:dataset}
To test the performance of the algorithm simulated data, corresponding to the model $\textbf{Y}=\textbf{A}\textbf{X}$, is needed. All data sets are simulated based on the following approach, satisfying the sufficient conditions for recovery, displayed in theorem \ref{th:conditions}.
 
A source matrix $\mathbf{X} \in \mathbb{R}^{N \times L}$ is constructed, such that each row makes an independent signal which varies over $L$ time samples or a zero row. As such the non-zero rows of $\mathbf{X}$ are mutually orthogonal, which fulfils the first conditions of theorem \ref{th:conditions}.   
Then a mixing matrix $\mathbf{A} \in \mathbb{R}^{M \times N}$ is constructed with equally distributed and independent entries. As such the source signals are randomly mixed and the mixing matrix fulfils the second condition of theorem \ref{th:conditions}.
With known $\mathbf{A}$ and $\mathbf{X}$ the measurement matrix $\mathbf{Y} \in \mathbb{R}^{M \times L}$ is simulated according to the model, by the matrix product $\mathbf{Y} = \mathbf{AX}$.  

Two different kinds of data sets are simulated.
One deterministic data set with simple and predictable source signals to ensure a solution and easy visualization.
Another stochastic data set with randomised and fluctuating source signals to resemble realistic EEG measurements.

\subsection{Deterministic Data Set}\label{subseg_simpledata}
Two different deterministic data sets are simulated, with a different number of zero rows. 
One specified by $N = 5$, $k = 4$, $M = 3$ and $L = 1000$. That is a source matrix $\mathbf{X}$ with $4$ independent signals and $1$ zero row which is mixed into a data set with $3$ measurement per sample.  
The second deterministic data set is specified by $N = 8$, $k = 4$, $M = 3$ and $L = 1000$. This is 3 additional zero rows.
From the specifications the first data set comply to $N \leq \frac{M(M+1)}{2}$ which imply the use of COV-DL2.
The second data set comply to $N > \frac{M(M+1)}{2}$ and $k \leq \frac{M(M+1)}{2}$ implying the use of COV-DL1. As such it is possible to test both branches of the COV-DL algorithm. 
     
The 4 independent source signals of $\mathbf{X}$ are defined by 
\begin{itemize}
\item[1.] a sinus signal $\sin(2t)$
\item[2.] a sawtooth signal with period $2 \pi t$
\item[3.] a sinus signal $\sin(4t)$
\item[4.] a sign function of a sinus signal $\sin(3t)$
\end{itemize}
with $t$ being a time index defined in the interval $[0,4]$ with $L$ samples. Each of the four signals are randomly drawn and used to construct a source matrix $\mathbf{X}$ of size $k \times L$, then zero rows are inserted randomly, such that $\mathbf{X} \in \mathbb{R}^{N \times L}$. 
The mixing matrix $\mathbf{A}$ of size $M \times N$ is randomly generated from a Gaussian distribution. 
By multiplying the source matrix and the mixing matrix the measurement matrix $\mathbf{Y}$ is achieved.
The deterministic data set then consist of $\{ \mathbf{Y}, \mathbf{X}, \mathbf{A} \}$.
In figure \ref{fig:simple} the first deterministic data set is illustrated by the source signals plotted in the top and the measurement signal $\mathbf{Y}$ plotted in the bottom. This illustrates how the sources signal are transformed by the mixing matrix $\mathbf{A}$.
\begin{figure}[H]
\centering
\includegraphics[scale=0.5]{figures/ch_6/simple_data.png}
\caption{Visualization of the source signals $\mathbf{X}$ in comparison to the measurements $\mathbf{Y}$ from the deterministic data set specified by $N = 5, M = 3$, $k = 4$ and $L=1000$.}
\label{fig:simple}
\end{figure}
\noindent

\subsection{Stochastic Data Set}
The purpose of this second kind of data is to resemble EEG measurements for which the model is intended. Here different data sets are simulated depending on the chosen specifications of $N$, $k$, $M$ and $L$. 
Every data set is constructed based on four different autoregressive processes of various order, each process representing one source signal:
\begin{itemize}
\item[-] $x_{t}^{1} = \sum_{i=1}^{2} \phi_i x_{t-i}^{1} + w_t^{1}$
\item[-] $x_{t}^{2} = \sum_{i=1}^{2} \zeta_i x_{t-i}^{2} + w_t^{2}$
\item[-] $x_{t}^{3} = \sum_{i=1}^{3} \eta_i x_{t-i}^{3} + w_t^{3}$
\item[-] $x_{t}^{4} = \sum_{i=1}^{4} \xi_i x_{t-i}^{4} + w_t^{4}$
\end{itemize}
where $\boldsymbol{\phi}, \boldsymbol{\zeta}, \boldsymbol{\eta}$ and $\boldsymbol{\xi}$ are different model parameters and $w_t^{j}$ for $j = 1,\hdots ,4$ is white noise, corresponding to process $j$.
$\mathbf{X}$ is constructed by drawing $k$ autoregressive processes randomly among the four, if $k < N$ zero rows are inserted randomly such that $\mathbf{X} \in \mathbb{R}^{N \times L}$  
The mixing matrix $\mathbf{A}$ of size $M \times N$ is, like the previously, generated randomly from a Gaussian distribution.
By multiplying the source matrix and the mixing matrix the measurement matrix $\mathbf{Y}$ is achieved.
The stochastic data set then consist of $\{ \mathbf{Y}, \mathbf{X}, \mathbf{A} \}$. 

One simulation of a stochastic data set is illustrated in figure \ref{fig:AR}. The illustrated data set specified by $N = 5$, $M = 3$, $k = 4$ and $L = 1000$.
\begin{figure}[H]
\centering
\includegraphics[scale=0.5]{figures/ch_6/AR_data.png}
\caption{Visualization of the source signals $\mathbf{X}$ in comparison to the measurements $\mathbf{Y}$ from an stochastic data set specified by $N = 5$, $M = 3$, $k = 4$ and $L=1000$. For simplicity only samples [0:100] are plotted.}
\label{fig:AR}
\end{figure}
\noindent

