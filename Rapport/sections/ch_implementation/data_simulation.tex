\section{Data Simulation}\label{sec:dataset}
To evaluate the performance of the main algorithm as well as the individual stages, synthetic data are simulated with respect to the model $\mathbf{Y} = \mathbf{A}\mathbf{X}$. All data sets are simulated based on the following approach, satisfying the sufficient conditions for recovery, displayed in theorem \ref{th:conditions}.
 
A source matrix $\mathbf{X} \in \mathbb{R}^{N \times L}$ is constructed, such that every non-zero row is sampled individually by some function restricted by having zero mean. By this approach the non-zero rows of $\mathbf{X}$ become close to orthogonal \cite{Balkan2014}, which approximates the first conditions of theorem \ref{th:conditions}.   
Then a mixing matrix $\mathbf{A} \in \mathbb{R}^{M \times N}$ is constructed with identically distributed and independent entries. 
As such the source signals are randomly mixed and the mixing matrix fulfils the second condition of theorem \ref{th:conditions}.
With known $\mathbf{A}$ and $\mathbf{X}$, the measurement matrix $\mathbf{Y} \in \mathbb{R}^{M \times L}$ is simulated according to the model, by the matrix product $\mathbf{Y} = \mathbf{AX}$. Note that the error matrix $\mathbf{E}$ is omitted in this chapter, as noise is not included in the synthetic data.  

Two different kinds of data sets are simulated.
Deterministic data having simple and predictable source signals to ensure a solution and easy visualization.
And stochastic data having randomized and fluctuating source signals to resemble realistic EEG measurements.

Note that each simulated data set fulfils the sufficient conditions for recovery, thus segmentation of the synthetic data is not necessary. 

\subsection{Deterministic Data}\label{subseg_simpledata}
Two different deterministic data sets are simulated, with a different number of zero rows. 
The first is specified by $N = 5$, $k = 4$, $M = 3$ and $L = 1000$. 
That is a source matrix $\mathbf{X}$ with $4$ rows individually generated and $1$ zero row. By the specifications the source matrix $\mathbf{X}$ is mixed into a measurement matrix $\mathbf{Y}$ with $3$ measurements per sample.  
The second deterministic data set is specified by $N = 8$, $k = 4$, $M = 3$ and $L = 1000$. 
That is 3 additional zero rows.
From the specifications the first data set comply to $N \leq \widetilde{M}$ which imply the use of Cov-DL2.
The second data set comply to $N > \widetilde{M}$ and $k \leq \widetilde{M}$ implying the use of Cov-DL1. 
As such it is possible to test both branches of the Cov-DL method. 
     
The four non-zero source signals of $\mathbf{X}$ are defined by the following individual functions, causing the rows to be approximately orthogonal
\begin{itemize}
\item[1.] a sinus signal $\sin(2t)$
\item[2.] a sawtooth signal with period $2 \pi t$
\item[3.] a sinus signal $\sin(4t)$
\item[4.] a sign function of a sinus signal $\sin(3t)$
\end{itemize}
with $t$ being a time index defined in the interval $[0,4]$ with $L$ samples. 
Each of the four signals are randomly drawn and used to construct a source matrix $\mathbf{X}$ of size $k \times L$, then zero rows are inserted randomly, such that $\mathbf{X} \in \mathbb{R}^{N \times L}$. 
The mixing matrix $\mathbf{A}$ of size $M \times N$ is randomly generated from a Gaussian distribution. 
By multiplying the source matrix and the mixing matrix a measurement matrix $\mathbf{Y}$ is simulated.
The resulting deterministic data set then consist of $\{ \mathbf{Y}, \mathbf{X}, \mathbf{A} \}$.

In figure \ref{fig:simple} the first deterministic data set, triggering Cov-DL2, is illustrated by the source signals plotted in the top and the measurement signals plotted in the bottom. 
This illustrates how the source signals are transformed by the mixing matrix $\mathbf{A}$.
\begin{figure}[H]
\centering
\includegraphics[scale=0.5]{figures/ch_6/simple_data.png}
\caption{Visualization of the source signals $\mathbf{X}$ in comparison to the measurement signals $\mathbf{Y}$ from the deterministic data set specified by $N = 5, M = 3$, $k = 4$ and $L=1000$.}
\label{fig:simple}
\end{figure}
\noindent

\subsection{Stochastic Data}\label{sec:stoch_data}
The purpose of this second kind of data is to resemble EEG measurements for which the main algorithm is intended. 
Here different data sets are simulated depending on the chosen specifications of $N$, $k$, $M$ and $L$. 
Every data set is constructed based on four different linear auto-regressive processes of various orders, each process representing one source signal
\begin{align*}
&x_{t}^{1} = \sum_{i=1}^{2} \phi_i x_{t-i}^{1} + w_t^{1} &x_{t}^{2} = \sum_{i=1}^{2} \zeta_i x_{t-i}^{2} + w_t^{2} \\
&x_{t}^{3} = \sum_{i=1}^{3} \eta_i x_{t-i}^{3} + w_t^{3}  &x_{t}^{4} = \sum_{i=1}^{4} \xi_i x_{t-i}^{4} + w_t^{4}
\end{align*}
%\begin{itemize}
%\item[-] $x_{t}^{1} = \sum_{i=1}^{2} \phi_i x_{t-i}^{1} + w_t^{1}$
%\item[-] $x_{t}^{2} = \sum_{i=1}^{2} \zeta_i x_{t-i}^{2} + w_t^{2}$
%\item[-] $x_{t}^{3} = \sum_{i=1}^{3} \eta_i x_{t-i}^{3} + w_t^{3}$
%\item[-] $x_{t}^{4} = \sum_{i=1}^{4} \xi_i x_{t-i}^{4} + w_t^{4}$
%\end{itemize}
where $\boldsymbol{\phi}, \boldsymbol{\zeta}, \boldsymbol{\eta}$ and $\boldsymbol{\xi}$ are different model parameters and $w_t^{j}$ for $j = 1,\dots ,4$ are mutually independent Gaussian distributed white noise coefficients.
The source matrix $\mathbf{X}$ is constructed by drawing $k$ auto-regressive processes, randomly drawn among the four, each of length $L$. 
If $k < N$ zero rows are inserted randomly such that $\mathbf{X} \in \mathbb{R}^{N \times L}$. 
The mixing matrix $\mathbf{A} \in \mathbb{R}^{M \times N}$ is, like previously, generated randomly from a Gaussian distribution.
By multiplying the source matrix and the mixing matrix, the measurement matrix $\mathbf{Y}$ is simulated. 
The stochastic data set then consist of $\{ \mathbf{Y}, \mathbf{X}, \mathbf{A} \}$. 

One simulation of a stochastic data set is illustrated in figure \ref{fig:AR}. 
The illustrated data set is specified by $N = 5$, $M = 3$, $k = 4$ and $L = 1000$.
\begin{figure}[H]
\centering
\includegraphics[scale=0.5]{figures/ch_6/AR_data.png}
\caption{Visualization of the source signals $\mathbf{X}$ in comparison to the measurement signals $\mathbf{Y}$ from a stochastic data set specified by $N = 5$, $M = 3$, $k = 4$ and $L=1000$. For simplicity only samples from $L = 0, \dots, 100$ are visualized.}
\label{fig:AR}
\end{figure}
\noindent

