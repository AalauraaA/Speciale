\section{Data Simulation}\label{sec:dataset}
To test the performance of the algorithms simulated data, corresponding to the model $\textbf{Y}=\textbf{A}\textbf{X}$, is needed. All data sets are simulated based on the following approach, satisfying the sufficient conditions for recovery, displayed in theorem \ref{th:conditions}:
 
A source matrix $\textbf{X}\in \mathbb{R}^{N\times L}$ is constructed, such that each row makes an independent signal which varies over $L$ times samples or a zero row. As such the non zero rows of $\textbf{X}$ are mutually orthogonal, which fulfils the first conditions of theorem \ref{th:conditions}.   
Then a mixing matrix $\textbf{A}\in \mathbb{R}^{M\times N}$ is constructed with equally distributed and independent entries. As such the source signals are randomly mixed and the mixing matrix fulfils the second condition of theorem \ref{th:conditions}.
With known $\textbf{A}$ and $\textbf{X}$ the a dataset $\textbf{Y}\in \mathbb{R}^{M\times L}$ is simulated according to the model, by the matrix product $\textbf{Y}= \textbf{AX}$.  

Two different kinds of data sets are simulated.
One simple data set with simple and predictable source signals to ensure a solution and easy visualisation.
Another data set with randomised and fluctuating source signals to resemble realistic EEG measurements.

\subsection{Simple Data Set}
The simple data set is specified by $N = 5$, $k=4$, $M = 3$ and $L = 100$. That is a source matrix $\textbf{X}$ with $4$ independent signals and $1$ zero row which is mixed into a data set with $3$ measurement per sample.       
The 5 rows of $\textbf{X}$ are defined by 
\begin{itemize}
\item[1.] a sinus signal $\sin(2t)$
\item[2.] zero row
\item[3.] a sawtooth signal with period $2 \pi t$
\item[4.] a sinus signal $\sin(4t)$
\item[5.] a sign function of a sinus signal $\sin(3t)$
\end{itemize}
with $t$ being a time index defined in the interval $[0,4]$ with $L$ samples. Each of the four signals are randomly drawn and used to construct a source matrix $\mathbf{X}$ of size $k \times L$.
The mixing matrix $\mathbf{A}$ of size $M \times N$ is randomly generated from a Gaussian distribution. 
By multiplying the source matrix and the mixing matrix the measurement matrix $\mathbf{Y}$ is achieved.
The simple data set then consist of $\{ \mathbf{Y}, \mathbf{X}, \mathbf{A} \}$.
In figure \ref{fig:simple} the source signals are plotted over the measurement signal $\mathbf{Y}$ to illustrates how the sources signal are transformed by the mixing matrix $\textbf{A}$.
\begin{figure}[H]
\centering
\includegraphics[scale=0.5]{figures/ch_6/simple_data.png}
\caption{Visualization of the source signals $\textbf{X}$ in comparison to the measurements $\mathbf{Y}$ from the simple data set with $M = 3$, $k=4$ and $L=100$.}
\label{fig:simple}
\end{figure}
\noindent

\subsection{Autoregressive Data Set}
The purpose of this second kind of data is to resemble EEG measurements for which the model is intended. Here different data sets are simulated depending on the chosen specifications of $N$, $k$, $M$ and $L$. 
Every data set is constructed based on four different autoregressive processes each representing a source signals:
\begin{itemize}
\item[-] $x_{1}^{t} = \sum_{i=1}^{2} \phi_i x^{t-i} + w_t$
\item[-] $x_{2}^{t} = \sum_{i=1}^{2} \zeta_i x^{t-i} + w_t$
\item[-] $x_{3}^{t} = \sum_{i=1}^{3} \eta_i x^{t-i} + w_t$
\item[-] $x_{4}^{t} = \sum_{i=1}^{4} \xi_i x^{t-i} + w_t$
\end{itemize}
where $\boldsymbol{\phi},\boldsymbol{\zeta},\boldsymbol{\eta}$ and $\boldsymbol{\xi}$ are different model parameters and $w_t$ is white noise.
$\textbf{X}$ is constructed by drawing $k$ AR processes randomly among the four, if $k<N$ zero rows are inserted randomly such that $\textbf{X}\in \mathbb{R}^{N \times L}$  
The mixing matrix $\mathbf{A}$ of size $M \times N$ is, like the previously, generated randomly from a Gaussian distribution.
By multiplying the source matrix and the mixing matrix the measurement matrix $\mathbf{Y}$ is achieved.
The autoregressive data set then consist of $\{ \mathbf{Y}, \mathbf{X}, \mathbf{A} \}$. 

One simulation of an autoregressive data set is illustrated in figure \ref{fig:AR}. The data set specified by $M = 3$, $k = 4$ and $L = 1000$.
\begin{figure}[H]
\centering
\includegraphics[scale=0.5]{figures/ch_6/AR_data.png}
\caption{Visualization of the source signals $\textbf{X}$ in comparison to the measurements $\mathbf{Y}$ from an AR data set specified by $M = 3$, $k=4$ and $L=1000$. For simplicity only samples [0:100] are plotted.}
\label{fig:AR}
\end{figure}
\noindent

