\subsection{Error Measurement}  
To evaluate performance of the algorithms it is evident to look at the differences between the true and estimated matrices, mixing matrix $\mathbf{A}$ and source matrix $\mathbf{X}$ -- which is possible due to the input data being simulated. 
For this task THE mean squared error (MSE) has been chosen. 
The MSE measures the average squared difference between some estimated value and the true value. 
The MSE can be	 written as
\begin{align*}
\text{MSE} = \frac{1}{T} \sum_{i=1}^T (G_i - \hat{G}_i)^2,  
\end{align*}
with $T = \{M, N\}$ the number of rows of $\mathbf{A}$ or $\mathbf{X}$, $G_i = \{ \mathbf{A}_i, \mathbf{X}_i\}$ is a measurement row of the actual matrix and $\hat{G}_i = \{\hat{\mathbf{A}}_i,\hat{\mathbf{X}}_i\}$ is an estimated row of the estimated matrix\todo{vi skal lige være skarpe på at vi faktisk har en sum mere ikke? altså vi tager mSE af hvert række og så tager gennem snit at dem, således at i løber over indgange i rækken(original MSE ikke?) og ikke antal rækker?}.

The MSE is viewed as a measure of the quality of an estimator, in this case of how M-SBL and Cov-DL perform. 
For a large MSE value the estimated matrix/values are dispersed widely around its mean while for a small MSE value the estimated matrix/values is closely dispersed around the mean. 
Usually, a small MSE value indicates a good estimator but the value cannot be to small as this would indicate that the data has been overfitted \todo{Vi skal lige finde en god kilde som siger dette}. 
Therefore, a good MSE and therefore a good performance would be depending on how the data is scattered as widely scattered data may lead to a MSE value not close to zero but it would still be the a good measure for the estimator.
\todo[inline]{MSE: jeg er ikke sikker på jeg forstår det med scatteret data. evt. vi skal beskrive at vi tager rækken det er fejlen pr enkelt signal vi har fundet og så tages gennemsnittet over det, det er selvføglelig mest $\textbf{X}$ det er relevant for. Har vi forholdt os til forskel i amplitude her, det ser den bor fra ikke? hvilket vi jo gerne vil have.\\
det kan have betydning hvad rangen er for hvert enkelt indgang som række, når man tager summen af differenser? hvis en række har meget større range, kan den dominere de andre? }
