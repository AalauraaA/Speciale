\subsection{Error Measurement}  
To evaluate performance of the algorithms it is evident to look at the differences between the true and estimated matrices, mixing matrix $\mathbf{A}$ and source matrix $\mathbf{X}$ -- which is possible due to the input data being simulated. 
For this task the mean squared error (MSE) has been chosen. 
The MSE measures the average squared difference between some estimated value and the true value. 
For $\hat{\textbf{g}}$ being the estimate of the vector $\textbf{g}$ the MSE can be written as 
\begin{align*}
\text{MSE}(\textbf{g},\hat{\textbf{g}}) = \frac{1}{T} \sum_{i=1}^T (g_i - \hat{g}_i)^2,  
\end{align*}
with $T$ being the number of elements in the vector $\textbf{g}$. 

For this project the estimates form a matrix. Here the MSE is computed for each row, which for $\mathbf{X}$ is the estimate of one source signal, then the resulting MSE is the average over all rows. 
For $\mathbf{X}, \hat{\mathbf{X}} \in \mathbb{R}^{N \times L}$ the MSE is written as 
\begin{align*}
\text{MSE}(\mathbf{X},\hat{\mathbf{X}}) = \frac{1}{N} \sum_{i}^{N} \left( \frac{1}{L} \sum_{j=1}^L (\mathbf{X}_{ij} - \hat{\mathbf{X}}_{ij})^2\right).  
\end{align*}
Similarly, the MSE can be written for $\mathbf{A},\hat{\mathbf{A}} \in \mathbb{R}^{M \times N}$.  

The MSE is viewed as a measure of the quality of an estimator, in this case of how M-SBL and Cov-DL perform. 
For a large MSE the estimated values are dispersed widely around its mean while for a small MSE value the estimated matrix/values is closely dispersed around the mean. 
Usually, a small MSE value indicates a good estimator but the value cannot be to small as this would indicate that the data has been overfitted \todo{Vi skal lige finde en god kilde som siger dette}. 
Therefore, a good MSE and therefore a good performance would be depending on how the data is scattered as widely scattered data may lead to a MSE value not close to zero but it would still be the a good measure for the estimator.
\todo[inline]{MSE:måske dette kan uddybes lidt i forhold til hvordan vores målingere opføre sig, ved vi hvad vi kan forvente}
