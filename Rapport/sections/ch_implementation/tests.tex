\section{Verification of Algorithms}
In this section the implementations of Cov-DL and M-SBL are verified separately, based on the MSE between the true and the estimated model parameters.
Remember that the segmentation stage is ignored as the simulated data form one single segment.    

\subsection{Test of Cov-DL}
As seen from the flowchart \ref{fig:flow} Cov-DL takes a measurement matrix $\mathbf{Y}$, $N$ and $k$ as input and returns an estimate $\hat{\mathbf{A}}$ of the mixing matrix $\mathbf{A}$. 
The Cov-DL algorithm is tested on the two simulations of the deterministic data, specified in section \ref{subseg_simpledata}. 

\subsubsection{Cov-DL1}
For measurement matrix $\mathbf{Y}$ specified by $N > \widetilde{M}$ and $k \leq \widetilde{M}$, implying Cov-DL1, the true and estimated values of the mixing matrix $\mathbf{A}$ are plotted in figure \ref{fig:cov1_simple} for visual comparison. 
Note that each matrix is vectorized such that the corresponding entries are compared.  
The resulting $\text{MSE}(\mathbf{A}, \hat{\mathbf{A}})$ is seen below. 
As a reference the MSE is measured between $\mathbf{A}$ and a corresponding estimate being a zero matrix.   
\begin{align*}
\text{MSE}(\mathbf{A}, \hat{\mathbf{A}}) = 1.74 \\
\text{MSE}(\mathbf{A}, \mathbf{0}) = 1.40.
\end{align*}
From figure \ref{fig:cov1_simple} it is seen that the error of the estimate varies significantly for each entry. 
Though, the estimated values are seen to fall within the same range as the true values.
Furthermore, the $\text{MSE}(\mathbf{A}, \hat{\mathbf{A}})$ is fairly small suggesting that the estimate is acceptable. However, a smaller MSE is obtained from the estimate being a zero matrix, which argues against $\hat{\mathbf{A}}$ being an acceptable estimate. 

\begin{figure}[H]
\centering
\includegraphics[scale=0.5]{figures/ch_6/COV1_simple.png}
\caption{Estimated values of $\hat{\mathbf{A}}$ compared to the true 				values $\mathbf{A}$.}
\label{fig:cov1_simple}
\end{figure}

\subsubsection{Cov-DL2}
For the measurement matrix $\mathbf{Y}$ specified by $N \leq \widetilde{M}$, implying Cov-DL2, the true and estimated values of $\mathbf{A}$ are plotted in figure \ref{fig:cov2_simple} for visual comparison. 
Additionally, $\mathbf{A}_{\text{init}}$ is plotted in the same figure. 
The matrix $\mathbf{A}_{\text{init}}$ is the initial matrix provide to the optimization solver -- a realization of a Gaussian matrix with zero mean and unit variance. 
The resulting MSE values is seen below, again the zero matrix estimate is used as a reference. 
\begin{align*}
\text{MSE}(\mathbf{A}, \hat{\mathbf{A}}) = 3.00 \\
\text{MSE}(\mathbf{A}, \mathbf{0}) = 0.90.
\end{align*}
From figure \ref{fig:cov2_simple} the estimate $\hat{\mathbf{A}}$ shows visual tendencies from the true $\mathbf{A}$. 
However, when it is compared to the initial guess of $\mathbf{A}$, $\mathbf{A}_{\text{init}}$, it is observed that the estimate $\hat{\mathbf{A}}$ have moved further away from the true $\mathbf{A}$ compared to $\mathbf{A}_{\text{init}}$. 
This suggests some flaw within the optimization process. 
By printing the convergence message from the used optimization solver, it is confirmed that the optimization process was found to be terminated successfully. 
With a current cost function value at $0.0$ after 26 iterations. 
This suggests that a global minimum has been found, but the minimum, $\hat{\mathbf{A}}$, does not correspond to the true $\mathbf{A}$. 
To confirm this the following evaluations of the cost function was conducted. 
\begin{align*}
&\text{cost}(\hat{\mathbf{A}}) = 0.0\\
&\text{cost}(\mathbf{A}_{\text{init}}) = 1.64\\
&\text{cost}(\mathbf{A}) = 1.65
\end{align*}
These evaluations ensure that the optimization solver did manage to find the solution that minimizes the cost function. 
By evaluating the cost function with respect to the true $\mathbf{A}$ it is seen that the true mixing matrix is not a global minimizer to the optimization problem. 
This suggests that the optimization problem, derived in section \ref{sec:over_det}, do not fulfil the purpose.
However, it has to be mentioned that this is merely an interesting observation rather than a concluding result, as $\text{cost}(\mathbf{A}) = \text{cost}(\hat{\mathbf{A}})$ is not guaranteed. 
\begin{figure}[H]
\centering
\includegraphics[scale=0.5]{figures/ch_6/COV2_simple.png}
\caption{The initial $\mathbf{A}_{\text{init}}$ and the estimate $\hat{\mathbf{A}}$ compared to the true values $\mathbf{A}$.}
\label{fig:cov2_simple}
\end{figure}

\noindent
Finally, an additional observation can be gained from both figure \ref{fig:cov1_simple} and \ref{fig:cov2_simple}. 
It appears visually to that the issue of the model being invariant towards the order of the rows in the estimate $\hat{\mathbf{A}}$ is not an issue in this example. 
For instance neither of the three most negative values in $\mathbf{A}$ is found to be estimated at a different index. 
As such this potential flaw within the error measurement is not found to contribute to the insufficient results. 

\subsubsection{Summary with respect to verification of Cov-DL}
From the above results it is found that the estimate $\hat{\mathbf{A}}$, especially within the Cov-DL2 branch, can not be considered as a valid estimate of the mixing matrix $\mathbf{A}$. 
The results suggest immediately that the flaw lies within the derivation of the cost function to the optimization problem. More specifically within the assumptions made throughout the derivation concerning the relation between $\mathbf{A}, \mathbf{D}$ and $\mathbf{U}$. 
However, in general three scenarios can be considered. 
The occurrence of a mistake with respect to the implementation, a misinterpretation of the source \cite{Balkan2015} leading to wrong implementation or lastly the method do not work as claimed by the source. 
In order to investigate the source of the insufficient results, the main attribute would be a thorough step by step evaluation of the implementation. 
Elements of special interest could be the found $\mathbf{D}$ relative to $\mathbf{A}$ and the amount of noise resulting from the rows of $\mathbf{X}$ being close to orthogonal. 
A different aspect could be the assumption of the optimization problem \eqref{eq:Cov_DL2} being convex without further investigation regarding the truthfulness of this assumption. 
The inconsistent results might suggest the optimization problem might not be convex. 
However, the fact that the optimization is terminated successfully with a cost equal to zero supports the existence of a global minimum.

Due to the time limitation of the thesis, the described investigation towards the source of the error is not conducted. 
It is concluded that the estimate of $\mathbf{A}$ is not valid.
Hence, it will not be used as an input to the next stage of the main algorithm, M-SBL. 
This conclusion suggests that some alternative to the estimate must be considered. 
This is discussed further in section \ref{sec:test_base}.

\subsection{Test of M-SBL}
From the flowchart \ref{fig:flow} it seen that the M-SBL algorithm takes the estimated mixing matrix $\hat{\mathbf{A}}$ and measurement matrix $\mathbf{Y}$ as input. however, in order to not, let the performance of Cov-DL affect the result of M-SBL the true mixing matrix $\mathbf{A}$ is used as an input throughout this section, along with the corresponding $\mathbf{Y}$. 
The implementation is first tested on a deterministic data set specified by $M = N = k = 4$ and $L=1000$. 
This result will serve as a reference, showing the best possible performance due to the system having an equal number of equations and unknowns, where a unique solution exists.
The resulting estimate is seen in figure \ref{fig:M-SBL_simple0}. 
It is seen that the source signals are estimated exact, with MSE$(\mathbf{X}, \hat{\mathbf{X}}) = 0$. 
\begin{figure}[H]
\centering
\includegraphics[scale=0.5]{figures/ch_6/M-SBL_simple0.png}
\caption{Estimated values of $\hat{\mathbf{X}}$ compared to the true values $\mathbf{X}$. From deterministic data set specified by $M = N = k = 4$ and $L=1000$ given the true mixing matrix $\mathbf{A}$.}
\label{fig:M-SBL_simple0}
\end{figure}
\noindent
Now the desired case of $M < N$ is considered. 
Two tests are performed on the same two deterministic data sets, as used in the previous section, specified by $M = 3$, $k = 4$, $L=1000$ and respectively $N = 5$ and $N = 8$.

The estimate $\hat{\mathbf{X}}$ is visualized in figure \ref{fig:M-SBL_simple1} and \ref{fig:M-SBL_simple2}. 
The zero rows of the estimate $\hat{\mathbf{X}}$ is not visualized and therefore the figures do not visualize the exact localization of the source signals.
\begin{figure}[H]
\begin{widepage}
    \begin{minipage}[t]{.45\textwidth}
    	\centering
		\includegraphics[scale=0.45]{figures/ch_6/M-SBL_simple1.png}
		\caption{Estimated values of $\hat{\mathbf{X}}$ compared to the true 					values $\mathbf{X}$. From deterministic data set specified by $N = 5$, $M = 3$, $k = 4$ and $L = 1000$ and given the true mixing matrix $\mathbf{A}$.}
		\label{fig:M-SBL_simple1}
    \end{minipage} 
    \hfill
    \begin{minipage}[t]{.45\textwidth}
        \centering
		\includegraphics[scale=0.45]{figures/ch_6/M-SBL_simple2.png}
		\caption{Estimated values of $\hat{\mathbf{X}}$ compared to the true values $\mathbf{X}$. From deterministic data set specified by $N = 8$, $M = 3$, $k = 4$ and $L = 1000$ and given the true mixing matrix $\mathbf{A}$.}
		\label{fig:M-SBL_simple2}
    \end{minipage}
\end{widepage}
\end{figure}
\noindent
The resulting MSE between the true $\mathbf{X}$ and the estimate $\hat{\mathbf{X}}$ from figure \ref{fig:M-SBL_simple1} with $N = 5$, becomes 
\begin{align*}
\text{MSE}(\mathbf{X}, \hat{\mathbf{X}}) = 0.13.
\end{align*}
From figure \ref{fig:M-SBL_simple1} it is seen that all four source signals are recovered at the right locations relative to the removal of the zero rows from $\mathbf{X}$ and $\hat{\mathbf{X}}$. 
As suggested by the achieved MSE the estimate is not exact, but it is clear that the estimates, by looking at figure \ref{fig:M-SBL_simple1}, manage to follow the right pattern of the true signals. 

The resulting MSE between the true $\mathbf{X}$ and the estimated $\hat{\mathbf{X}}$ from figure \ref{fig:M-SBL_simple2} with $N = 8$ thus more sparse, becomes 
\begin{align*}
\text{MSE}(\mathbf{X}, \hat{\mathbf{X}}) = 0.162. 
\end{align*}
From figure \ref{fig:M-SBL_simple2} it is again seen that the source signals are recovered at the right locations. 
However visually the estimates appear slightly more imprecise. 
This indicates that the implementation of M-SBL manages to locate and estimate the source signals, however the increased zero rows improve the chance of dislocation and it decrease the accuracy of the estimate.     

\subsubsection*{Possibilities of $N=k$}
Due to the problem statement in chapter \ref{ch:problemstatement} it is an issue that $k$ has to be known a priori, in order to estimate $\mathbf{A}$ and $\mathbf{X}$. 
A short discussion in section \ref{subsec:kestimate}, describes how $k$ can be estimated within the M-SBL algorithm. 
However, one still needs to provide $k$ in order to estimate $\mathbf{A}$, thus a qualified estimate of $k$ can not be avoided. 

Similar to $k$, the maximum number of active sources $N$ is unknown in practice as it is described in chapter \ref{ch:motivation}. 
The difference between $k$ and $N$ defines the number of zero rows in $\mathbf{X}$.
During the estimation of $\mathbf{X}$ the localization of the non-zero rows are, in general, significant in order to minimize the MSE. 
However, the fact that the true $N$ can not be known for EEG measurements weakens the argument for focusing on the localization rather than only focusing on the value estimation of the source signals. 
When considering the linear system, $\mathbf{Y} = \mathbf{AX}$, which the model is built upon, $\mathbf{Y}$ does not change by removing the zero rows of $\mathbf{X}$ and the corresponding columns in $\mathbf{A}$.

From this it can be argued that $N = k$ is a sufficient estimate of $N$. 
However, remember from chapter \ref{ch:M-SBL} that the existence of a solution is limited to $N = k \leq \widetilde{M}$.

Now consider the effect of letting $N = k$ within the M-SBL algorithm. 
Here it is only the estimation of the support set which is eliminated, as non zero rows will occur. 
Figure \ref{fig:M-SBL_simple3} shows the estimated source signals for a simulation of the deterministic data set now specified by $N = k = 4$, $M = 3$ and $L = 1000$. 
\begin{figure}[H]
\centering
\includegraphics[scale=0.5]{figures/ch_6/M-SBL_simple3.png}
\caption{Estimated values of $\hat{\textbf{X}}$ compared to the true 				values $\textbf{X}$. From deterministic data $\textbf{Y}$ specified by $N=k=4$, $M = 3$ and $L=1000$  and the true mixing matrix $\mathbf{A}$.}
\label{fig:M-SBL_simple3}
\end{figure}
\noindent
The resulting MSE becomes
\begin{align*}
\text{MSE}(\mathbf{X}, \hat{\mathbf{X}}) = 0.124
\end{align*}
From the above discussion and the results in figure \ref{fig:M-SBL_simple3} it is confirmed that letting $N = k$ has no disadvantage when a correct localization of the source signal is not a priority. 
It is chosen that $N = k$ will be used throughout the thesis. 

\subsection{Test on Stochastic Data}\label{sec:testMsbl_stoch}
The M-SBL algorithm is now tested on two stochastic data sets which resembles real EEG measurements. 
The first stochastic data set is simulated with specifications $N = k = 8$, $M = 6$, $L=1000$. 
The resulting estimate is visualized in figure \ref{fig:AR1} and the MSE becomes 
\begin{align*}
\text{MSE}(\mathbf{X}, \hat{\mathbf{X}}) = 1.1.
\end{align*} 
The second stochastic data set is simulated with specifications $N = k =16$, $M = 6$ and $L=1000$. 
This tests the capabilities of the implementation of M-SBL when the distance between $M$ and $N$ is enlarged. 
The performance relative to the relation between $N$ and $M$ is further investigated for the main algorithm in section \ref{sec:test_base}.
The resulting estimate is plotted in figure \ref{fig:AR2} and the MSE becomes 
\begin{align*}
\text{MSE}(\mathbf{X}, \hat{\mathbf{X}}) = 3.652. 
\end{align*}  
\begin{figure}[H]
\begin{widepage}
    \begin{minipage}[t]{.45\textwidth}
    	\centering
		\includegraphics[scale=0.5]{figures/ch_6/M-SBL_AR1.png}
		\caption{Estimated values of $\hat{\mathbf{X}}$ compared to the true values $\mathbf{X}$. From a stochastic data set specified by $N = k = 8$, $M = 6$ and $L=1000$ and given the true mixing matrix $\mathbf{A}$.}
		\label{fig:AR1}
    \end{minipage} 
    \hfill
    \begin{minipage}[t]{.45\textwidth}
        \centering
		\includegraphics[scale=0.5]{figures/ch_6/M-SBL_AR2.png}
		\caption{Estimated values of $\hat{\mathbf{X}}$ compared to the true 					values $\mathbf{X}$. From a stochastic data set specified by $N = k = 16$, $M = 6$ and $L=1000$ and given the true mixing matrix $\mathbf{A}$.}
		\label{fig:AR2}
    \end{minipage}
\end{widepage}
\end{figure}
\noindent
From figure \ref{fig:AR1} it is visually confirmed that the implementation of M-SBL manages to sufficiently recover the stochastic source signals $\mathbf{X}$. 
Some source signals are nearly perfectly estimated while other are having minor differences. 
From figure \ref{fig:AR2} the same tendency is seen, though more visual flaws appears compared to figure \ref{fig:AR1}. 
This result suggests that a bigger distance between $M$ and $N$ results in a worse performance from the M-SBL algorithm.        

%\subsection{Model Fitting}
Two model variables are now considered with the purpose of improving the performance of the COV-DL algorithm. The initial A given to the optimization problem \eqref{eq:Cov_DL2} within COV-DL2, and the segmentation size used within the covariance domain.   
\subsubsection*{Initial $\textbf{A}$}
Consider the COV-DL algorithm in the case where the system transformed into the covariance domain results in an overdetermined system. In this case the COV-DL2 branch of the algorithm is used. 
When estimating the mixing matrix $\mathbf{A}$ a matrix $\mathbf{D}$ is used in the process. For the over-determined system \ref{sec:over_det} $\mathbf{D}$ is found by solving the optimisation problem \eqref{eq:Cov_DL2} with respect to $\hat{\textbf{A}}$. To solve the optimization problem an initial $\mathbf{A}_{\text{ini}}$ is given. The choice of this initial $\mathbf{A}_{\text{ini}}$ may affect how the good an estimate the recovered mixing matrix $\mathbf{A}$ is.
Three different choices of $\mathbf{A}_{\text{ini}}$ are considered:
\begin{itemize}
\item[-] A matrix $\mathbf{A1}$ drawn from a continuous uniform distribution in the half-open interval $[0.0, 1.0)$
\item[-] A matrix $\mathbf{A2}$ drawn from a uniform distribution in the half-open interval $[-1.0, 1.0)$
\item[-] A matrix $\mathbf{A3}$ drawn from a Gaussian distribution with mean 0 and variance 1
\end{itemize}
The test of different initial $\mathbf{A}_{\text{ini}}$ is performed on the 
an AR data set specified by $N=5$ $M = 3$, $k = 4$ and $L = 1000$. 
The MSE of the three tests are seen in table \ref{tab:iniA} 
\begin{table}[H]
\centering
\begin{tabular}{|l|l|l|l|} 
\hline
                          & \textbf{A1} & \textbf{A2} & \textbf{A3} \\
\hline $\text{MSE}_{\mathbf{A}}$ &   2.15          & 2.00            & 1.88\\
\hline           
\end{tabular}
\caption{Resulting MSE for varying initial A, used within COV-DL2.}
\label{tab:iniA}
\end{table}
\todo[inline]{is it okay that we do not include the mse of X here, I don't think is any reasoning to do it, but it should be check later whether the error of A follows the error of X as it is asumped at this moment.}
From the results it is seen that \textbf{A3} achieves the lowest MSE, thus an Gaussian distributed initial $\textbf{A}$ will be used.    

\subsubsection{Segmentation in Covariance domain}
For the Cov-DL algorithm when estimating the mixing matrix $\mathbf{A}$ the measurement matrix is transformed into the covariance domain as part of the recovering process. During the transformation the measurement matrix $\mathbf{Y}$ is divided into segments consisting of $L_s$ samples each.
During this test different numbers of samples within a segment will be tested to see how this affect the performance of the algorithm.

The autoregressive data sets,  $M = 3$, $k = 4$ and $L = 1000$ samples and with respectively $N = 5$ and $N=8$, will be used for the testing. With $k = 4$ the number of segments can not be less than the number of sources. For this system each segment can have maximum $L_s = 200$ corresponding to minimum 5 segments. 

For the test, six different number of samples within each segments will be tested, $L_s = \{ 10, 20, 30, 50, 100, 150, 200 \}$. Furthermore, each $L_s$ will be run 10 times and the output of the test will be average such that each $L_s$ have one average MSE.

\begin{table}[H]
\centering
\begin{minipage}{.45\textwidth}
\centering
\begin{tabular}{|c|c|c|c|c|c|c|c|}
\hline 
& 10 & 20 & 30 & 50 & 100 & 150 & 200 \\ 
\hline 
Average MSE of $\hat{\mathbf{A}}$ & 1.86 & 2.14 & 2.21 & 2.03 & 2.31 & 1.89 & 2.06 \\ 
\hline
\end{tabular} 
\caption{MSE values from measurement specified by $N=5$, $M = 3$, $k = 4$ and $L = 1000$ achieved from the used of Cov-DL2}
\label{tab:seg1}
\end{minipage}
\\
\begin{minipage}{.45\textwidth}
\centering
\begin{tabular}{|c|c|c|c|c|c|c|c|}
\hline 
& 10 & 20 & 30 & 50 & 100 & 150 & 200 \\ 
\hline 
Average MSE of $\hat{\mathbf{A}}$ & 1.29 & 1.37 & 1.22 & 1.16 & 1.46 & 1.20 & 1.32 \\ 
\hline
\end{tabular} 
\caption{MSE values from measurement specified by $N=8$, $M = 3$, $k = 4$ and $L = 1000$ achieved the used of Cov-DL1.}
\label{tab:seg2}
\end{minipage}
\end{table}
\noindent
Overall, in table \ref{tab:seg1} and \ref{tab:seg2} there is not a big variation between the MSE values of both data sets. One could argument that the number of samples within each segment does not affect the performance of the algorithms as much and therefore is more free choice. 






 
