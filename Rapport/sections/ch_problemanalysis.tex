\chapter{Problem Analysis}
This chapter examines existing literature concerning source localisation from EEG measurements. At first a motivation for the problem is given, considering especially the application within the hearing aid industry. Further, the state of the art methods are presented follow by a description of the desired contribution. 

\section{Motivation}
(Hvad er EEG)\\
EEG recordings or measurements are used within medicine as an imaging technique measuring electric signals on the scalp, caused by brain activity. \\
\\
The brain consist of a enormous amounts of cells, called neurons. These neurons are mutually connected in neural nets and when a neuron is activated, for instance by some physical stimuli, local current flows are produced\cite{fundamentalEEG}. As such the neurons are somehow communicating(?). \\
\\
The EEG measurements are provided by a varies number of metal electrodes referred to as sensors, places on the head surface reading the electrical signals which are masively amplified and displayed on the computer as a sum of sinusoidal waves relative to time.
It takes a large amount of active neurons to generate an electrical activity that is recordable on the scalp as the current then have to penetrate the skull, skin and several other layers.  \\
From this it is clear that the measurements from a single sensor do not correspond to the activity of a single neuron in the brain, but rather a collection of many activities. Here the same neuron activities can be measured by two or more sensors. Furthermore, interfering signals can occur resulting from physical movement of e.g. eyes and jawbone\cite{fundamentalEEG}.\\
\\
The waves resulting from EEG have been classified into four groups...use photo from EEG signal processing p. 12 

(Hvad bruges det til)\\


(Hvad er problemet, localisation)\\

(Anvendelse i praksis)

evt. lave subsections her..

\section{Related Work and Our Contribution } 





 
