\chapter*{Danish Summary}
%\addcontentsline{toc}{chapter}{Danish Summary}
% Hvad er det identificerede problem? Indiker afhandlingens forskningsmål, problemformulering og forskningsspørgsmål. Hvis du har udarbejdet hypoteser i afhandlingen, skal disse indikeres her.
I dette kandidatspeciale undersøges der hvordan man kan lokalisere og identificere den direkte hjerne aktivitet ud fra EEG-målinger. Dette omtales som at løse det inverse EEG-problem hvor det ønskes at finde en matrix bestående af hjerneaktiviteten – sources – samt en mixing matrix som tilføjer støj til sources. Det ønskes at løse det inverse EEG-problem for EEG-målinger hvor flere sources er tilstede end sensorer.  Der vil tages udgangspunkt i to ”state of the art” metoder, covariance-domain dictionary learning (Cov-DL) og multiple sparse Bayesian learning (M-SBL), til at løse dette problem. Ud fra disse to valgte metoder vil deres reproducerbarhed blive undersøgt for at se om samme konklusion opnås, at disse metoder effektivt løser det inverse problem. I dette projekt vil disse to metoder indgå i en samlet metode, Cov-DL vil bruges til at finde mixing matricen og M-SBL benytter sig af den fundne mixing matrix til at finde source matricen. Sekundært vil der undersøges i at modificere den samlede metode ud fra et praktisk perspektiv.
Ved implementering af den samlede metode vil de enkle metoder blive testet og analyseret på simuleret data for at undersøge om metoden lykkes. Til dette vil de fundne estimater blive sammenlignet med de rigtige estimater ved brug af mean-square-error (MSE). For Cov-DL lykkes det ikke at genskabe den samme konklusion som den videnskabelige artikel bag metoden så denne har ikke en tilstrækkelig grad af reproducerbarhed. Den anden metode, M-SBL, findes vellykket når den sande mixing matrix benyttes. Herfra kan det konkluderes, at den tilsvarende videnskabelige artikel tilvejebragte en tilstrækkelig grad af reproducerbarhed.
For at kunne anvende den samlede metode på rigtig EEG-måling så vælges der at erstattet Cov-DL med en fast mixing matrix som er fundet ud fra empiriske tests. Med denne ændring og ud fra tests af den samlede metode så forventes det ikke at der opnås en tilstrækkelig identificering og lokalisering af source matricen. Da de sande estimater ikke kendes for de rigtige EEG-målinger, så introduceres independent component analysis (ICA) og dens estimater da dette allerede er en eksisterende og succesfuld metode for det inverse EEG-problem – for systemer med et lige antal sensorer og sources. Estimaterne fra den samlede metode sammenlignes med estimaterne fra ICA. For det lige system ses en tilstrækkelige identificering af source matrix med den samlede metode men for systemer med flere sources end sensorer fejler metoden. En alternativ test blev udført som et forsøg på at analysere det fundne source matrix fra et andet perspektiv. Resultatet af dette ændrer ikke den tidligere konklusion.
For det praktiske perspektiv undersøges det om man kan identificere de aktive sources i forhold til det samlede antal sources igennem empiriske tests. Dog findes resultatet ikke tilstrækkelige og upålideligt.
