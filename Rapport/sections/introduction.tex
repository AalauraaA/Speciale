\chapter*{Introduction}\label{ch:introduction}
\addcontentsline{toc}{chapter}{Introduction}
The topic of this thesis arises from the increasing use of electroencephalographic measurements for a wide range of scientific purposes, especially within the medical field. 
By sensors placed on the head, an electroencephalography captures a mixture of electric signals caused by activity within the brain. 
One essential issue concerning an electroencephalography is to recover the original source signals which were released inside the brain. 

The need for source recovery is confirmed by studies showing how analysis performed on electroencephalographic measurements differs significantly from similar analysis performed directly on the original sources \cite{Friston2002}.
One area of application, where the use of the recovered source signals has shown potential, is the hearing aid industry. Here it is of special interest to recover the source signals from only few sensors, which potentially can be placed within a hearing aid. 

Consider the issue of source recovery from a mathematical perspective. Here the electroencephalographic measurements can be modeled by a linear system of equations. From such model it is possible to recover a limited number of source signals under certain conditions. However, it is a general acknowledged issue that the true number of source signals inside the human brain is unknown.
The task complexity of recovering the source signals from the linear system is increased in cases where the number of sources exceeds the number of sensors providing the measurements.

This thesis explores a state of the art mathematical method for source recovery, embracing the case of more sources than sensors. 
Overall this method, published in 2015, consist of two steps. That is receptively to recover the mixing process that the source signals have undergone and then recover the source signals.
The two steps originate from two different approaches considering the mathematical orientation. 
The main goal of the thesis is to study the two methods with respect to proposing a united algorithm, to be applied on electroencephalographic measurements. The purpose is to support the current results of source recovery from electroencephalographic measurements of few sensors. Furthermore, the issue of the unknown number of active source signals is considered from a perspective of practical application. 

The thesis consists of a motivational part, introducing electroencephalography and the potential use within research. Existing literatures are examined, with respect to identification of state of the art approaches within source signal recovery. The motivational part is concluded by the problem statement specifying the objective of the thesis.
Next is the theoretical part. The system model is specified and the solution approach is presented based on existing methods. This is followed by an extensive study of the required theory. 
The practical aspect of the thesis includes an implementation of the proposed solution to be tested on both synthetic data and new electroencephalographic measurements. 
Finally, discussion and conclusion upon the achieved results are presented followed by a consideration upon further studies. 
