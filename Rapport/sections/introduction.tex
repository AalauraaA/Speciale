\chapter*{Introduction}\label{ch:introduction}
\addcontentsline{toc}{chapter}{Introduction}
Introduktion til hele projektet, skal kunne læses som en appetitvækker til resten af rapporten, det vi skriver her skal så uddybes senere. Brug dog stadigvæk kilder.

\begin{itemize}
\item[-] kort intro a EEG og den brede anveldelse, anvendelse indenfor høreapperat. 
\item[-] intro af model, problem med overbestemt system
\item[-] Seneste forslag til at løse dette 
\item[-] vi vil efterviser dette og udvide til realtime tracking
\item[-] opbygningen af rapporten
\end{itemize}
seneste tanker:\\
appetitvækker, der ligger større fokus på hvad vi arbejder med og hvordan rapporten er bygget op omkring det. -- skal resultater fremgå?   \\ \\
---------------------------klade-----------------------------\\ \\

The problem that is addressed throughout this thesis arise from the increasing use of electroencephalographic measurements for a wide range of scientific purposes, especially within the medical field. 
An electroencephalography captures electric signals caused by activity within the brain. The signals from the brain is recorded over time by multiple electrodes placed on the scalp.

One essential issue concerning an electroencephalography is to  
find the exact sources of the captured brain activity. This is of interest when studying correlation among activities in different parts of the brain, referred to as functional integration.  
The recorded signal from one electrode is basically a mixture of electric signals released from a various number of active neurons mutually communicating within the brain. 
Furthermore, this mixture is distorted as it travels through the scalp. This is confirmed by studies showing how analysis performed on electroencephalographic measurements differs significantly from analysis performed on source level\cite{Friston2002}.\\



              

