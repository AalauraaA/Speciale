\chapter*{Preface\markboth{Preface}{Preface}}\label{ch:preface}
\addcontentsline{toc}{chapter}{Preface}
This master thesis has been written by Trine Nyholm Kragh and Laura Nyrup Mogensen, 4th semester students of the Master programme in Mathematical Engineering at Aalborg University. 
The students want to thank their supervisors Jan Østergaard from Department of Electronical Systems and Rasmus Waagepetersen from Department of Mathematical Sciences for their guidance and help to conduct this study. 
Furthermore, a thank you goes to PhD student Payam Shahsavari Baboukani for sharing the real EEG scalp data base and corresponding guidance. 

The theme of this master thesis is mathematical modeling of EEG measurements for source recovery based on an existing method, aiming to support the existing results.
As for prerequisites, the reader is expected to be familiar with linear algebra, optimization theory and probability theory.
Citations are provided in the form [citation number] or [citation number, page number] especially for books, with every source having a unique citation number to be found the bibliography. 
In appendix \ref{app:Cov-DL}, \ref{app:B} and \ref{app:ICA} supplementary theory for the presented methods can be found. 
In appendix \ref{App:code} a list and outlines of the scripts developed during this thesis can be found. The scripts are written in Python 3.6 and are available at \url{https://github.com/TrineNyholm/Enclosure_Mattek10b_thesis_2020.git}.



\vspace{\baselineskip}\hfill Aalborg University, \today
\vfill\noindent
\begin{minipage}[b]{0.45\textwidth}
 \centering
 \rule{\textwidth}{0.5pt}\\
  Trine Nyholm Kragh\\
 {\footnotesize <trijen15@student.aau.dk>}
\end{minipage}
\hfill
\begin{minipage}[b]{0.45\textwidth}
 \centering
 \rule{\textwidth}{0.5pt}\\
  Laura Nyrup Mogensen\\
 {\footnotesize <lmogen15@student.aau.dk>}
\end{minipage}
